\documentclass{Phyport}
\usepackage{longtable}
\usepackage{multirow}      % 用于跨行
\usepackage{float} % 在导言区添加
\usepackage{amsmath}  % 必须加这一行!
\usepackage{ctex}     % 如果要显示中文,加上这一行

\usepackage{hyperref}
\hypersetup{
    colorlinks=true,
}

\exname{光速测量} %实验名称
\extable{6} %实验桌号
\instructor{乐静飞老师} %指导教师
\class{-} %班级
\name{-} %姓名
\stuid{-} %学号

\nyear{2025} %年
\nmonth{10} %月
\nday{27} %日
\nweekday{一} %星期几

\redate{} %如有实验补做,补做日期
\resitu{} %情况说明:

\begin{document}
\setcounter{page}{0}
\makecover

\section{预习报告(10分)}
(注:将已经写好的“物理实验预习报告”内容拷贝过来)

\subsection{实验综述(5分)}
(自述实验现象、实验原理和实验方法,包括必要的光路图、电路图、公式等。不超过500字。)

\textbf{光速测量实验}旨在利用\textbf{光调制法}来测定光在空气中的传播速度,从而理解光信号在电光系统中的延迟规律。该方法通过调制光强信号、测量信号传播的时间延迟来计算光速,体现了光学与电子测量技术的结合。与传统的旋转镜法相比,调制法具有\textbf{结构简单、可重复性强、测量精度高}等优点。

\textbf{实验现象:}

实验中,光源经过调制器输出周期性变化的光强信号,通过平行光管和反射镜后被光电接收器接收。当改变反射镜与接收器间的距离时,示波器上两路信号的相位差随之变化,可以观察到波形延迟增大或减小的现象。通过测量时间延迟的变化,即可定量求得光在空气中的传播速度。

\textbf{实验原理:}

\textbf{光速测量仪器原理示意图:}

\begin{figure}[H]
    \centering
    \includegraphics[width=0.5\textwidth]{../images/测量光速.drawio.png}
    \caption{测量光速}
    \label{fig:测量光速}
\end{figure}

调制光强可表示为:
\[
I = I_0 + \Delta I_0 \cos(2\pi f t)
\]

经光电探测器转换为电信号:
\[
U_1 = U_1 \cos(2\pi f t), \quad
U_2 = U_2 \cos\!\left[2\pi f (t - \Delta t)\right]
\]

其中,光在空气中传播的时间差为:
\[
\Delta t = \frac{L}{c}
\]

两信号的相位差为:
\[
\Delta \phi = 2\pi f \Delta t = 2\pi f \frac{L}{c}
\]

若调节接收器位置使相位差变化 $2\pi$,则传播距离变化 $\Delta L$ 满足:
\[
c = f \frac{\Delta L}{\Delta \phi / 2\pi} = \frac{\Delta L}{\Delta t}
\]

因此,通过测量距离变化与时间延迟之间是关系即可计算光速。

而在本实验中,光速最终的计算公式为:
$$
c=\frac{2(S_2-S_1)}{\Delta t'} \cdot \frac{\nu}{\nu'}
$$

\textbf{实验装置与方法:}

实验系统由光速测量仪(含光源、调制器、反射镜、光电接收器)、平行光管、导轨及示波器组成。通过调节折光器和接收器的位置,利用示波器的 Track 功能测得不同距离下的时间延迟 $\Delta t$,并通过直接计算或作图法求出光速。

\subsection{实验重点(3分)}
(简述本实验的学习重点,不超过100字。)

本实验的学习重点在于掌握利用\textbf{光调制法}测量光速的基本思路和操作方法,理解\textbf{调制信号、相位差与时间延迟}之间的定量关系。我们需要熟悉示波器的双通道测量与 \texttt{Track} 功能,能够准确读取光信号与参考信号的时间差;同时应掌握利用作图法求斜率计算光速的过程,并理解实验中\textbf{信号调制、波形相移、传播距离与光速计算}之间的物理联系,从而培养光电测量与数据处理的综合能力。

\subsection{实验难点(2分)}
(简述本实验的实现难点,不超过100字。)

本实验的主要难点在于\textbf{时间延迟的精确测量与光路调节}。由于信号频率较高,示波器上两路波形相位差微小,读数稍有偏差即可造成较大误差;同时光路需严格准直,否则接收信号幅值减弱,影响时间差测定。此外,实验中多次移动反射镜时需保持光轴一致,避免机械误差累积。我们需要通过反复调节与求平均值,来保证测量的稳定性与结果的可靠性。

\section{原始数据(20分)}
(将有老师签名的“自备数据记录草稿纸”的扫描或手机拍摄图粘贴在下方,完整保留姓名,学号,教师签字和日期。)

\begin{figure}[H]
    \centering
    \includegraphics[width=0.85\textwidth]{../images/originaldata.jpg}
    \caption{original data}
    \label{fig:original_data}
\end{figure}

\section{结果与分析(60分)}
\subsection{数据处理与结果(30分)}
(列出数据表格、选择适合的数据处理方法、写出测量或计算结果。)

\textbf{1.第一组实验数据整理}

\textbf{其中,$\nu = 100 MHz$, $\nu' = 459.6KHz$}

\begin{longtable}{|c|c|c|c|c|}
  \caption{不同初始与结束位置数据表} \label{tab:不同初始与结束位置数据表} \\
  \hline
    \textbf{实验次数} & \textbf{初始位置S1/m} & \textbf{终止位置S2/m} & \textbf{$\Delta t'/s $} & \textbf{c/(m/s)}  \\
  \hline
  \textbf{1} & 0.0215 & 0.4726 & $0.680 \times 10^{-6}$ & $2.95 \times 10^{8}$ \\
  \hline
  \textbf{2} & 0.0304 & 0.4854 & $0.690 \times 10^{-6}$ & $2.93 \times 10^{8}$\\
  \hline
  \textbf{3} & 0.0353 & 0.5004& $0.685 \times 10^{-6}$ & $2.91 \times 10^{8}$\\
  \hline
  \textbf{4} & 0.0112 & 0.5055 & $0.730 \times 10^{-6}$ & $2.99 \times 10^{8}$\\
  \hline
  \textbf{5} & 0.0155 & 0.4825 & $0.690 \times 10^{-6}$ & $2.93 \times 10^{8}$\\
  \hline
  \textbf{6} & 0.0255 & 0.5281 & $0.740 \times 10^{-6}$ & $2.88 \times 10^{8}$\\
  \hline
\end{longtable}

计算实验的平均值为:
$$
\overline{c} = 2.93 \times 10^{8} m/s
$$

计算得到相对误差为:$\epsilon_1 = \frac{|2.926-2.998|}{2.998}\times 100\% = 2.4\% $

而样本标准差(无偏估计)为:
\[
s=\sqrt{\frac{1}{N-1}\sum_{i=1}^N (c_i-\bar{c})^2}
\]

故A类不确定度为
\[
u_A=\frac{s}{\sqrt{N}}=\sqrt{\frac{1}{N(N-1)}\sum_{i=1}^N (c_i-\bar{c})^2} = 0.016 \times 10^{8} m/s
\]

故实验测得的光速的修正值为$c = (2.93 \pm 0.016) \times 10^{8} m/s$

\textbf{第二组实验数据整理}

\textbf{此时的$\nu' = 459.7KHz$}

\begin{longtable}{|c|c|c|c|}
  \caption{相同初始位置、不同结束位置数据表} \label{tab:相同初始位置、不同结束位置数据表} \\
  \hline
    \textbf{实验次数} & \textbf{初始位置S1/m} & \textbf{终止位置S2/m} & \textbf{$\Delta t'/s $}  \\
  \hline
  \textbf{1} & 0.0184 & 0.1005 & $0.130 \times 10^{-6}$ \\
  \hline
  \textbf{2} & 0.0184 & 0.1815 & $0.245 \times 10^{-6}$ \\
  \hline
  \textbf{3} & 0.0184 & 0.2645 & $0.365 \times 10^{-6}$ \\
  \hline
  \textbf{4} & 0.0184 & 0.3504 & $0.495 \times 10^{-6}$ \\
  \hline
  \textbf{5} & 0.0184 & 0.4333 & $0.620 \times 10^{-6}$ \\
  \hline
  \textbf{6} & 0.0184 & 0.5163 & $0.735 \times 10^{-6}$ \\
  \hline
\end{longtable}

使用matplotlib对实验数据进行线性拟合,得到拟合结果如下图所示:

\begin{figure}[H]
    \centering
    \includegraphics[width=0.85\textwidth]{../images/output.png}
    \caption{linear fitting result}
    \label{fig:linear fitting result}
\end{figure}

根据光速的计算公式,我们可以知道拟合直线的斜率等于光速的一半,因此通过线性拟合得到光速的测量值为$2.968\times 10^8 m/s$

计算得到相对误差为:$\epsilon = \frac{|2.968-2.998|}{2.998}\times 100\% = 1.00\% $

而根据不确定度的传播,通过线性拟合方法测得的光速的a类不确定度为:
\[
u_a = 2\frac{\nu}{\nu'}\,u_k=1.89\times10^{6}m/s
\]

因此,使用线性拟合方法测得的光速c的修正值为$c = (2.968 \pm 0.019) \times 10^{8} m/s$

\subsection{误差分析(20分)}
(运用测量误差、相对误差或不确定度等分析实验结果,写出完整的结果表达式,并分析误差原因。)

\textbf{相对误差与不确定度的分析在前一部分已经完成,这部分直接讨论误差的来源}

1.实验中,示波器上间隔时间读书月只能以5ns为单位进行读数,导致时间延迟的测量存在较大误差,从而影响光速的计算结果;同时,我们使用的是通过肉眼来判断与x轴的交点。可能会导致获得的时间间隔存在较大的误差,并最终影响光速的计算结果。

2.实验中,光路的调节需要非常精确,任何微小的偏差都会导致接收信号的强度减弱,影响时间延迟的测量精度。此外,反射镜和接收器的位置调整过程中可能存在误差,进一步影响测量结果,如我在实验中,无论怎么调节反射镜的高度,都无法完全使它们处于同一高度,在移动放射镜时可以观察到示波器上波峰的高度仍是会发生一定的变化,从而影响结果的测量与光速的计算。

3.实验环境中的光学元件可能存在一定的质量问题,如反射镜表面的不平整或灰尘等,这些因素都会影响光信号的传输和接收,进而影响时间延迟的测量精度。

4.导轨刻度的误差与我们的读数误差,也可能导致实验结果的不准确。

5.实验中,环境光的干扰也可能影响光电接收器的信号接收,导致测量结果出现偏差;此外光源的稳定性也会影响测量结果。如果光源的强度或频率不稳定,可能会导致接收信号的波形发生变化,从而影响时间延迟的测量精度。

\textbf{从第二组实验数据的线性拟合结果来看,拟合得到的光速值和第一组相比与理论值的偏差更小,说明线性拟合方法在处理实验数据时,能够有效地减少随机误差以及我们操作误差的影响,从而提高测量结果的准确性。}

\subsection{实验探讨(10分)}
(对实验内容、现象和过程的小结,不超过100字。)

光速测量实验使用光调制法测量光速,利用调制光信号在传播过程中产生的时间延迟,结合示波器测得的相位差关系,计算出光在空气中的传播速度。实验过程中观察到两路信号波形随反射镜位置变化而产生明显相移,实验数据经线性拟合结果良好,所得光速约为 $2.97\times10^{8}\,\mathrm{m/s}$,与理论值较为符合,验证了使用调制法来测量光速的可行性与准确性。

从本次实验的结果中我发现,使用测得的数据直接计算光速时,得到的光速值与理论值的偏差较大,而通过线性拟合方法得到的光速值则更接近理论值。这表明线性拟合方法在处理实验数据时,能够有效地减少随机误差以及我们操作误差的影响,从而提高测量结果的准确性。因此,在今后的实验中,我将更加重视数据处理方法的选择,尽量采用能够减小误差影响的方法,以获得更为准确的实验结果。

\section{思考题(10分)}
(解答教材或讲义或老师布置的思考题,请先写题干,再作答。)

\textbf{思考题1:我们要如何克服实验中的可能出现的波形假位移?}

在本实验中,波形假位移主要由于光路未完全准直、信号幅值波动或电子系统的相位延迟造成的。为了克服这个问题,首先应该保证光束严格准直,使光信号在各位置均能稳定入射到接收器;其次,保持调制频率和电源稳定,避免因信号幅值变化导致示波器触发点漂移;此外,可通过调整示波器触发方式与耦合模式,减少电路噪声引起的波形漂移。

重复测量并取平均值也能有效减小假位移带来的误差。

\textbf{思考题2:分析影响实验精度的主要因素}

本实验精度主要受位置测量精度、时间延迟测量精度以及系统稳定性影响。导轨刻度读数误差和反射镜定位误差会直接影响传播距离差 $\Delta S$;示波器的时基分辨率、通道延迟及触发抖动会导致时间差 $\Delta t'$ 的测量精度。此外,调制电路和光电探测器的相位响应、环境温度及空气折射率的变化也会引入系统性的误差。提高实验精度可通过选用高精度的仪器、严格准直光路、增大测距范围以及重复测量取平均值等方法实现。

\textbf{思考题3:描述光速测量的其他方法}

除光调制法外,光速的测量方法还包括\textbf{旋转镜法、旋转齿轮法、迈克耳孙干涉法以及微波谐振腔法}等。

\textbf{旋转镜法}通过测定反射光束因镜面高速旋转产生的角度偏移来计算光的传播时间;

\textbf{旋转齿轮法}利用光束被周期性遮挡的时间差求得光速;

\textbf{迈克耳孙干涉法}通过干涉条纹的变化精确测出光程差;

\textbf{微波谐振腔法}则利用腔体的共振频率与波长关系间接求取光速。

这些方法各具特点,其中干涉法的精度最高,而旋转法更直观,适合宏观演示。

\vspace{3\baselineskip}
\input{注意事项}
\end{document}
