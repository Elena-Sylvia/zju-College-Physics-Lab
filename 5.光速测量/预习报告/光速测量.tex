\documentclass{Preport}
\usepackage{longtable}
\usepackage{multirow}      % 用于跨行
\usepackage{float} % 在导言区添加
\usepackage{amsmath}  % 必须加这一行!
\usepackage{ctex}     % 如果要显示中文,加上这一行

\usepackage{hyperref}
\hypersetup{
    colorlinks=true,
}

\exname{光速测量} %实验名称
\instructor{乐静飞老师} %指导教师
\class{-} %班级
\name{-} %姓名
\stuid{-} %学号

\nyear{2025} %年
\nmonth{10} %月
\nday{27} %日
\nweekday{一} %星期几

\begin{document}
\setcounter{page}{0}
\makecover

\section{预习报告(10分)}
\subsection{实验综述(5分)}
(自述实验现象、实验原理和实验方法,包括必要的光路图、电路图、公式等。不超过500字。)

\textbf{光速测量实验}旨在利用\textbf{光调制法}来测定光在空气中的传播速度,从而理解光信号在电光系统中的延迟规律。该方法通过调制光强信号、测量信号传播的时间延迟来计算光速,体现了光学与电子测量技术的结合。与传统的旋转镜法相比,调制法具有\textbf{结构简单、可重复性强、测量精度高}等优点。

\textbf{实验现象:}

实验中,光源经过调制器输出周期性变化的光强信号,通过平行光管和反射镜后被光电接收器接收。当改变反射镜与接收器间的距离时,示波器上两路信号的相位差随之变化,可以观察到波形延迟增大或减小的现象。通过测量时间延迟的变化,即可定量求得光在空气中的传播速度。

\textbf{实验原理:}

\textbf{光速测量仪器原理示意图:}

\begin{figure}[H]
    \centering
    \includegraphics[width=0.5\textwidth]{../images/测量光速.drawio.png}
    \caption{测量光速}
    \label{fig:测量光速}
\end{figure}

调制光强可表示为:
\[
I = I_0 + \Delta I_0 \cos(2\pi f t)
\]

经光电探测器转换为电信号:
\[
U_1 = U_1 \cos(2\pi f t), \quad
U_2 = U_2 \cos\!\left[2\pi f (t - \Delta t)\right]
\]

其中,光在空气中传播的时间差为:
\[
\Delta t = \frac{L}{c}
\]

两信号的相位差为:
\[
\Delta \phi = 2\pi f \Delta t = 2\pi f \frac{L}{c}
\]

若调节接收器位置使相位差变化 $2\pi$,则传播距离变化 $\Delta L$ 满足:
\[
c = f \frac{\Delta L}{\Delta \phi / 2\pi} = \frac{\Delta L}{\Delta t}
\]

因此,通过测量距离变化与时间延迟之间是关系即可计算光速。

\textbf{实验装置与方法:}

实验系统由光速测量仪(含光源、调制器、反射镜、光电接收器)、平行光管、导轨及示波器组成。通过调节折光器和接收器的位置,利用示波器的 Track 功能测得不同距离下的时间延迟 $\Delta t$,并通过直接计算或作图法求出光速。

\subsection{实验重点(3分)}
(简述本实验的学习重点,不超过100字。)


本实验的学习重点在于掌握利用\textbf{光调制法}测量光速的基本思路和操作方法,理解\textbf{调制信号、相位差与时间延迟}之间的定量关系。我们需要熟悉示波器的双通道测量与 \texttt{Track} 功能,能够准确读取光信号与参考信号的时间差;同时应掌握利用作图法求斜率计算光速的过程,并理解实验中\textbf{信号调制、波形相移、传播距离与光速计算}之间的物理联系,从而培养光电测量与数据处理的综合能力。


\subsection{实验难点(2分)}
(简述本实验的实现难点,不超过100字。)

本实验的主要难点在于\textbf{时间延迟的精确测量与光路调节}。由于信号频率较高,示波器上两路波形相位差微小,读数稍有偏差即可造成较大误差;同时光路需严格准直,否则接收信号幅值减弱,影响时间差测定。此外,实验中多次移动反射镜时需保持光轴一致,避免机械误差累积。我们需要通过反复调节与求平均值,来保证测量的稳定性与结果的可靠性。


\vspace{3\baselineskip}

\input{注意事项}
\end{document}
