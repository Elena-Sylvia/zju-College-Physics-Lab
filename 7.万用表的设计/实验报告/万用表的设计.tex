\documentclass{Phyport}
\usepackage{longtable}
\usepackage{multirow}      % 用于跨行
\usepackage{float} % 在导言区添加
\usepackage{amsmath}  % 必须加这一行!
\usepackage{ctex}     % 如果要显示中文,加上这一行

\usepackage{hyperref}
\hypersetup{
    colorlinks=true,
}

\exname{万用表的设计} %实验名称
\extable{8} %实验桌号
\instructor{谭艾林老师} %指导教师
\class{-} %班级
\name{-} %姓名
\stuid{-} %学号

\nyear{2025} %年
\nmonth{11} %月
\nday{10} %日
\nweekday{一} %星期几

\redate{} %如有实验补做,补做日期
\resitu{} %情况说明:

\begin{document}
\setcounter{page}{0}
\makecover

\section{预习报告(10分)}
(注:将已经写好的“物理实验预习报告”内容拷贝过来)

\subsection{实验综述(5分)}
(自述实验现象、实验原理和实验方法,包括必要的光路图、电路图、公式等。不超过500字。)

\textbf{万用表的设计实验}旨在通过改装电流计,使其成为具有多种量程的电流表、电压表和欧姆表,并了解其工作原理和设计方法。万用表是常用的测量工具,主要是由直流电流计及若干电阻构成。由于万用表具有多用途及使用方便等优点,它有着广泛的应用。本实验主要学会多量程电流表、电压表和万用表的设计及校正。

\textbf{实验现象:}

在实验中,通过在电流计两端并联或串联不同阻值的电阻,我们可以观察到电流计的量程发生变化,从而实现多量程电流表和电压表的功能。改装欧姆表时,通过调节电阻,可以观察到电流计指针的偏转与被测电阻之间的非线性关系。校准过程中,通过与标准电表的比较,可以发现改装后的电表存在误差,并可以通过调整电阻进行校正。

\textbf{实验原理:}

万用表主要由磁电式电流计和一系列电阻构成。电流计的两个重要参数是量程 $I_g$ 和内阻 $R_g$。$R_g$可以用替代法或者中值法获得。

1. \textbf{改装多量程电流表}

将磁电式电流计改装成量程为 $I$ 的电流表,需在电表表头两端并联一个分流电阻 $R_s$。分流电阻阻值按以下公式计算:
$$ R_s = \frac{R_g I_g}{I - I_g} $$

通过并联不同的分流电阻,可以构成不同量程的电流表。例如,设计 5mA 和 10mA 两个量程的电流表,其电路图如下:
\begin{figure}[htbp]
    \centering
    \begin{minipage}[t]{0.45\linewidth}
        \centering
        \includegraphics[width=\linewidth]{../images/多量程电流表改装电路.png} 
        \caption{多量程电流表改装电路}
        \label{fig:多量程电流表改装电路}
    \end{minipage}
    \hfill
    \begin{minipage}[t]{0.45\linewidth}
        \centering
        \includegraphics[width=\linewidth]{../images/多量程电流表校正电路.png} 
        \caption{多量程电流表校正电路}
        \label{fig:多量程电流表校正电路}
    \end{minipage}
    \label{fig:combined}
\end{figure}

计算 $R_1$ 和 $R_2$ 值的公式为:
$$
\begin{cases}
(5 - I_g)(R_1 + R_2) = I_g R_g \\
I_g (R_2 + R_g) = (10 - I_g) R_1
\end{cases}
$$

最后用标准安培表对改装后的电流表进行校正,并分析误差,校正电路见上图。

2. \textbf{改装多量程电压表}

将电流计改装成量程为 $U$ 的电压表,需串联一个分压电阻 $R_v$。分压电阻阻值按以下公式计算:

$$ R_v = \frac{U}{I_g'} - R_g' $$

其中 $R_g'$ 为电流计等效内阻,$I_g'$ 为电流计等效量程(采用改装后的电流表参数)。通过串联不同的分压电阻,可以得到不同量程的电压表。例如,设计 5V 和 10V 两个量程的电压表,其电路图如下:
\begin{figure}[htbp]
    \centering
    \begin{minipage}[t]{0.45\linewidth}
        \centering
        \includegraphics[width=\linewidth]{../images/多量程电压表改装电路.png} 
        \caption{多量程电压表改装电路}
        \label{fig:多量程电压表改装电路}
    \end{minipage}
    \hfill
    \begin{minipage}[t]{0.45\linewidth}
        \centering
        \includegraphics[width=\linewidth]{../images/多量程电压表校正电路.png} 
        \caption{多量程电压表校正电路}
        \label{fig:多量程电压表校正电路}
    \end{minipage}
    \label{fig:combined_2}
\end{figure}

计算 $R_3$ 和 $R_4$ 值的公式为:
$$
\begin{cases}
R_3 = \frac{5V - I_g'R_g'}{I_g'} \\
R_4 = \frac{10V - 5V}{I_g'} \\
R_g' = \frac{R_g(R_1+R_2)}{R_g + R_1+R_2} \\
I_g' = 5mA
\end{cases}
$$

最后用标准伏特表对改装后的电压表进行校正,并分析误差,校正电路见上图。

3. \textbf{改装欧姆表}

欧姆表改装原理是利用电流计的偏转与被测电阻之间的关系。电路图如下:
\begin{figure}[H]
    \centering
    \includegraphics[width=0.6\textwidth]{../images/改装欧姆表电路.png}
    \caption{改装欧姆表设计电路}
    \label{fig:ohmmeter_circuit}
\end{figure}

短接 a、b 两端,调节电阻 $R$ 使电流计满刻度,此时 $I_0 = \frac{\epsilon }{R_g' + R'}$。当接入待测电阻 $R_x$ 后,回路电流为 $I_x = \frac{\epsilon}{R_g' + R' + R_x}$。由于 $I_x$ 与 $R_x$ 呈非线性关系,因此欧姆表的刻度是非均匀的。

\textbf{实验装置与方法:}

实验器材包括电流计、滑动变阻器、电源、旋钮电阻箱、保护电阻、标准安培表、标准伏特表和面包板等。

1. \textbf{电流表改装:} 根据公式计算分流电阻值,通过旋钮电阻箱设置相应电阻,并联到电流计两端,构成多量程电流表。

2. \textbf{电压表改装:} 根据公式计算分压电阻值,通过旋钮电阻箱设置相应电阻,串联到电流计回路中,构成多量程电压表。

3. \textbf{欧姆表改装:} 搭建欧姆表电路,通过调节可调电阻进行调零,然后接入不同阻值的电阻,记录电流计读数,绘制 $I_x \sim R_x$ 曲线。

4. \textbf{校准与误差分析:} 使用标准安培表和标准伏特表对改装后的电流表和电压表进行校准,记录数据并进行误差分析。

\subsection{实验重点(3分)}
(简述本实验的学习重点,不超过100字。)

1. 理解多量程电流表、电压表、欧姆表的设计原理和改装方法。

2. 掌握并联分流和串联分压在电表改装中的应用。

3. 学会万用表的校准方法和误差分析方法。

\subsection{实验难点(2分)}
(简述本实验的实现难点,不超过100字。)

本实验的难点在于精确计算和选择合适的电阻值,以保证改装后电表的准确性。同时,在面包板上正确搭建复杂电路并进行精细调节,以及进行严谨的校准和误差分析,也是实验成功的关键挑战。

\section{原始数据(20分)}
(将有老师签名的“自备数据记录草稿纸”的扫描或手机拍摄图粘贴在下方,完整保留姓名,学号,教师签字和日期。)

\begin{figure}[H]
    \centering
    \includegraphics[width=0.75\textwidth]{../images/originaldata.jpg}
    \caption{original data}
    \label{fig:original_data}
\end{figure}

\section{结果与分析(60分)}
\subsection{数据处理与结果(30分)}
(列出数据表格、选择适合的数据处理方法、写出测量或计算结果。)

\textbf{预实验:测量电流计内阻}

在本次实验中,我们使用中值法来测量电流计的内阻 $R_g$。调整标准电流计示数至1.00mA,此时调整可变电阻的阻值至电流计的示数$I'=0.5mA$,记录此时可变电阻的阻值$R'=235 \Omega $。因此我们可以知道,电流计的内阻等于此时可变电阻的阻值,即$R_g=235 \Omega $。

\textbf{实验一:改装多量程电流表}

首先,根据公式:
$$ R_s = \frac{R_g I_g}{I - I_g} $$

计算出5mA量程所需的分流电阻值为:$R=59 \Omega $

\begin{longtable}{|c|c|c|c|c|c|}
  \caption{改装多量程电流表} \label{tab:改装多量程电流表} \\
  \hline
    \textbf{实验次数} & \textbf{1} & \textbf{2} & \textbf{3} & \textbf{4} & \textbf{5} \\
  \hline
  \textbf{$I_{\text{标准}}/mA$} & 1.00 & 2.00 & 3.00 & 4.00 & 5.00 \\
  \hline
  \textbf{$I_{\text{改装}}/mA$} & 1.06 & 2.05 & 3.04 & 3.95 & 5.00 \\
  \hline
  \textbf{$\Delta I/mA$} & 0.06 & 0.05 & 0.04 & -0.05 & 0 \\
  \hline
  \textbf{校准后的并联电阻$R_2$} & \multicolumn{5}{|c|}{59$\Omega$} \\
  \hline
\end{longtable}

用matplotlib绘制改装电流表与标准电流表的对比图以及$\Delta I$变化曲线如下:

\begin{figure}[htbp]
    \centering
    \begin{minipage}[t]{0.45\linewidth}
        \centering
        \includegraphics[width=\linewidth]{../images/output1.png} 
        \caption{$\Delta I$变化曲线}
        \label{fig:Delta I变化曲线}
    \end{minipage}
    \hfill
    \begin{minipage}[t]{0.45\linewidth}
        \centering
        \includegraphics[width=\linewidth]{../images/output2.png} 
        \caption{测量电流与标准电流对比图}
        \label{fig:测量电流与标准电流对比图}
    \end{minipage}
    \label{fig:combined3}
\end{figure}

我的改装电流表等级为:
$$
\text{电流表等级}=\frac{\text{最大绝对误差}}{\text{量程}}\times 100\% = \frac{0.06mA}{5mA} \times 100\% = 1.2\%
$$

\textbf{实验二:改装多量程电压表}

首先,根据公式:
$$ R_v = \frac{U}{I_g'} - R_g' $$

计算出5V量程所需的分压电阻值为:$R=953 \Omega $。

\begin{longtable}{|c|c|c|c|c|c|}
  \caption{改装多量程电压表} \label{tab:改装多量程电压表} \\
  \hline
    \textbf{实验次数} & \textbf{1} & \textbf{2} & \textbf{3} & \textbf{4} & \textbf{5} \\
  \hline
  \textbf{$U_{\text{标准}}/mA$} & 1.00 & 2.00 & 3.00 & 4.00 & 5.00 \\
  \hline
  \textbf{$U_{\text{改装}}/mA$} & 1.11 & 2.12 & 3.10 & 4.05 & 5.00 \\
  \hline
  \textbf{$\Delta U/mA$} & 0.11 & 0.12 & 0.10 & 0.05 & 0 \\
  \hline
  \textbf{校准后的串联电阻$R_2$} & \multicolumn{5}{|c|}{943$\Omega$} \\
  \hline
\end{longtable}

用matplotlib绘制改装电压表与标准电压表的对比图以及$\Delta U$变化曲线如下:

\begin{figure}[htbp]
    \centering
    \begin{minipage}[t]{0.45\linewidth}
        \centering
        \includegraphics[width=\linewidth]{../images/output3.png} 
        \caption{$\Delta U $变化曲线}
        \label{fig:Delta U变化曲线}
    \end{minipage}
    \hfill
    \begin{minipage}[t]{0.45\linewidth}
        \centering
        \includegraphics[width=\linewidth]{../images/output4.png} 
        \caption{测量电压与标准电压对比图}
        \label{fig:测量电压与标准电压对比图}
    \end{minipage}
    \label{fig:combined4}
\end{figure}

我的改装电压表等级为:
$$
\text{电压表等级}=\frac{\text{最大绝对误差}}{\text{量程}}\times 100\% = \frac{0.12V}{5V} \times 100\% = 2.4\%
$$

串联电阻的相对误差为:
$$
\text{相对误差}=\frac{953\Omega - 943\Omega}{953\Omega} \times 100\% \approx 1.05\%
$$

\textbf{实验三:改装欧姆表}

\begin{longtable}{|c|c|c|c|c|c|c|c|c|c|c|c|c|c|c|}
  \caption{改装欧姆表} \label{tab:改装欧姆表} \\
  \hline
  \textbf{$I_x/mA$} & 5.0 & 4.6 & 4.2 & 3.8 & 3.4 & 3.0 & 2.6 & 2.2 & 1.8 & 1.4 & 1.0 & 0.6 & 0.4 & 0.2 \\
  \hline
  \textbf{$R_x/\Omega$} & 0 & 30.2 & 60.2 & 96.2 & 140.2 & 200.1 & 260.1 & 361.1 & 501.1 & 701.1 & 1101.1 & 1801.1 & 2643.1 & 4443.1   \\
  \hline
\end{longtable}

\begin{figure}[htbp]
    \centering
    \begin{minipage}[t]{0.45\linewidth}
        \centering
        \includegraphics[width=\linewidth]{../images/output5.png} 
        \caption{$I_x - R_x$曲线}
        \label{fig:电流电阻曲线}
    \end{minipage}
    \hfill
    \begin{minipage}[t]{0.45\linewidth}
        \centering
        \includegraphics[width=\linewidth]{../images/output6.png} 
        \caption{$I_x - log(R_x)$曲线}
        \label{fig:电流电阻对数曲线}
    \end{minipage}
    \label{fig:combined5}
\end{figure}

右图是对电阻值取对数后的曲线,从两张曲线图中我们可以看出,在电流较大的区域(低电阻区),电阻刻度比较稀疏;而在电流较小的区域(高电阻区),电阻刻度则非常密集。总体来说,欧姆表刻度盘上的电阻值刻度是非均匀的。

\subsection{误差分析(20分)}
(运用测量误差、相对误差或不确定度等分析实验结果,写出完整的结果表达式,并分析误差原因。)

在本实验中,我们通过改装电流计来设计多量程电流表、多量程电压表和欧姆表,并对其进行了校准和误差分析。以下是本次实验中的误差来源分析:

(电表等级分析以及相对误差的计算在实验数据处理部分已经完成)

1.在对电流计进行读数的时候,我发现电流计的指针有时会出现抖动现象,导致读数不稳定,从而引入了读数误差。

2.存在视差。电流计的指针在偏转角度较大的时候,并不与刻度完全重合(存在一定的夹角),导致我在读数时存在一定的估读误差。

3.连接电路的导线也存在电阻,尤其是在小电阻测量时,导线的电阻不可忽略,且不同元件之间接触不良也可能引入额外的接触电阻。

4.实验中使用的直流电源可能存在电压波动或内阻变化。在改装欧姆表的实验过程中,测量结果直接依赖于电源电压的稳定性。因此在测量过程中,如果电源输出不稳定,会导致测量结果不准确。

\subsection{实验探讨(10分)}
(对实验内容、现象和过程的小结,不超过100字。)

本次\textbf{万用表的设计实验},旨在根据已有的电流计改装获得电流表电压表以及欧姆表。我们首先基于磁电式电流计,通过理论计算确定并联电阻阻值,改装并校准了5mA量程的电流表。接着,通过串联电阻的方法,改装并校准了5V量程的电压表。最后,我们搭建了欧姆表电路,观察到电流与电阻之间明显的非线性关系,经过这次实验,我对电表的改装以及量程的扩展有了更加深刻的认识。

\section{思考题(10分)}
(解答教材或讲义或老师布置的思考题,请先写题干,再作答。)

\textbf{思考题一:为什么不能用万用表欧姆挡测量电源的电阻?}

万用表的欧姆挡是通过表内电池向被测电阻供电,并测量由此产生的电流来间接测量电阻值的。当使用欧姆挡测量电源电阻时,被测电源本身也提供电动势。这就会形成两个电动势同时作用的电路。如果被测电源的电压高于欧姆表内电池电压,或者两者方向相反,可能会导致:

    \textbf{1.损坏欧姆表:} 过大的电流可能烧毁欧姆表内部的电流计或精密电阻。

    \textbf{2.测量结果不准确:} 被测电源的电动势会干扰欧姆表内电池产生的电流,使流过电流计的电流不再仅仅由被测电阻决定,从而导致测量值严重偏离实际电阻。

因此,欧姆挡设计用于测量无源电阻(不带电的电阻),测量有源电阻(带电的电源)会带来危险并导致错误结果。

\vspace{1\baselineskip}

\textbf{思考题二: 为什么不能用欧姆表测量另一表头内阻?}

欧姆表测量原理是利用表内电池提供电流,通过测量此电流来推算被测电阻的阻值。当用欧姆表测量另一个电流表(或电压表)的表头内阻时:

    \textbf{1.欧姆表对表头放电:} 欧姆表内电池会向被测表头供电,使表头指针发生偏转,此时流过表头的电流是欧姆表电池驱动的。

    \textbf{2.表头内阻的特性:} 另一个表头本身也是一个精密仪器,其内阻通常比较小。当欧姆表接入时,表头内部可能会产生瞬间过大的冲击电流,特别是对于灵敏度较高的电流表,这可能导致其线圈损坏,或使指针因过流而弯曲甚至烧毁。

因此,为了保护被测表头和确保测量安全,我们不应使用欧姆表来测量另一个表头的内阻。测量表头内阻通常需要采用替代法或中值法等方法。

\vspace{1\baselineskip}

\textbf{思考题三:为什么$I_x$与$R_x$为非线性关系?}

根据欧姆表的工作原理,当外接被测电阻$R_x$时,回路总电阻为$R_\text{total} + R_x$,流过电流计的电流$I_x$为:
$$ 
I_x = \frac{\epsilon_0}{R_\text{total} + R_x} 
$$

其中,$\epsilon_0$为欧姆表内电池的电动势,$R_{total}$是欧姆表内部电路的总电阻(包括电流计内阻、调零电阻、限流电阻等)。

从上述公式可以看出,$I_x$与$R_x$之间的关系并不是一个简单的线性正比或反比关系,而是一种非线性反比关系。

\vspace{3\baselineskip}
\input{注意事项}
\end{document}
