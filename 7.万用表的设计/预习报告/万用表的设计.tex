\documentclass{Preport}
\usepackage{longtable}
\usepackage{multirow}      % 用于跨行
\usepackage{float} % 在导言区添加
\usepackage{amsmath}  % 必须加这一行!
\usepackage{ctex}     % 如果要显示中文,加上这一行

\usepackage{hyperref}
\hypersetup{
    colorlinks=true,
}

\exname{万用表的设计} %实验名称
\instructor{谭艾林老师} %指导教师
\class{-} %班级
\name{-} %姓名
\stuid{-} %学号

\nyear{2025} %年
\nmonth{11} %月
\nday{10} %日
\nweekday{一} %星期几

\begin{document}
\setcounter{page}{0}
\makecover

\section{预习报告(10分)}
\subsection{实验综述(5分)}
(自述实验现象、实验原理和实验方法,包括必要的光路图、电路图、公式等。不超过500字。)

\textbf{万用表的设计实验}旨在通过改装电流计,使其成为具有多种量程的电流表、电压表和欧姆表,并了解其工作原理和设计方法。万用表是常用的测量工具,主要是由直流电流计及若干电阻构成。由于万用表具有多用途及使用方便等优点,它有着广泛的应用。本实验主要学会多量程电流表、电压表和万用表的设计及校正。

\textbf{实验现象:}

在实验中,通过在电流计两端并联或串联不同阻值的电阻,我们可以观察到电流计的量程发生变化,从而实现多量程电流表和电压表的功能。改装欧姆表时,通过调节电阻,可以观察到电流计指针的偏转与被测电阻之间的非线性关系。校准过程中,通过与标准电表的比较,可以发现改装后的电表存在误差,并可以通过调整电阻进行校正。

\textbf{实验原理:}

万用表主要由磁电式电流计和一系列电阻构成。电流计的两个重要参数是量程 $I_g$ 和内阻 $R_g$。$R_g$可以用替代法或者中值法获得。

1. \textbf{改装多量程电流表}

将磁电式电流计改装成量程为 $I$ 的电流表,需在电表表头两端并联一个分流电阻 $R_s$。分流电阻阻值按以下公式计算:
$$ R_s = \frac{R_g I_g}{I - I_g} $$

通过并联不同的分流电阻,可以构成不同量程的电流表。例如,设计 5mA 和 10mA 两个量程的电流表,其电路图如下:
\begin{figure}[htbp]
    \centering
    \begin{minipage}[t]{0.45\linewidth}
        \centering
        \includegraphics[width=\linewidth]{../images/多量程电流表改装电路.png} 
        \caption{多量程电流表改装电路}
        \label{fig:多量程电流表改装电路}
    \end{minipage}
    \hfill
    \begin{minipage}[t]{0.45\linewidth}
        \centering
        \includegraphics[width=\linewidth]{../images/多量程电流表校正电路.png} 
        \caption{多量程电流表校正电路}
        \label{fig:多量程电流表校正电路}
    \end{minipage}
    \label{fig:combined}
\end{figure}

计算 $R_1$ 和 $R_2$ 值的公式为:
$$
\begin{cases}
(5 - I_g)(R_1 + R_2) = I_g R_g \\
I_g (R_2 + R_g) = (10 - I_g) R_1
\end{cases}
$$

最后用标准安培表对改装后的电流表进行校正,并分析误差,校正电路见上图。

2. \textbf{改装多量程电压表}

将电流计改装成量程为 $U$ 的电压表,需串联一个分压电阻 $R_v$。分压电阻阻值按以下公式计算:

$$ R_v = \frac{U}{I_g'} - R_g' $$

其中 $R_g'$ 为电流计等效内阻,$I_g'$ 为电流计等效量程(采用改装后的电流表参数)。通过串联不同的分压电阻,可以得到不同量程的电压表。例如,设计 5V 和 10V 两个量程的电压表,其电路图如下:
\begin{figure}[htbp]
    \centering
    \begin{minipage}[t]{0.45\linewidth}
        \centering
        \includegraphics[width=\linewidth]{../images/多量程电压表改装电路.png} 
        \caption{多量程电压表改装电路}
        \label{fig:多量程电压表改装电路}
    \end{minipage}
    \hfill
    \begin{minipage}[t]{0.45\linewidth}
        \centering
        \includegraphics[width=\linewidth]{../images/多量程电压表校正电路.png} 
        \caption{多量程电压表校正电路}
        \label{fig:多量程电压表校正电路}
    \end{minipage}
    \label{fig:combined_2}
\end{figure}

计算 $R_3$ 和 $R_4$ 值的公式为:
$$
\begin{cases}
R_3 = \frac{5V - I_g'R_g'}{I_g'} \\
R_4 = \frac{10V - 5V}{I_g'} \\
R_g' = \frac{R_g(R_1+R_2)}{R_g + R_1+R_2} \\
I_g' = 5mA
\end{cases}
$$

最后用标准伏特表对改装后的电压表进行校正,并分析误差,校正电路见上图。

3. \textbf{改装欧姆表}

欧姆表改装原理是利用电流计的偏转与被测电阻之间的关系。电路图如下:
\begin{figure}[H]
    \centering
    \includegraphics[width=0.6\textwidth]{../images/改装欧姆表电路.png}
    \caption{改装欧姆表设计电路}
    \label{fig:ohmmeter_circuit}
\end{figure}

短接 a、b 两端,调节电阻 $R$ 使电流计满刻度,此时 $I_0 = \frac{\epsilon }{R_g' + R'}$。当接入待测电阻 $R_x$ 后,回路电流为 $I_x = \frac{\epsilon}{R_g' + R' + R_x}$。由于 $I_x$ 与 $R_x$ 呈非线性关系,因此欧姆表的刻度是非均匀的。

\textbf{实验装置与方法:}

实验器材包括电流计、滑动变阻器、电源、旋钮电阻箱、保护电阻、标准安培表、标准伏特表和面包板等。

1. \textbf{电流表改装:} 根据公式计算分流电阻值,通过旋钮电阻箱设置相应电阻,并联到电流计两端,构成多量程电流表。

2. \textbf{电压表改装:} 根据公式计算分压电阻值,通过旋钮电阻箱设置相应电阻,串联到电流计回路中,构成多量程电压表。

3. \textbf{欧姆表改装:} 搭建欧姆表电路,通过调节可调电阻进行调零,然后接入不同阻值的电阻,记录电流计读数,绘制 $I_x \sim R_x$ 曲线。

4. \textbf{校准与误差分析:} 使用标准安培表和标准伏特表对改装后的电流表和电压表进行校准,记录数据并进行误差分析。


\subsection{实验重点(3分)}
(简述本实验的学习重点,不超过100字。)

1. 理解多量程电流表、电压表、欧姆表的设计原理和改装方法。

2. 掌握并联分流和串联分压在电表改装中的应用。

3. 学会万用表的校准方法和误差分析方法。

\subsection{实验难点(2分)}
(简述本实验的实现难点,不超过100字。)

本实验的难点在于精确计算和选择合适的电阻值,以保证改装后电表的准确性。同时,在面包板上正确搭建复杂电路并进行精细调节,以及进行严谨的校准和误差分析,也是实验成功的关键挑战。

\vspace{3\baselineskip}
\input{注意事项}
\end{document}
