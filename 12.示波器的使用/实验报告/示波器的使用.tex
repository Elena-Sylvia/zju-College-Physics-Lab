\documentclass{Phyport}
\usepackage{longtable}
\usepackage{multirow}      % 用于跨行
\usepackage{float} % 在导言区添加
\usepackage{amsmath}  % 必须加这一行!
\usepackage{ctex}     % 如果要显示中文,加上这一行

\usepackage{hyperref}
\hypersetup{
    colorlinks=true,
}

\exname{示波器的使用} %实验名称
\extable{8} %实验桌号
\instructor{王鲲老师} %指导教师
\class{-} %班级
\name{-} %姓名
\stuid{-} %学号

\nyear{2025} %年
\nmonth{12} %月
\nday{15} %日
\nweekday{一} %星期几

\redate{} %如有实验补做,补做日期
\resitu{} %情况说明:

\begin{document}
\setcounter{page}{0}
\makecover

\section{预习测试(10分)}
上课前到学在浙大上完成,注意测试仅1次机会。期末时测试分数会与报告其他部分的分数进行加和处理。

\section{原始数据(20分)}
(将有老师签名的“自备数据记录草稿纸”的扫描或手机拍摄图粘贴在下方,完整保留姓名,学号,教师签字和日期。)

\begin{figure}[H]
    \centering
    \includegraphics[width=0.70\textwidth]{../images/originaldata.jpg}
    \caption{original data}
    \label{fig:originaldata}
\end{figure}

\section{结果与分析(60分)}
\subsection{数据处理与结果(30分)}
(列出数据表格、选择适合的数据处理方法、写出测量或计算结果。)

\vspace{1\baselineskip}
\textbf{实验一:用比较法验证$f_y=n\times f_x$}

\begin{longtable}{|c|c|c|}
  \caption{用比较法验证$f_y=n\times f_x$} \label{tab:比较法} \\
  \hline
    \textbf{波形个数n} & \textbf{信号频率$f_y/Hz$} & \textbf{测得的扫描频率$f_x/Hz$}  \\
  \hline
    \textbf{1} & 200.4 & 200.4 \\
  \hline
    \textbf{2} & 400.6 & 200.3 \\
  \hline
    \textbf{3} & 601.0 & 200.3 \\
  \hline
    \textbf{4} & 802.3 & 200.6 \\
  \hline
    \textbf{5} & 1001.0 & 200.2 \\
  \hline
\end{longtable}

$$
\overline{f_x}=\frac{\sum_{i=1}^{5}f_{xi}}{5}=200.36Hz
$$

$$
\text{相对误差为:}E=\frac{|\overline{f_x}-f_{x0}|}{f_{x0}} \times 100 \% =\frac{|200.36-200|}{200}= 0.18\%
$$

\vspace{1\baselineskip}

\textbf{实验二:用李萨如图形测量未知信号的频率}

\textbf{信号发生器背后输出的交流50Hz的电压,作为$f_y$}

\begin{longtable}{|c|c|c|c|c|c|}
  \caption{用李萨如图形测量未知信号的频率} \label{tab:李萨如图形} \\
  \hline
    \textbf{频率比$f_y:f_x$} & 1:1 & 1:2 & 1:3 & 2:1 & 2:3 \\
  \hline
    \textbf{图形} & 见下图2 & 见下图3 & 见下图4 & 见下图5 & 见下图6  \\
  \hline
    \textbf{垂直交点数$N_y$} & 2 & 4 & 6 & 2 & 6 \\
  \hline
    \textbf{水平交点数$N_x$} & 2 & 2 & 2 & 4 & 4 \\
  \hline
    \textbf{$f_x(Hz)$} & 50.030 & 100.262 & 150.225 & 25.020 & 75.000 \\
  \hline
    \textbf{$f_y(Hz)$} & 50.030 & 50.131 & 50.075 & 50.040 & 50.000 \\
  \hline
\end{longtable}

\begin{figure}[htbp]
    \centering
    \begin{minipage}[t]{0.2\linewidth}
        \centering
        \includegraphics[width=\linewidth]{../images/1.png} 
        \caption{1:1}
        \label{fig:1:1}
    \end{minipage}
    \hspace{0.05\linewidth}
    \begin{minipage}[t]{0.2\linewidth}
        \centering
        \includegraphics[width=\linewidth]{../images/2.png} 
        \caption{1:2}
        \label{fig:1:2}
    \end{minipage}
    \hspace{0.05\linewidth}
    \begin{minipage}[t]{0.2\linewidth}
        \centering
        \includegraphics[width=\linewidth]{../images/3.png} 
        \caption{1:3}
        \label{fig:1:3}
    \end{minipage}
    \label{fig:combined_3} 
\end{figure}

\begin{figure}[htbp]
    \centering
    \begin{minipage}[t]{0.2\linewidth}
        \centering
        \includegraphics[width=\linewidth]{../images/4.png} 
        \caption{2:1}
        \label{fig:2:1}
    \end{minipage}
    \hspace{0.05\linewidth}
    \begin{minipage}[t]{0.2\linewidth}
        \centering
        \includegraphics[width=\linewidth]{../images/5.png} 
        \caption{2:3}
        \label{fig:2:3}
    \end{minipage}
    \label{fig:combined_2}
\end{figure}

$$
\overline{f_y}=\frac{\sum_{i=1}^{5}f_{yi}}{5}=50.055Hz
$$

$$
\text{相对误差为:}E=\frac{|\overline{f_y}-f_{y0}|}{f_{y0}} \times 100 \% =\frac{|50.055Hz-50.000Hz|}{50.000Hz}= 0.11\%
$$

\vspace{1\baselineskip}

\textbf{实验三:二极管正向导通电压的测量}

\textbf{分别使用光标法和直读法进行测量}

\begin{longtable}{|c|c|c|c|}
  \caption{二极管正向导通电压的测量} \label{tab:二极管正向导通电压的测量} \\
  \hline
     & \textbf{光标法} & \textbf{直读法格数} & \textbf{直读法电压大小} \\
  \hline
    \textbf{输入电压的峰-峰值$U_{1p-p}$} & 4.80V & 4.8格 & 4.80V\\
  \hline
    \textbf{输出半波电压的峰值$U_{2p}$} & 1.56V & 3.2格 & 1.60V \\
  \hline
  \textbf{正向导通电压$U_\text{导通}$} & 0.84V & - & 0.80V \\
  \hline
\end{longtable}

其中$U_\text{导通}=\frac{1}{2}U_{1p-p}-U_{2p}$

\vspace{1\baselineskip}
\textbf{实验四:RC电路输入输出波形相位差的测量}

\textbf{实验参数:R=750 $\Omega$  ,C=  0.47 $\mu F$ ,输入信号频率= 2KHz}

\begin{longtable}{|c|c|c|c|}
  \caption{RC电路相位差测量} \label{tab:RC电路相位差测量} \\
  \hline
    \textbf{光标法} &  & \textbf{直读法} &  \\
  \hline
    \textbf{波形时间差$\Delta t$} & 0.120ms & \textbf{波形时间差$\Delta t$} & 0.120ms \\
  \hline
    \textbf{周期T} & 0.496ms & \textbf{周期T} & 0.500ms \\
  \hline
\end{longtable}

根据光标法的测量结果,相位差
$$
\Delta \phi = \frac{\Delta t}{T}\times 360^\circ = \frac{0.120ms}{0.496ms}\times 360^\circ \approx 87.1^\circ
$$

\subsection{误差分析(20分)}
(运用测量误差、相对误差或不确定度等分析实验结果,写出完整的结果表达式,并分析误差原因。)

本次实验的相对误差在实验数据处理部分已经计算完毕,因此本部分直接进行实验误差分析。

1.由于波形存在宽度,因此实验时在测量u以及t的过程中,图形移动到的位置是靠肉眼估计的,光标的位置也需要估计(完全相切以及波峰波谷的位置难以判断),人眼在读取波形时的误差也会对测量得到的t以及电压U产生影响。

2.在用李萨如图形测量未知信号的频率实验中,由于我们信号发生器的精度只到小数点后三位,精度有限,很难通过调节$f_x$来使李萨如图形保持稳定;同时在实验中我发现,信号发生器发出的频率似乎并不稳定,在调节到 稳定状态后数秒图形又会继续开始翻转。

3.根据第四个实验的测量结果我们可以发现,我们设置的频率值与实际实验中测得的频率值还是存在一定偏差的,这说明了信号发生器实际输出频率与设定值存在一定的误差,这会在各个实验中引入误差。

\subsection{实验探讨(10分)}
(对实验内容、现象和过程的小结,不超过100字。)

在本次示波器应用实验中,我成功实现了对多种信号参数的测量。首先通过比较法验证了示波器时基的精度,测得平均扫描频率 $\bar{f}_x = 200.36Hz$,相对误差 $E = 0.18\%$;李萨如图形法成功测得未知信号频率 $\bar{f}_y = 50.055Hz$,相对误差 $E = 0.11\%$,验证了该方法的准确性;在二极管实验中,我测得正向导通电压 $U_\text{导通} = 0.84V$;而在RC 电路实验中责编测得输入输出相位差 $\Delta\phi \approx 87.1^\circ$。实验过程中,波形对齐和李萨如图形稳定性的调节是实验的难点,测量结果的精度受限于人眼估读误差和时基精度,是主要的误差来源。


\section{思考题(10分)}
(解答教材或讲义或老师布置的思考题,请先写题干,再作答。)

\textbf{思考题一:示波器的主要结构与各部分的作用是什么?示波器为什么能显示被测信号的波形?}

示波器由\textbf{四个基本部分}组成:示波管(阴极射线管)、放大器(X轴和Y轴)、扫描与触发同步系统,以及电源。

\begin{enumerate}[label=\arabic*.]
    \item \textbf{示波管:} 包含电子枪、偏转系统和荧光屏。电子枪产生并加速电子束;偏转系统接收放大后的电压信号,控制电子束水平(X)和垂直(Y)方向的偏转;荧光屏接收电子束撞击并发光显示波形。
    
    \item \textbf{放大器:} 包括 Y 轴放大器(垂直)和 X 轴放大器(水平)。Y 放大器将待测信号电压放大后加到 Y 偏转板;X 放大器放大扫描电压。
    
    \item \textbf{扫描与触发同步系统:} 产生锯齿波扫描电压,使光点在 X 轴方向匀速运动,并保证每次扫描的起始时刻与被测信号的某一特定点(触发电平)同步,是显示稳定波形的关键。
    
    \item \textbf{电源:} 提供示波器正常工作所需的的各种高低电压。
\end{enumerate}

显示波形的原理是:

\begin{enumerate}[label=\arabic*.]
    \item 电子枪产生并聚焦电子束。
    
    \item 待测信号经过 Y 轴放大后加到示波管的 Y 轴垂直偏转板,使电子束垂直偏转,位移 $ y \propto U_{\text{测}}(t) $。
    
    \item 扫描与触发同步系统产生锯齿波扫描电压,加到 X 轴水平偏转板,使电子束水平方向匀速移动,位移 $ x \propto t $。
    
    \item 最终,电子束在屏幕上留下的光点轨迹 $ (x, y) $ 便反映了 $ y \propto U_{\text{测}}(t) $ 随时间 $ t $ 变化的波形图。
\end{enumerate}

\textbf{思考题二:在观察李萨如图形时为什么总是不断的来回翻动,翻动的快慢是受哪种因素所影响?}

李萨如图形是相互垂直的两个简谐振动的合成,其稳定显示的条件是\textbf{两个信号的频率成严格的整数比(或有理数比)且相位差恒定}。李萨如图形不断来回翻动,是由于两个信号的频率并非严格的整数比,导致它们的相位差在不断地缓慢变化。

图形翻动(或漂移)的快慢取决于两个信号的频率差 $ \Delta f = |pf_x - qf_y| $(其中 $ p, q $ 为两个整数)。频率差越大,相位差变化越快,图形翻动的速度也就越快。通过微调信号发生器输出频率 $ f_x $,使得频率比接近精确的有理数比,可以使图形的翻动(相位差的变化)变慢甚至静止。

\textbf{思考题三:切实理解示波器同步的概念,如果发生波形左移或右移时应该如何调整才能使其稳定下来?}

同步概念是保证每次扫描都从被测信号波形的同一点开始(即相同的相位、相同的电平),从而使不同时刻的波形在屏幕上完全重叠,显示出一条稳定的波形。

若发生波形左移或右移而无法稳定,说明扫描触发点没有被固定,即未同步。应通过调节 TRIGGER LEVEL (触发电平) 旋钮来解决不同步问题。具体的调节方法是:用 TRIG LEVEL 旋钮将触发电平设置在被测信号电压幅值范围以内(通常在波形的上升沿或下降沿上),使得每次扫描都从信号电压的同一电平点启动。这样每次捕捉到的波形相位都相同,波形就能在屏幕上静止稳定显示。


\vspace{3\baselineskip}
\input{注意事项}
\end{document}
