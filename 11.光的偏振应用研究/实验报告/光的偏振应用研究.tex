\documentclass{Phyport}
\usepackage{longtable}
\usepackage{multirow}      % 用于跨行
\usepackage{float} % 在导言区添加
\usepackage{amsmath}  % 必须加这一行!
\usepackage{ctex}     % 如果要显示中文,加上这一行

\usepackage{hyperref}
\hypersetup{
    colorlinks=true,
}

\exname{光的偏振应用研究} %实验名称
\extable{8} %实验桌号
\instructor{姚星星老师} %指导教师
\class{-} %班级
\name{-} %姓名
\stuid{-} %学号

\nyear{2025} %年
\nmonth{12} %月
\nday{8} %日
\nweekday{一} %星期几

\redate{} %如有实验补做,补做日期
\resitu{} %情况说明:

\begin{document}
\setcounter{page}{0}
\makecover

\section{预习报告(10分)}
(注:将已经写好的“物理实验预习报告”内容拷贝过来)

\subsection{实验综述(5分)}
(自述实验现象、实验原理和实验方法,包括必要的光路图、电路图、公式等。不超过500字。)

\textbf{光的偏振及其应用研究}实验旨在通过观测光在介质界面的反射特性,深入理解光的横波性质。本次实验的核心在于利用布儒斯特角原理测定介质的折射率,并验证线偏振光遵循马吕斯定律。

\vspace{1\baselineskip}

\textbf{实验原理:}

\textbf{1. 菲涅尔公式与偏振光的反射特性}

光波作为电磁波,在两种介质界面(如空气与玻璃)反射时,其反射特性强烈依赖于光的偏振态。如图1所示,将入射光矢量分解为两个正交分量:

\begin{itemize}
    \item \textbf{S分量(图1-1):} 光矢量垂直于入射面(垂直纸面,用圆点表示)。
    \item \textbf{P分量(图1-2):} 光矢量平行于入射面(在纸面内,用箭头表示)。
\end{itemize}

\begin{figure}[H]
    \centering
    \includegraphics[width=0.75\textwidth]{../images/光的偏振原理图.png}
    \caption{菲涅尔公式原理图}
    \label{fig:菲涅尔公式原理图}
\end{figure}

根据菲涅尔公式,P分量的反射振幅比为:

\begin{equation}
    \frac{E_1}{E_0} = \frac{\tan(\theta_0 - \theta_2)}{\tan(\theta_0 + \theta_2)}
\end{equation}

其中 $\theta_0$ 为入射角,$\theta_2$ 为折射角。

当入射角满足 $\theta_0 + \theta_2 = 90^\circ$ 时,公式分母趋于无穷大,反射系数为零。这意味着此时反射光中不包含P分量,只有S分量,反射光变为完全线偏振光。

该特定入射角称为\textbf{布儒斯特角}(Brewster's Angle, $\theta_B$),此时满足:

\begin{equation}
    \tan \theta_B = \frac{n_2}{n_1} = n \quad (\text{空气中 } n_1 \approx 1)
\end{equation}

实验通过测量使反射光中P分量消失的入射角 $\theta_B$,即可求得折射率 $n$。

\textbf{2. 马吕斯定律(Malus's Law)}

当线偏振光(光强为 $I_0$)透过检偏器时,若线偏振光的振动方向与检偏器透光轴的夹角为 $\phi$,则透射光强 $I$ 满足:
\[
I = I_0 \cos^2 \phi
\]

实验中利用硅光电池测量光电流 $i$。由于在一定范围内光电流与入射光强成正比($i \propto I$),通过测量光电流 $i$ 随转角 $\phi$ 的变化,即可验证该线性关系。


\vspace{1\baselineskip}

\textbf{实验方法:}

\textbf{1. 测量黑色平板的折射率(逐渐逼近法)}
\begin{itemize}
    \item 调节激光器、转盘中心、黑色平板及光探测器等高共轴。
    \item 旋转激光器(或加起偏器),使入射光为平行于入射面的线偏振光($P$光)。
    \item 转动载物台改变入射角,观察反射光强。当反射光强达到最弱(消光)时,记录刻度盘读数。
    \item 为消除偏心差,在转台左右两侧分别测量布儒斯特角对应的位置读数 $\theta_{L}$ 和 $\theta_{R}$,则布儒斯特角 $\theta_B$ 可由相关几何关系(如 $|\theta_L - \theta_R|/2$ 或根据具体仪器零点公式)计算得出,进而求出 $n = \tan \theta_B$。
\end{itemize}

\textbf{2. 验证光电流与光强的关系}
\begin{itemize}
    \item 固定入射角为布儒斯特角,此时反射光为垂直于入射面振动的线偏振光($S$光)。
    \item 在反射光路中加入偏振片。旋转偏振片,测量并记录光电流出现极大值 $i_{\max}$ 和极小值 $i_{\min}$ 的角度,确定偏振片的透光轴方向;随后改变偏振片角度 $\phi$(如每隔 $10^\circ$),记录对应的光电流 $i$。同时测量遮挡光路时的本底电流 $i'$。
    \item 绘制 $(i - i') \sim \cos^2 \phi$ 曲线,通过观察曲线是否拟合为通过原点的直线来验证马吕斯定律。
\end{itemize}

\subsection{实验重点(3分)}
(简述本实验的学习重点,不超过100字。)

1. 深刻理解光的偏振态(线偏振、自然光),以及布儒斯特角产生线偏振光(p分量反射为0)的物理机制。

2. 掌握光学实验中的共轴调节技术,以及利用“逐渐逼近法”准确寻找反射光强极小值位置的方法。

3. 学会处理实验数据,通过扣除本底电流和作图分析,验证光电流与偏振角余弦平方的线性关系(马吕斯定律)。

\subsection{实验难点(2分)}
(简述本实验的实现难点,不超过100字。)

1. \textbf{极小值的精确判断:} 在布儒斯特角附近反射光强变化平缓,肉眼或仪器难以直接锁定唯一的最小值点,需通过测量光强对称点来辅助定位,操作要求高。

2. \textbf{系统误差控制:} 激光器、转台与检测器的共轴等高调节直接影响角度测量的准确性;此外,环境背景光会引入本底电流,若不进行有效遮蔽或修正,将导致马吕斯定律验证曲线偏离原点。

\section{原始数据(20分)}
(将有老师签名的“自备数据记录草稿纸”的扫描或手机拍摄图粘贴在下方,完整保留姓名,学号,教师签字和日期。)

\begin{figure}[H]
    \centering
    \includegraphics[width=0.75\textwidth]{../images/originaldata.jpg}
    \caption{original data}
    \label{fig:originaldata}
\end{figure}

\section{结果与分析(60分)}
\subsection{数据处理与结果(30分)}
(列出数据表格、选择适合的数据处理方法、写出测量或计算结果。)

\textbf{实验一:测量黑色平板折射率}

\begin{longtable}{|c|c|c|c|}
  \caption{测量黑色平板折射率数据} \label{tab:测量黑色平板折射率数据} \\
  \hline
    \textbf{测量次数} & \textbf{左侧角坐标$\theta_1(\text{度})$} & \textbf{右侧角坐标$\theta_2(\text{度})$} & \textbf{$\theta_b=|\theta_1-\theta_2|/4$} \\
  \hline
    \textbf{1} & $75.3^\circ$ & $294.4^\circ$ & $54.8^\circ$ \\
  \hline
    \textbf{2} & $77.2^\circ$ & $294.5^\circ$ & $54.3 ^\circ$ \\
  \hline
    \textbf{3} & $76.3^\circ$ & $293.4^\circ$ & $54.3^\circ$ \\
  \hline
    \textbf{4} & $75.8^\circ$ & $293.6\circ$ & $54.5^\circ$ \\
  \hline
    \textbf{5} & $75.9^\circ$ & $293.5^\circ$ & $54.4^\circ$ \\
  \hline
    \textbf{6} & $76.1^\circ$ & $293.6^\circ$ & $54.4^\circ$ \\
  \hline
\end{longtable}

布儒斯特角的平均值为:

$$
\overline{\theta_B} = \frac{\sum_{1}^{6}\theta_{bi}}{6} = 54.45^\circ
$$

由此就可以求得折射率的平均值为:$\bar{n} = tan(54.45^\circ)\approx 1.3992$

接下来进行不确定度的计算。

$$
u_A=\sqrt{\frac{\sum (\theta_{bi}-\overline{\theta_b})^2}{n(n-1)}}=\sqrt{\frac{0.175}{30}}\approx 0.0764^\circ \;\;\;\;\; u_B= \frac{1}{4}\sqrt{u(\theta)^2+u(\theta)^2}=\frac{1}{4}\sqrt{2\times (\frac{0.5}{\sqrt{3}})^2}\approx0.1021^\circ
$$

$$
u_c(\theta_B)=\sqrt{u_A^2+u_B^2}\approx0.1275^\circ
$$

因为n和$\theta_B$的关系为$n=tan\theta_B$,所以根据不确定度传递规则:
$$
u(n)=\frac{1}{cos^2\theta_B}u_c(\theta_B)\approx0.00658
$$

故折射率的修正结果为:$n=1.399\pm0.007$

\textbf{实验二:偏振片的偏振化方向角度测量}

\textbf{偏振片编号:53}

\begin{longtable}{|c|c|c|c|c|}
  \caption{偏振片的偏振化方向角度测量实验数据} \label{tab:偏振片的偏振化方向角度测量实验数据} \\
  \hline
    \textbf{光电流} & \textbf{极大} & \textbf{极小} & \textbf{极大} & \textbf{极小} \\
  \hline
    \textbf{光电流极值处角坐标$\beta(\text{度})$} & $66.0^\circ$ & $158.6^\circ$  & $246.1 ^\circ$ & $338.1 ^\circ$ \\
  \hline
    \textbf{等效夹角$\Delta\beta(\text{度})$} & $66.0^\circ$ & $68.6^\circ$  & $66.1^\circ$ & $ 68.1^\circ$ \\
  \hline
    \textbf{光电流} & \textbf{极大} & \textbf{极小} & \textbf{极大} & \textbf{极小} \\
  \hline
    \textbf{光电流极值处角坐标$\beta(\text{度})$} & $68.1^\circ$ & $158.8^\circ$  & $249.2 ^\circ$ & $338.4 ^\circ$ \\
  \hline
    \textbf{等效夹角$\Delta\beta(\text{度})$} & $68.1^\circ$ & $68.8^\circ$  & $ 69.2^\circ$ & $ 68.4^\circ$ \\
    \hline
\end{longtable}

因此,我们可以得到夹角的平均值为:$\overline{\Delta\beta}=\frac{\sum\Delta\beta_i}{8}=67.9^\circ$

接下来进行不确定度的计算:
$$
u_A=\sqrt{\frac{\sum(\Delta\beta_i-\overline{\Delta\beta})^2}{n(n-1)}}\approx0.426^\circ \;\;\;\;\; u_B=\frac{\Delta_{\text{仪}}}{\sqrt{3}}\approx0.289^\circ
$$
$$
u_c=\sqrt{u_A^2+u_B^2}\approx0.515^\circ
$$

故夹角的修正结果为:$\Delta\beta=67.9^\circ\pm0.6^\circ$

\textbf{实验三:光电流和光强的关系}

\textbf{本底光电流$i'=0.3 \mu A$}

\textbf{根据实验二的测量结果可知,$\Delta\beta=67.9^\circ$}

\begin{longtable}{|c|c|c|c|c|c|c|c|c|}
  \caption{光电流随光强变化结果} \label{tab:光电流随光强变化结果} \\
  \hline
    \textbf{$\beta(\text{度})$} & $70^\circ$ & $80^\circ$ & $90^\circ$ & $100^\circ$ & $110^\circ$ & $120^\circ$ & $130^\circ$ & $140^\circ$ \\
  \hline
    \textbf{光电流i($\mu A$)} & 275 & 268 & 244 & 208 & 164.0 & 114.8 & 69.3 & 31.2 \\
  \hline
    \textbf{$\Phi=\beta-\Delta\beta(\text{度})$} & 2.1 & 12.1 & 22.1 & 32.1 & 42.1 & 52.1 & 62.1 & 72.1 \\
  \hline
    \textbf{$100 \times cos^2(\Phi)$} & 99.9 & 95.6 & 85.8 & 71.8 & 55.0 & 37.8 & 21.9 & 9.5  \\
  \hline
    \textbf{去本底光电流$i-i'(\mu A)$} & 274.7 & 267.7 & 243.7 & 207.7 & 163.7 & 114.5 & 69.0 & 30.9  \\
  \hline
    
\end{longtable}

做$i-i'$与$100 \cdot cos^2\Phi$的关系图如下:

\begin{figure}[H]
    \centering
    \includegraphics[width=0.70\textwidth]{../images/output.png}
    \caption{拟合曲线图}
    \label{fig:output}
\end{figure}

根据拟合结果我们可以看到决定系数$R^2=0.9984$,说明$i-i'$与$100 \cdot cos^2\Phi$之间存在线性关系,结合马吕斯定律$
I = I_0 \cos^2 \phi$,我们可以得出结论:光电流$i-i'$与入射光强I成线性关系。

\subsection{误差分析(20分)}
(运用测量误差、相对误差或不确定度等分析实验结果,写出完整的结果表达式,并分析误差原因。)

不确定度的分析在前一部分已经完成,这部分直接分析实验的误差来源。

\textbf{测量黑色平板折射率:}

1.激光束、转台中心轴与探测器可能未严格处于同一平面或未严格共轴,导致转动转台时入射点会发生微小位移,反射光斑偏离探测器中心,引入系统误差。

2.\textbf{消光点判断的主观误差:} 寻找光强最小的角坐标的时候,在一定的范围内光电流大小均未发生变化,这导致我们无法准确地判断光强最小的角坐标,也会对测量结果产生一定的误差。

\textbf{测量偏振片方向:}

1.\textbf{极值位置定位困难:}与第一个实验类似,光电流在极大值和极小值附近随角度变化非常缓慢,我们难以精确锁定极值对应的角度,因此我们的读数具有较大的主观性,会引入一定的误差。

\textbf{光电流与光强关系}

根据拟合结果曲线我们可以发现,拟合直线的截距大约为10微安,这与理论情况存在一点偏差,主要原因可能如下:

1.第二个实验测得的$\Delta \beta$值可能存在一定的误差,这会导致在第三个实验中也引入一定的系统性偏移,导致拟合结果截距不为0。

2.激光功率的不稳定性。激光器的输出功率可能会在测量的过程中发生漂移,导致测量结果产生一定的误差。此外,环境中的杂散光强度也可能发生变化,会引入一定的误差。

\subsection{实验探讨(10分)}
(对实验内容、现象和过程的小结,不超过100字。)

在本次实验中,我深入研究了光的偏振特性及其应用。首先,我利用渐进逼近法测得黑色平板的布儒斯特角约为 $54.45^\circ$,计算得出其折射率 $n = 1.399 \pm 0.007$。随后,我通过极值法确定了偏振片透光轴的等效夹角 $\Delta\beta \approx 67.9^\circ$。在验证马吕斯定律时,尽管拟合曲线存在约 $10.15 \mu A$ 的截距偏差,但去除本底电流后的光电流与 $\cos^2\Phi$ 呈现出极佳的线性关系($R^2 \approx 0.998$)。实验结果直观有力地验证了线偏振光强度随角度变化的物理规律。

\section{思考题(10分)}
(解答教材或讲义或老师布置的思考题,请先写题干,再作答。)

\textbf{1.请简述减小实验误差的方法}

在本次实验中,减小误差的方法主要包括以下几点:首先是双侧测量法,即在转台左右两侧分别测量消光角,取平均值以消除转台偏心差带来的系统误差。其次是采用逐渐逼近法来确定布儒斯特角,因为极小值附近光强变化平缓,通过反复微调寻找“最暗点”比单次读取更加准确。此外,在验证马吕斯定律时,扣除了本底电流的影响,以消除环境杂散光对线性关系的干扰。最后,实验前需仔细调节激光器、转台和探测器,确保它们严格等高共轴。

\textbf{2.如何调节激光通过转台的中心转轴}

打开激光器,调节黑色平板正对激光器,调节激光器的位置,使得光束大致打在黑色平板的中心转轴位置,接着将黑色平板旋转一定角度,如果光点的位置基本不变,说明激光已经通过转台的中心转轴了;如果光点发生了偏移,则再调节激光器的位置与方向;重复转动黑色平板和调节激光器的过程,直至光点的位置基本不发生改变。

\textbf{3.如何调节激光与转台转轴中心轴垂直}

打开激光器,调节黑色平板正对激光器,通过判断经过黑色平板反射回去的光束沿原路返回(即激光器上出现的光点的位置是否是光束出射的位置),来判断激光器与转台转轴中心轴是否垂直。调节激光器后方的螺丝,直至二者重合,此时激光器与转台转轴中心轴垂直。

\textbf{4.在探究光电流与光照强度关系时背景光照的影响是怎样的}

背景光照主要来源于实验室的照明灯或窗外射入的自然光,它会以本底电流的形式叠加在测量值上。在验证马吕斯定律时,理论公式为 $I = I_0 \cos^2 \Phi$,是一条过原点的直线。如果存在背景光照且未被扣除,实验测得的电流变为 $I_{测} = I_0 \cos^2 \Phi + i_{背景}$。这在数学上表现为拟合直线的截距不为零,导致直线整体向上平移,不再经过原点。虽然这不影响对线性规律的判断,但会带来明显的系统误差,使得实验结果无法精确吻合理论模型。

\vspace{3\baselineskip}

\input{注意事项}
\end{document}
