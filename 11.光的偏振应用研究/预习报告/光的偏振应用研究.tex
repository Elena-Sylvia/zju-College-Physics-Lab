\documentclass{Preport}
\usepackage{longtable}
\usepackage{multirow}      % 用于跨行
\usepackage{float} % 在导言区添加
\usepackage{amsmath}  % 必须加这一行!
\usepackage{ctex}     % 如果要显示中文,加上这一行

\usepackage{hyperref}
\hypersetup{
    colorlinks=true,
}

\exname{光的偏振应用研究} %实验名称
\instructor{姚星星老师} %指导教师
\class{-} %班级
\name{-} %姓名
\stuid{-} %学号

\nyear{2025} %年
\nmonth{12} %月
\nday{8} %日
\nweekday{一} %星期几

\begin{document}
\setcounter{page}{0}
\makecover

\section{预习报告(10分)}
\subsection{实验综述(5分)}
(自述实验现象、实验原理和实验方法,包括必要的光路图、电路图、公式等。不超过500字。)

\textbf{光的偏振及其应用研究}实验旨在通过观测光在介质界面的反射特性,深入理解光的横波性质。本次实验的核心在于利用布儒斯特角原理测定介质的折射率,并验证线偏振光遵循马吕斯定律。

\vspace{1\baselineskip}

\textbf{实验原理:}

\textbf{1. 菲涅尔公式与偏振光的反射特性}

光波作为电磁波,在两种介质界面(如空气与玻璃)反射时,其反射特性强烈依赖于光的偏振态。如图1所示,将入射光矢量分解为两个正交分量:

\begin{itemize}
    \item \textbf{S分量(图1-1):} 光矢量垂直于入射面(垂直纸面,用圆点表示)。
    \item \textbf{P分量(图1-2):} 光矢量平行于入射面(在纸面内,用箭头表示)。
\end{itemize}

\begin{figure}[H]
    \centering
    \includegraphics[width=0.75\textwidth]{../images/光的偏振原理图.png}
    \caption{菲涅尔公式原理图}
    \label{fig:菲涅尔公式原理图}
\end{figure}

根据菲涅尔公式,P分量的反射振幅比为:

\begin{equation}
    \frac{E_1}{E_0} = \frac{\tan(\theta_0 - \theta_2)}{\tan(\theta_0 + \theta_2)}
\end{equation}

其中 $\theta_0$ 为入射角,$\theta_2$ 为折射角。

当入射角满足 $\theta_0 + \theta_2 = 90^\circ$ 时,公式分母趋于无穷大,反射系数为零。这意味着此时反射光中不包含P分量,只有S分量,反射光变为完全线偏振光。

该特定入射角称为\textbf{布儒斯特角}(Brewster's Angle, $\theta_B$),此时满足:

\begin{equation}
    \tan \theta_B = \frac{n_2}{n_1} = n \quad (\text{空气中 } n_1 \approx 1)
\end{equation}

实验通过测量使反射光中P分量消失的入射角 $\theta_B$,即可求得折射率 $n$。

\textbf{2. 马吕斯定律(Malus's Law)}

当线偏振光(光强为 $I_0$)透过检偏器时,若线偏振光的振动方向与检偏器透光轴的夹角为 $\phi$,则透射光强 $I$ 满足:
\[
I = I_0 \cos^2 \phi
\]

实验中利用硅光电池测量光电流 $i$。由于在一定范围内光电流与入射光强成正比($i \propto I$),通过测量光电流 $i$ 随转角 $\phi$ 的变化,即可验证该线性关系。


\vspace{1\baselineskip}

\textbf{实验方法:}

\textbf{1. 测量黑色平板的折射率(逐渐逼近法)}
\begin{itemize}
    \item 调节激光器、转盘中心、黑色平板及光探测器等高共轴。
    \item 旋转激光器(或加起偏器),使入射光为平行于入射面的线偏振光($P$光)。
    \item 转动载物台改变入射角,观察反射光强。当反射光强达到最弱(消光)时,记录刻度盘读数。
    \item 为消除偏心差,在转台左右两侧分别测量布儒斯特角对应的位置读数 $\theta_{L}$ 和 $\theta_{R}$,则布儒斯特角 $\theta_B$ 可由相关几何关系(如 $|\theta_L - \theta_R|/2$ 或根据具体仪器零点公式)计算得出,进而求出 $n = \tan \theta_B$。
\end{itemize}

\textbf{2. 验证光电流与光强的关系}
\begin{itemize}
    \item 固定入射角为布儒斯特角,此时反射光为垂直于入射面振动的线偏振光($S$光)。
    \item 在反射光路中加入偏振片。旋转偏振片,测量并记录光电流出现极大值 $i_{\max}$ 和极小值 $i_{\min}$ 的角度,确定偏振片的透光轴方向;随后改变偏振片角度 $\phi$(如每隔 $10^\circ$),记录对应的光电流 $i$。同时测量遮挡光路时的本底电流 $i'$。
    \item 绘制 $(i - i') \sim \cos^2 \phi$ 曲线,通过观察曲线是否拟合为通过原点的直线来验证马吕斯定律。
\end{itemize}

\subsection{实验重点(3分)}
(简述本实验的学习重点,不超过100字。)

1. 深刻理解光的偏振态(线偏振、自然光),以及布儒斯特角产生线偏振光(p分量反射为0)的物理机制。

2. 掌握光学实验中的共轴调节技术,以及利用“逐渐逼近法”准确寻找反射光强极小值位置的方法。

3. 学会处理实验数据,通过扣除本底电流和作图分析,验证光电流与偏振角余弦平方的线性关系(马吕斯定律)。

\subsection{实验难点(2分)}
(简述本实验的实现难点,不超过100字。)

1. \textbf{极小值的精确判断:} 在布儒斯特角附近反射光强变化平缓,肉眼或仪器难以直接锁定唯一的最小值点,需通过测量光强对称点来辅助定位,操作要求高。

2. \textbf{系统误差控制:} 激光器、转台与检测器的共轴等高调节直接影响角度测量的准确性;此外,环境背景光会引入本底电流,若不进行有效遮蔽或修正,将导致马吕斯定律验证曲线偏离原点。

\vspace{3\baselineskip}

\input{注意事项}
\end{document}
