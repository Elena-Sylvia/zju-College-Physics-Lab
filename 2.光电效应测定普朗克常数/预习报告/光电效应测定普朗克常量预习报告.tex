\documentclass{Preport}

\usepackage{hyperref}
\hypersetup{
    colorlinks=true,
}

\exname{光电效应测定普朗克常量} %实验名称
\instructor{张妍老师} %指导教师
\class{-} %班级
\name{-} %姓名
\stuid{-} %学号

\nyear{2025} %年
\nmonth{9} %月
\nday{29} %日
\nweekday{一} %星期几

\begin{document}
\setcounter{page}{0}
\makecover

\section{预习报告(10分)}
\subsection{实验综述(5分)}
(自述实验现象、实验原理和实验方法,包括必要的光路图、电路图、公式等。不超过500字。)

光电效应测定普朗克常量实验,旨在帮助我们理解光电效应方程和光量子概念,在掌握理论知识的基础上,进一步掌握光电效应试验方法并验证相关的实验现象,并最终测定普朗克常数的准确值,同时对饱和光电流强度的影响因素进行探究。

\textbf{实验现象:}当不同频率的单色光照射在金属阴极上时,会有电子在瞬时逸出,形成光电流。在实验过程中,我们会观察到:

\begin{itemize}[leftmargin=4em]
    \item 电子的动能随入射光频率增加而增大,而与光强无关;
    \item 当光频率低于某一阈值时,无光电子发射;
    \item 光电流强度与光强成正比
\end{itemize}

这些现象不能用经典波动理论解释,而是符合爱因斯坦提出的\textbf{光量子假说}。

\vspace{1\baselineskip}
\textbf{实验原理:} 本实验基于爱因斯坦光电效应方程:
$$
h\nu = \frac{1}{2}mv^2 + W = eU_a + W
$$

其中 (h) 为普朗克常数,($\nu$) 为光频率,(W) 为金属的逸出功,($U_a$) 为截止电压。

\begin{figure}[htbp]
    \centering
    \begin{minipage}[t]{0.4\linewidth}
        \centering
        \includegraphics[width=\linewidth]{../images/不同光强下UAK-I曲线.png} 
        \caption{不同光强下$U_{AK}-i$曲线}
        \label{fig:UAK-i}
    \end{minipage}
    \hfill
    \begin{minipage}[t]{0.4\linewidth}
        \centering
        \includegraphics[width=\linewidth]{../images/v-Ua曲线.png} 
        \caption{$\nu-U_a$关系图}
        \label{fig:v-Ua}
    \end{minipage}
    \label{fig:combined}
\end{figure}

由此我们可以得出截止电压与频率成线性关系:
$$
U_a = \frac{h}{e}\nu - \frac{W}{e}
\nu_0 = \frac{W}{h}
$$

通过测量不同频率下的截止电压并进行线性拟合,即可求得斜率 (h/e),从而得到普朗克常数。

实验装置由汞灯光源、滤光片(提供多组单色光)、光电管、可调直流电源、微安表和电压表组成。
\begin{itemize}[leftmargin=4em]
    \item 光路:汞灯发射特征谱线,经滤色片选择单色光,照射光电管阴极产生光电子
    \item 电路:通过调节阳极电压,测量在不同光频率下电流随电压变化的伏安特性曲线,并使用零电流法确定截止电压
\end{itemize}

\vspace{1\baselineskip}
\textbf{实验方法:}
\begin{enumerate}[leftmargin=4em]
    \item 预热光源并调零仪器
    \item 用不同滤光片选择不同的波长,测定光电流随电压的变化,并做出伏安特性曲线
    \item 使用零电流法确定截止电压
    \item 线性拟合截止电压与频率,求出普朗克常数、功函数和红限频率
    \item 改变光源距离与光阑孔径,探究光饱和电流强度的影响因素
\end{enumerate}

\subsection{实验重点(3分)}
(简述本实验的学习重点,不超过100字。)

本实验的学习重点是理解光电效应的基本规律,包括电子动能与入射光频率的线性关系及与光强无关的特性;掌握截止电压的测定方法;学会线性拟合不同频率下的实验数据,从而求得普朗克常数和金属材料的功函数,并探究光饱和电流强度的影响因素,进一步加深对量子理论和光的粒子性的认识。

\subsection{实验难点(2分)}
(简述本实验的实现难点,不超过100字。)

本实验的难点在于准确测定截止电压,需有效消除暗电流、本底电流及仪器零点漂移等的影响;在实际操作中,我们还可能面临光源不稳定、滤光片选择有限以及数据线性拟合精度不高等问题。


\vspace{3\baselineskip}
\input{注意事项}
\end{document}
