\documentclass{Preport}
\usepackage{longtable}
\usepackage{multirow}      % 用于跨行
\usepackage{float} % 在导言区添加
\usepackage{amsmath}  % 必须加这一行!
\usepackage{ctex}     % 如果要显示中文,加上这一行

\usepackage{hyperref}
\hypersetup{
    colorlinks=true,
}

\exname{双棱镜干涉实验} %实验名称
\instructor{殷立明老师} %指导教师
\class{-} %班级
\name{-} %姓名
\stuid{-} %学号

\nyear{2025} %年
\nmonth{11} %月
\nday{3} %日
\nweekday{一} %星期几

\begin{document}
\setcounter{page}{0}
\makecover

\section{预习报告(10分)}
\subsection{实验综述(5分)}
(自述实验现象、实验原理和实验方法,包括必要的光路图、电路图、公式等。不超过500字。)

\textbf{双棱镜干涉实验}旨在利用光的干涉现象来测定光的波长。该实验利用双棱镜形成的两个相干光束进行干涉,通过观察干涉条纹的变化,推算出光波的波长。与传统的单棱镜干涉法相比,双棱镜干涉法能够在较短的距离内产生更明显的干涉条纹,且具备更高的测量精度。

\textbf{实验现象:}

在实验中,单色光源通过双棱镜后被分成两束相干光束,这两束光经过不同的路径后重新合并,形成干涉条纹。当调节其中一束光的路径时,可以观察到干涉条纹的变化,条纹的增大或减小可以通过示波器进行记录,进而我们可以计算出光的波长。

\textbf{实验原理:}

\begin{figure}[htbp]
    \centering
    \begin{minipage}[t]{0.45\linewidth}
        \centering
        \includegraphics[width=\linewidth]{../images/光路图1.png} 
        \caption{双棱镜干涉原理}
        \label{fig:双棱镜干涉原理}
    \end{minipage}
    \hfill
    \begin{minipage}[t]{0.45\linewidth}
        \centering
        \includegraphics[width=\linewidth]{../images/光路图2.png} 
        \caption{光波波长测量原理}
        \label{fig:光波波长测量原理}
    \end{minipage}
    \label{fig:combined}
\end{figure}

1. 双棱镜干涉原理

如图1所示,He-Ne 激光发出的光波会聚于狭缝 S,当狭缝 S 发出的光波投射到双棱镜 MN 上时,其波面被分割成两部分,通过双棱镜来观察这两束光波,就好像它们是由虚光源 $S_1$ 和 $S_2$发出的一样。所以在两束光相互叠加区域 $P'P''$ 内产生干涉现象。

2. 光波波长测量原理

如图2所示,$\delta$ 是 $S_1$ 和 $S_2$ 到毛玻璃屏上某一相干点 $P_k$ 的光程差。$x_{k+1}$ 和 $x_k$ 分别为 $S_1$ 和 $S_2$ 到毛玻璃屏上相邻两相干点 $P_{k+1}$ 和 $P_k$ 分别到 $P_0$ 的距离。当 $D \gg d, D \gg x_k$ 时,有:
$$
\delta = \frac{x_k}{D} d 
$$
所以,两相邻亮条纹的间距为:
$$
\Delta x = x_{k+1} - x_k = \frac{D}{d}\lambda\;, \;\lambda = \frac{\Delta x \cdot d}{D}
$$

3.二次成像原理:

\begin{figure}[H]
    \centering
    \includegraphics[width=0.5\textwidth]{../images/光路图3.png}
    \caption{二次成像原理}
    \label{fig:二次成像原理}
\end{figure}

如图3所示,设透镜的焦距为f,则我们可以根据相似关系得到:
$$
\frac{d_1}{d+d_1}=\frac{S_1'-f}{S_1'}
$$
$$
\frac{d_2}{d+d_2}=\frac{S_2'-f}{S_2'}
$$
由上面两式可得,$d=\sqrt{d_1d_2}$

根据透镜成像定理可以推导出,狭缝到屏的距离D为:
$$
D=\frac{\sqrt{d_2}+\sqrt{d_1}}{\sqrt{d_2}-\sqrt{d_1}}f
$$

\textbf{实验装置与方法:}

实验系统包括光源、双棱镜、光屏、调节器和干涉观察系统。光源发出的单色光经过双棱镜分成两束相干光,光束经过不同的路径后再汇合,形成干涉条纹。实验中,通过调节双棱镜的位置或观察屏上的条纹的位置,测量干涉条纹的间距,从而推算出光源的波长。

\subsection{实验重点(3分)}
(简述本实验的学习重点,不超过100字。)

1.领会分波面法干涉实验原理

2.了解双棱镜干涉装置及光路调节技巧

3.观察双棱镜干涉现象并测定光波波长

\subsection{实验难点(2分)}
(简述本实验的实现难点,不超过100字。)

本实验的难点主要在于干涉条纹的清晰成像与精确测量。由于双棱镜干涉要求两束光具有较高的相干性,实验中光源的位置、狭缝宽度及双棱镜的角度必须精确调整,否则条纹会模糊或者条纹的对比度会下降。此外,虚光源间距的测量与干涉条纹间距的测定均需要较高的精度,稍有误差都会显著影响波长计算结果。


\vspace{3\baselineskip}
\input{注意事项}
\end{document}
