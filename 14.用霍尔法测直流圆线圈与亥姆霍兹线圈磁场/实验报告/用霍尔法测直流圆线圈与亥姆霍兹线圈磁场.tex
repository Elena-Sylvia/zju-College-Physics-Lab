\documentclass{Phyport}
\usepackage{longtable}
\usepackage{multirow}      % 用于跨行
\usepackage{float} % 在导言区添加
\usepackage{amsmath}  % 必须加这一行!
\usepackage{ctex}     % 如果要显示中文,加上这一行

\usepackage{hyperref}
\hypersetup{
    colorlinks=true,
}

\exname{用霍尔法测直流圆线圈与亥姆霍兹线圈磁场} %实验名称
\extable{9} %实验桌号
\instructor{费莹老师} %指导教师
\class{-} %班级
\name{-} %姓名
\stuid{-} %学号

\nyear{2025} %年
\nmonth{12} %月
\nday{29} %日
\nweekday{一} %星期几

\redate{} %如有实验补做,补做日期
\resitu{} %情况说明:

\begin{document}
\setcounter{page}{0}
\makecover

\section{预习报告(10分)}
(注:将已经写好的“物理实验预习报告”内容拷贝过来)

\subsection{实验综述(5分)}
(自述实验现象、实验原理和实验方法,包括必要的光路图、电路图、公式等。不超过500字。)

\textbf{用霍尔法测直流圆线圈与亥姆霍兹线圈磁场}实验旨在利用霍尔效应原理,测量载流圆线圈及亥姆霍兹线圈轴线上的磁场分布,并验证磁场叠加原理及亥姆霍兹线圈产生匀强磁场的条件。

\vspace{1\baselineskip}

\textbf{实验原理:}

\textbf{1. 霍尔效应测磁场原理}

如图1所示,厚度为d的矩形半导体薄片垂直磁场B放置,通有电流I。载流子在洛伦兹力的作用下运动方向发生改变,产生横向偏转,在边界累积产生横向电场E。直到E产生的$F_E$作用与洛伦兹力$F_B$抵消,即:
\begin{equation}
    q \cdot (\vec{v} \times \vec{B}) = q \cdot \vec{E}
\end{equation}

时,电荷不再发生偏转。

由于$I = nqSv = nq\omega dv$,则$v=\frac{I}{nq\omega d}$

因此我们有$U_H = \frac{IB}{nqd}$。记霍尔系数$R_H=\frac{1}{nq}$,霍尔元件的灵敏度$K_H = \frac{R_H}{d} = \frac{1}{nqd}$,则$U_H=K_HIB$。


\begin{figure}[H]
    \centering
    \includegraphics[width=0.5\textwidth]{../images/霍尔效应原理图.png}
    \caption{霍尔效应原理图}
    \label{fig:霍尔效应原理图}
\end{figure}

\textbf{2. 载流圆线圈与亥姆霍兹线圈的磁场}

\begin{itemize}
    \item \textbf{载流圆线圈:} 半径为 $R$,匝数为 $N$,通以电流 $I$ 的圆线圈,其轴线上距圆心 $X$ 处的磁感应强度为:
    \begin{equation}
        B = \frac{\mu_0 N I R^2}{2(R^2 + X^2)^{3/2}}
    \end{equation}
    其分布图为关于圆心对称的单峰曲线。
    
    \item \textbf{亥姆霍兹线圈:} 由两个半径 $R$、匝数 $N$ 相同的线圈平行共轴放置,通以同向等大电流 $I$。当线圈间距 $d = R$ 时,两线圈产生的磁场叠加,在两线圈中心连线区域内磁场导数 $\frac{dB}{dX}=0$ 且 $\frac{d^2B}{dX^2}=0$,形成较宽的\textbf{匀强磁场}区。
\end{itemize}

\begin{figure}[htbp]
    \centering
    \begin{minipage}[t]{0.45\linewidth}
        \centering
        \includegraphics[width=\linewidth]{../images/载流圆线圈.png} 
        \caption{载流圆线圈原理图}
        \label{fig:2:1}
    \end{minipage}
    \hspace{0.05\linewidth}
    \begin{minipage}[t]{0.45\linewidth}
        \centering
        \includegraphics[width=\linewidth]{../images/亥姆霍兹线圈.png} 
        \caption{亥姆霍兹线圈原理图}
        \label{fig:2:3}
    \end{minipage}
    \label{fig:combined_2}
\end{figure}

\vspace{1\baselineskip}

\textbf{实验方法:}

\vspace{1\baselineskip}

\textbf{1. 测量载流圆线圈轴线上的磁场分布}
\begin{itemize}
    \item 连接FB511型实验仪,将霍尔探头固定在测试架上。
    \item \textbf{调零:} 在励磁电流 $I=0$ 时,调节微特斯拉计的调零旋钮,以消除地磁场及不等位电势的影响。
    \item 设置励磁电流 $I=0.400\text{A}$,移动探头,以线圈中心为原点,每隔 $1.0\text{cm}$ 测量一次 $B$ 值,记录数据并绘制 $B-X$ 曲线。
\end{itemize}

\textbf{2. 测量亥姆霍兹线圈轴线上的磁场分布}
\begin{itemize}
    \item 调节两个线圈间距 $d=R$(即移动线圈至刻度 $R$ 处)。
    \item 同样通以 $I=0.400\text{A}$ 电流,以两线圈中心连线中点为原点,沿轴线测量磁场分布。
    \item 观察并记录中心区域磁场出现的“平台”现象,验证匀强磁场特性。
\end{itemize}

\subsection{实验重点(3分)}
(简述本实验的学习重点,不超过100字。)

1.深入理解霍尔效应测量磁场的物理机制,以及亥姆霍兹线圈产生匀强磁场的几何条件($d=R$)。

2.通过实测数据绘制磁场分布曲线,验证载流圆线圈的理论公式,并分析亥姆霍兹线圈中心区域磁场的均匀性。

3.掌握FB511型磁场实验仪的操作,特别是霍尔探头的定标与调零方法。

\subsection{实验难点(2分)}
(简述本实验的实现难点,不超过100字。)

1. \textbf{地磁场与环境干扰的消除:} 霍尔传感器灵敏度高,地磁场及周围铁磁物质会引入误差,因此必须在电流为零时仔细调零,且实验过程中探头方向改变或仪器移动后均需重新调零。

2. \textbf{探头位置的准确定位:} 霍尔探头必须严格处于线圈轴线上并保持垂直于磁场方向,探头的径向或角度偏差会导致测量值 $B$ 偏小。

3. \textbf{微弱变化的观测:} 在亥姆霍兹线圈中心匀强区,磁场变化极小,需仔细读取数据以分辨微小差异。

\section{原始数据(20分)}
(将有老师签名的“自备数据记录草稿纸”的扫描或手机拍摄图粘贴在下方,完整保留姓名,学号,教师签字和日期。)

\begin{figure}[H]
    \centering
    \includegraphics[width=0.85\textwidth]{../images/originaldata.jpg}
    \caption{original data}
    \label{fig:originaldata}
\end{figure}

\section{结果与分析(60分)}
\subsection{数据处理与结果(30分)}
(列出数据表格、选择适合的数据处理方法、写出测量或计算结果。)

\textbf{实验一:测绘单个圆线圈磁场强度分布}

相关参数:\textbf{$N_0 = 400 $匝 , I = 0.4A , R = 0.1m , 线圈位置$x_0=15.00cm$}

其中,理论磁感应强度的计算公式为:$B(x)=\dfrac{\mu_0N_0IR^2}{2(R^2+X^2)^{3/2}}$

\begin{longtable}{|c|c|c|c|c|c|c|c|c|c|c|}
  \caption{单个圆线圈磁场强度分布} \label{tab:单个圆线圈磁场强度分布} \\
  \hline
    \textbf{传感器所在位置(cm)} & 5.00 & 6.00 & 7.00 & 8.00 & 9.00 & 10.00 & 11.00 & 12.00 & 13.00 & 14.00 \\
  \hline
    \textbf{轴向距离(cm)} & -10.00 & -9.00 & -8.00 & -7.00 & -6.00 & -5.00 & -4.00 & -3.00 & -2.00 & -1.00  \\
  \hline
    \textbf{正向磁感应强度$B_{\text{正}}(\mu T)$} & 361 & 418 & 478 & 551 & 628 & 715 & 798 & 876 & 939 & 985 \\
  \hline
    \textbf{反向磁感应强度$B_{\text{反}}(\mu T)$} & -332 & -389 & -454 & -525 & -605 & -686 & -776 & -854 & -925 & -970 \\
  \hline
    \textbf{平均磁感应强度$B(\mu T)$} & 347 & 404 & 466 & 538 & 616 & 700 & 787 & 865 & 932 & 978  \\
  \hline
    \textbf{理论磁感应强度$B_x(\mu T)$} & 355 & 413 & 479 & 553 & 634 & 719 & 805 & 883 & 948 & 990  \\
  \hline
    \textbf{相对误差} & 2.54 & 2.18 & 2.71 & 2.71 & 2.84 & 2.64 & 2.24 & 2.04 & 1.69 & 1.21  \\
    \hline
\end{longtable}

\begin{longtable}{|c|c|c|c|c|c|c|c|c|c|c|c|}
\caption{单个圆线圈磁场强度分布(续表)} \label{tab:coil_part2} \\
\hline
\textbf{传感器所在位置(cm)} & 15.00 & 16.00 & 17.00 & 18.00 & 19.00 & 20.00 & 21.00 & 22.00 & 23.00 & 24.00 & 25.00\\
\hline
\textbf{轴向距离(cm)} & 0.00 & 1.00 & 2.00 & 3.00 & 4.00 & 5.00 & 6.00 & 7.00 & 8.00 & 9.00 & 10.00 \\
\hline
\textbf{正向磁感应强度$B_{\text{正}}(\mu T)$} & 1002 & 994 & 954 & 893 & 814 & 732 & 645 & 565 & 490 & 425 & 367\\
\hline
\textbf{反向磁感应强度$B_{\text{反}}(\mu T)$} & -986 & -976 & -937 & -879 & -804 & -719 & -636 & -556 & -481 & -415 & -360 \\
\hline
\textbf{平均磁感应强度$B(\mu T)$} & 994 & 985 & 946 & 886 & 809 & 726 & 640 & 560 & 486 & 420 & 364 \\
\hline
\textbf{理论磁感应强度$B_x(\mu T)$} & 1005 & 990 & 948 & 883 & 805 & 719 & 634 & 553 & 479 & 413 & 355 \\
\hline
\textbf{相对误差} & 1.09 & 0.51 & 0.21 & 0.34 & 0.5 & 0.97 & 0.95 & 1.27 & 1.46 & 1.69 & 2.54 \\
\hline
\end{longtable}

使用matplotlib绘制出了B-x 和 $B_x$-x曲线,如下图所示:

\begin{figure}[H]
    \centering
    \includegraphics[width=0.55\textwidth]{../images/output1.png}
    \caption{B-x 和 $B_x$-x曲线}
    \label{fig:output1}
\end{figure}

从图中我们可以看出,曲线呈现关于x=0对称的单峰分布,磁感应强度在x=0处达到最大值,并随轴向距离的增加而逐渐减小。实验测量曲线与理论计算曲线走势高度吻合,很好地验证了载流圆线圈轴向磁场分布的理论规律。

\textbf{实验二:测绘单个圆线圈径向磁场强度分布}

\begin{longtable}{|c|c|c|c|c|c|c|c|c|c|c|c|}
  \caption{单个圆线圈径向磁场强度分布} \label{tab:单个圆线圈磁场径向强度分布} \\
  \hline
    \textbf{径向距离y(cm)} & -5.00 & -4.00 & -3.00 & -2.00 & -1.00 & 0.00 & 1.00 & 2.00 & 3.00 & 4.00 & 5.00 \\
  \hline
    \textbf{正向磁感应强度$B_{\text{正}}(\mu T)$} & 1235 & 1140 & 1076 & 1038 & 1013 & 1008 & 1017 & 1043 & 1088 & 1161 & 1269 \\
  \hline
    \textbf{反向磁感应强度$B_{\text{反}}(\mu T)$} & -1208 & -1110 & -1046 & -1008 & -984 & -975 & -985 & -1012 & -1051 & -1124 & -1230 \\
  \hline
    \textbf{平均磁感应强度$B(\mu T)$} & 1222 & 1125 & 1061 & 1023 & 998 & 992 & 1001 & 1028 & 1070 & 1142 & 1250 \\
  \hline
\end{longtable}

\begin{figure}[H]
    \centering
    \includegraphics[width=0.55\textwidth]{../images/output2.png}
    \caption{单个圆线圈径向磁场强度分布}
    \label{fig:output2}
\end{figure}

我们可以发现,曲线呈现关于中心呈高度对称的凹形分布,磁感应强度在中心处达到最小值;随着径向距离向两侧增加(即从圆心向线圈边缘移动),磁感应强度逐渐增大;这符合载流圆线圈内部的磁场分布规律,即越靠近电流所在的导线,磁场越强。

\textbf{实验三:测绘亥姆霍兹线圈的轴向磁场强度分布}

相关参数:\textbf{$N_0 = 400 $匝 , I = 0.4A , R = 0.1m , $x_1=10.00cm$,$x_2=10.00cm$}

\begin{longtable}{|c|c|c|c|c|c|c|c|c|c|c|}
  \caption{亥姆霍兹线圈磁场强度分布} \label{tab:亥姆霍兹线圈磁场强度分布} \\
  \hline
    \textbf{传感器所在位置(cm)} & 5.00 & 6.00 & 7.00 & 8.00 & 9.00 & 10.00 & 11.00 & 12.00 & 13.00 & 14.00 \\
  \hline
    \textbf{轴向距离(cm)} & -10.00 & -9.00 & -8.00 & -7.00 & -6.00 & -5.00 & -4.00 & -3.00 & -2.00 & -1.00  \\
  \hline
    \textbf{正向磁感应强度$B_{\text{正}}(\mu T)$} & 876 & 983 & 1094 & 1190 & 1276 & 1341 & 1388 & 1408 & 1424 & 1423 \\
  \hline
    \textbf{反向磁感应强度$B_{\text{反}}(\mu T)$} & -880 & -988 & -1093 & -1196 & -1279 & -1348 & -1389 & -1418 & -1425 & -1435 \\
  \hline
    \textbf{平均磁感应强度$B(\mu T)$} & 878 & 986 & 1094 & 1193 & 1278 & 1344 & 1388 & 1413 & 1424 & 1429 \\
  \hline
\end{longtable}

\begin{longtable}{|c|c|c|c|c|c|c|c|c|c|c|c|}
\caption{亥姆霍兹线圈磁场强度分布(续表)} \label{tab:亥姆霍兹线圈磁场强度分布} \\
\hline
\textbf{传感器所在位置(cm)} & 15.00 & 16.00 & 17.00 & 18.00 & 19.00 & 20.00 & 21.00 & 22.00 & 23.00 & 24.00 & 25.00\\
\hline
\textbf{轴向距离(cm)} & 0.00 & 1.00 & 2.00 & 3.00 & 4.00 & 5.00 & 6.00 & 7.00 & 8.00 & 9.00 & 10.00 \\
\hline
\textbf{正向磁感应强度$B_{\text{正}}(\mu T)$} & 1423 & 1422 & 1423 & 1412 & 1390 & 1350 & 1290 & 1209 & 1114 & 1006 & 894\\
\hline
\textbf{反向磁感应强度$B_{\text{反}}(\mu T)$} & -1432 & -1433 & -1428 & -1420 & -1401 & -1358 & -1295 & -1216 & -1115 & -1010 & -900 \\
\hline
\textbf{平均磁感应强度$B(\mu T)$} & 1428 & 1428 & 1426 & 1416 & 1396 & 1354 & 1292 & 1212 & 1114 & 1008 & 897 \\
\hline
\end{longtable}

\begin{figure}[H]
    \centering
    \includegraphics[width=0.55\textwidth]{../images/output3.png}
    \caption{亥姆霍兹线圈磁场强度分布}
    \label{fig:output3}
\end{figure}

在中心附近,磁感应强度数值变化极小,形成了一个明显的平台。这有力地验证了亥姆霍兹线圈的特性:当两线圈间距等于半径时,能在其轴线中心产生范围较宽的匀强磁场。

\subsection{误差分析(20分)}
(运用测量误差、相对误差或不确定度等分析实验结果,写出完整的结果表达式,并分析误差原因。)

1.我们可以注意到左右两侧距离中心相同距离的点,在先后两次的测量中获得的磁感应强度并不相同,这可能是由于环境中的磁场发生了变化,特别是实验教室中的其他仪器产生的磁场干扰。

2.实验过程中我们也发现,线圈在导轨上的固定其实不是很牢固,我们在移动线圈后,仅是通过肉眼观察是否竖直,可能实际上线圈并未达到竖直,这也会对实验结果产生一定的误差。

3.调零的时候,读数会一直在跳动,有时因为环境磁场的干扰,读数的扰动会非常大,这导致我们的调零可能并不是很准确;同时从调零时读数的跳动我们也可以看出,在实验过程的实际读数中,也会存在一定的误差。

4.调节霍尔元件的位置时,我们对刻度尺位置的判断可能存在一定的误差。

\subsection{实验探讨(10分)}
(对实验内容、现象和过程的小结,不超过100字。)

本次实验中,我利用霍尔效应测量了载流圆线圈及亥姆霍兹线圈的磁场分布情况。测量结果表明:单线圈的轴向磁场呈关于圆心对称的单峰分布,而径向磁场强度则由中心向外递增;亥姆霍兹线圈轴线中心区域出现宽平的“平台”,验证了其能够在一定范围内产生匀强磁场的特性;我的实验结果与理论分析高度一致。

\section{思考题(10分)}
(解答教材或讲义或老师布置的思考题,请先写题干,再作答。)

\textbf{思考题一:为什么在测量直流磁场时,必须考虑地球磁场对被测磁场的影响}

地球本身是一个巨大的磁体,地磁场在实验环境中的水平分量通常在几十微特斯拉的量级。由于霍尔传感器测量的是空间中磁感应强度的矢量叠加值,微特斯拉计的读数实际上是实验线圈产生的磁场与地磁场(以及环境中其他杂散磁场)之和。在测量弱直流磁场时,地磁场的大小可能与被测磁场处于同一数量级,若不加以消除,它会给测量结果引入显著的系统误差。因此,在实验开始前必须在励磁电流为零的状态下进行调零,或者采用改变电流方向测量正反两组数据取平均值的方法,以有效抵消地磁场的影响,确保测量结果真实反映线圈本身产生的净磁场。

\textbf{思考题二:亥姆霍兹线圈是怎样组成的? 其基本条件有哪些? 它的磁场分布特点又怎样? 改变两圆线圈间距后,线圈轴线上的磁场分布情况如何?}

亥姆霍兹线圈由两个完全相同的圆形线圈组成,其构建的基本条件要求两个线圈彼此平行且共轴放置,具有相同的半径R和匝数,通以大小相等且方向相同的电流,最关键的是两线圈之间的距离d必须严格等于线圈半径R。

在这种特定的几何结构下,两线圈产生的磁场在中心区域相互叠加,使得中心点磁感应强度的一阶导数和二阶导数均为零,从而在两线圈中心连线的中点附近形成了一个范围较宽、均匀度极高的匀强磁场区

若改变间距,当\textbf{间距d<R}时,合磁场呈现中心高、两侧低的单峰分布,中心磁场变强但均匀范围变窄;当\textbf{间距d>R}时,合磁场呈双峰(马鞍形)分布,即在靠近中心点的过程中,磁场强度大小先变大后变小(不会在变大后保持不变),中心处出现极小值,不再具备匀强磁场的特性。

\textbf{思考题三:霍尔元件放入磁场时,不同方向上特斯拉计指示值不同,哪个方向最大?}

霍尔传感器是利用霍尔效应工作的,其产生的霍尔电压 $U_H$大小直接正比于流经元件的电流、灵敏度系数以及垂直于霍尔元件平面的磁感应强度分量。这意味着霍尔元件只能检测到与其表面法线方向平行的磁场分量。当探头在磁场中旋转时,磁场线与霍尔元件平面的夹角发生变化,导致有效磁分量改变,从而使示数发生变化。\textbf{当霍尔元件的平面与磁力线方向完全垂直时}(此时载流子受到的洛伦兹力最大,且全部用于建立霍尔电场),有效磁感应强度达到最大值,因此在这个方向上特斯拉计的指示值最大。

\vspace{3\baselineskip}
\input{注意事项}
\end{document}
