\documentclass{Preport}
\usepackage{longtable}
\usepackage{multirow}      % 用于跨行
\usepackage{float} % 在导言区添加
\usepackage{amsmath}  % 必须加这一行!
\usepackage{ctex}     % 如果要显示中文,加上这一行

\usepackage{hyperref}
\hypersetup{
    colorlinks=true,
}

\exname{用霍尔法测直流圆线圈与亥姆霍兹线圈磁场} %实验名称
\instructor{费莹老师} %指导教师
\class{-} %班级
\name{-} %姓名
\stuid{-} %学号

\nyear{2025} %年
\nmonth{12} %月
\nday{29} %日
\nweekday{一} %星期几

\begin{document}
\setcounter{page}{0}
\makecover

\section{预习报告(10分)}
\subsection{实验综述(5分)}
(自述实验现象、实验原理和实验方法,包括必要的光路图、电路图、公式等。不超过500字。)

\textbf{用霍尔法测直流圆线圈与亥姆霍兹线圈磁场}实验旨在利用霍尔效应原理,测量载流圆线圈及亥姆霍兹线圈轴线上的磁场分布,并验证磁场叠加原理及亥姆霍兹线圈产生匀强磁场的条件。

\vspace{1\baselineskip}

\textbf{实验原理:}

\textbf{1. 霍尔效应测磁场原理}

如图1所示,厚度为d的矩形半导体薄片垂直磁场B放置,通有电流I。载流子在洛伦兹力的作用下运动方向发生改变,产生横向偏转,在边界累积产生横向电场E。直到E产生的$F_E$作用与洛伦兹力$F_B$抵消,即:
\begin{equation}
    q \cdot (\vec{v} \times \vec{B}) = q \cdot \vec{E}
\end{equation}

时,电荷不再发生偏转。

由于$I = nqSv = nq\omega dv$,则$v=\frac{I}{nq\omega d}$

因此我们有$U_H = \frac{IB}{nqd}$。记霍尔系数$R_H=\frac{1}{nq}$,霍尔元件的灵敏度$K_H = \frac{R_H}{d} = \frac{1}{nqd}$,则$U_H=K_HIB$。


\begin{figure}[H]
    \centering
    \includegraphics[width=0.5\textwidth]{../images/霍尔效应原理图.png}
    \caption{霍尔效应原理图}
    \label{fig:霍尔效应原理图}
\end{figure}

\textbf{2. 载流圆线圈与亥姆霍兹线圈的磁场}

\begin{itemize}
    \item \textbf{载流圆线圈:} 半径为 $R$,匝数为 $N$,通以电流 $I$ 的圆线圈,其轴线上距圆心 $X$ 处的磁感应强度为:
    \begin{equation}
        B = \frac{\mu_0 N I R^2}{2(R^2 + X^2)^{3/2}}
    \end{equation}
    其分布图为关于圆心对称的单峰曲线。
    
    \item \textbf{亥姆霍兹线圈:} 由两个半径 $R$、匝数 $N$ 相同的线圈平行共轴放置,通以同向等大电流 $I$。当线圈间距 $d = R$ 时,两线圈产生的磁场叠加,在两线圈中心连线区域内磁场导数 $\frac{dB}{dX}=0$ 且 $\frac{d^2B}{dX^2}=0$,形成较宽的\textbf{匀强磁场}区。
\end{itemize}

\begin{figure}[htbp]
    \centering
    \begin{minipage}[t]{0.45\linewidth}
        \centering
        \includegraphics[width=\linewidth]{../images/载流圆线圈.png} 
        \caption{载流圆线圈原理图}
        \label{fig:2:1}
    \end{minipage}
    \hspace{0.05\linewidth}
    \begin{minipage}[t]{0.45\linewidth}
        \centering
        \includegraphics[width=\linewidth]{../images/亥姆霍兹线圈.png} 
        \caption{亥姆霍兹线圈原理图}
        \label{fig:2:3}
    \end{minipage}
    \label{fig:combined_2}
\end{figure}

\vspace{1\baselineskip}

\textbf{实验方法:}

\vspace{1\baselineskip}

\textbf{1. 测量载流圆线圈轴线上的磁场分布}
\begin{itemize}
    \item 连接FB511型实验仪,将霍尔探头固定在测试架上。
    \item \textbf{调零:} 在励磁电流 $I=0$ 时,调节微特斯拉计的调零旋钮,以消除地磁场及不等位电势的影响。
    \item 设置励磁电流 $I=0.400\text{A}$,移动探头,以线圈中心为原点,每隔 $1.0\text{cm}$ 测量一次 $B$ 值,记录数据并绘制 $B-X$ 曲线。
\end{itemize}

\textbf{2. 测量亥姆霍兹线圈轴线上的磁场分布}
\begin{itemize}
    \item 调节两个线圈间距 $d=R$(即移动线圈至刻度 $R$ 处)。
    \item 同样通以 $I=0.400\text{A}$ 电流,以两线圈中心连线中点为原点,沿轴线测量磁场分布。
    \item 观察并记录中心区域磁场出现的“平台”现象,验证匀强磁场特性。
\end{itemize}

\subsection{实验重点(3分)}
(简述本实验的学习重点,不超过100字。)

1.深入理解霍尔效应测量磁场的物理机制,以及亥姆霍兹线圈产生匀强磁场的几何条件($d=R$)。

2.通过实测数据绘制磁场分布曲线,验证载流圆线圈的理论公式,并分析亥姆霍兹线圈中心区域磁场的均匀性。

3.掌握FB511型磁场实验仪的操作,特别是霍尔探头的定标与调零方法。

\subsection{实验难点(2分)}
(简述本实验的实现难点,不超过100字。)

1. \textbf{地磁场与环境干扰的消除:} 霍尔传感器灵敏度高,地磁场及周围铁磁物质会引入误差,因此必须在电流为零时仔细调零,且实验过程中探头方向改变或仪器移动后均需重新调零。

2. \textbf{探头位置的准确定位:} 霍尔探头必须严格处于线圈轴线上并保持垂直于磁场方向,探头的径向或角度偏差会导致测量值 $B$ 偏小。

3. \textbf{微弱变化的观测:} 在亥姆霍兹线圈中心匀强区,磁场变化极小,需仔细读取数据以分辨微小差异。

\vspace{3\baselineskip}

\input{注意事项}
\end{document}
