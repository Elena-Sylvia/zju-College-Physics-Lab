\documentclass{Preport}
\usepackage{longtable}
\usepackage{multirow}      % 用于跨行
\usepackage{float} % 在导言区添加
\usepackage{amsmath}  % 必须加这一行!
\usepackage{ctex}     % 如果要显示中文,加上这一行

\usepackage{hyperref}
\hypersetup{
    colorlinks=true,
}

\exname{惠斯登电桥} %实验名称
\instructor{郑昕颖老师} %指导教师
\class{-} %班级
\name{-} %姓名
\stuid{-} %学号

\nyear{2025} %年
\nmonth{11} %月
\nday{17} %日
\nweekday{一} %星期几

\begin{document}
\setcounter{page}{0}
\makecover

\section{预习报告(10分)}
\subsection{实验综述(5分)}
(自述实验现象、实验原理和实验方法,包括必要的光路图、电路图、公式等。不超过500字。)

\textbf{惠斯登电桥的实验}旨在通过自组惠斯登电桥和使用盒式惠斯登电桥,测量未知电阻并进行误差分析,从而掌握惠斯登电桥的工作原理和特点。惠斯登电桥是一种经典的直流平衡单臂电桥,广泛应用于中等数值电阻($10^1-10^6\Omega$)的精确测量。它通过比较法在平衡条件下确定待测电阻的阻值,是众多电学实验的基础。

\textbf{实验现象:}

在自组电桥的实验中,当电桥接近平衡时,检流计的指针偏转会逐渐减小,直至指示为零。通过调节比较臂电阻$R_s$和比率臂$R_1/R_2$,可以观察到电桥的灵敏度对平衡判断的影响。在交换法测量中,互换待测电阻和比较臂电阻的位置后,需要重新调节以达到平衡,这消除了R1、R2自身的误差对测量结果的影响。使用盒式惠斯登电桥时,通过旋钮的调节,可以快速实现电桥平衡,并且可以观察到在不同的倍率设置下,测量的范围和精度会有所不同。

\vspace{1\baselineskip}

\textbf{实验原理:}

\textbf{1. 惠斯登电桥测量电阻的原理}

下图是惠斯登电桥的原理图。

\begin{figure}[H]
    \centering
    \includegraphics[width=0.6\textwidth]{../images/实验原理图.png}
    \caption{惠斯登电桥原理图}
    \label{fig:惠斯登电桥原理图}
\end{figure}

当通过检流计 $G$ 的电流 $I_g$ 等于零时,B、D 两点电位相同,电桥达到平衡。此时,流过电阻 $R_1$ 和 $R_x$ 的电流同为 $I_1$,流过电阻 $R_2$ 和 $R_s$ 的电流同为 $I_2$,满足:

$$
U_{AB} = U_{AD} \quad \text{即} \quad I_1 R_1 = I_2 R_2
$$
$$
U_{BC} = U_{DC} \quad \text{即} \quad I_1 R_x = I_2 R_s
$$

两式相除,得到电桥的平衡条件:
$$
\frac{R_1}{R_2} = \frac{R_x}{R_s} \quad \text{即} \quad R_x = \frac{R_1}{R_2} \cdot R_s
$$

式中 $\frac{R_1}{R_2}$ 称为电桥的比率臂,$R_s$ 称为电桥的比较臂。通过调节 $R_s$ 使检流计 $G$ 无电流通过,即可求得 $R_x$ 值。

\textbf{2.交换法减小自组电桥系统误差}

为了尽量减小自组电桥的系统误差,可采用交换法。在电桥平衡后,将 $R_x$ 和 $R_s$ 位置互换,重新调节 $R_s$ 达到平衡,得到 $R_s'$。此时有:
$$
R_s' = \frac{R_1}{R_2} \cdot R_x
$$

因此我们可得:
$$
R_x = \sqrt{R_s R_s'}
$$

这样可以消除比率臂 $R_1, R_2$ 自身误差对测量结果的影响。

\textbf{3.电桥灵敏度与不确定度}

电桥灵敏度 $S$ 定义为检流计偏转格数 $\Delta d$ 与比较臂电阻相对改变量 $\Delta R_s / R_s$ 之比:
$$
S = \frac{\Delta d}{\Delta R_s / R_s}
$$

灵敏度越高,电桥对电阻变化的响应越明显,测量结果越准确。待测电阻 $R_x$ 的相对不确定度 $E$ 可表示为:
$$
E = \frac{\Delta R_x}{R_x} = \sqrt{ \left( \frac{\Delta R_s}{R_s} \right)^2 + \left( \frac{\Delta S}{S} \right)^2 }
$$

其中 $\displaystyle \frac{\Delta R_s}{R_s} = \pm \left( a + b \frac{m}{R_s} \right)\%$ 是电阻箱的不确定度,而电桥灵敏度引入的不确定度
$$
\frac{\Delta S}{S} = \frac{0.2\, R_s}{S}
$$

最终测量结果表示为
$$
R_x = \bar{R}_x \pm \Delta R_x
$$

其中 $\Delta R_x = E \cdot \bar{R}_x$。

\vspace{1\baselineskip}

\textbf{实验装置与方法:}  

实验器材包括检流计、电阻箱(四旋钮和六旋钮)、待测电阻、直流稳压电源和盒式惠斯登电桥(QJ-23 型)。

\textbf{自组电桥测量未知电阻:}  

 1.利用检流计、电阻箱、待测电阻及电源等组装电桥,其中 $R_1, R_2$ 选用四旋钮电阻箱,$R_s$ 选用六旋钮电阻箱。

 2.选取适当的比率臂,使测量结果的有效数字最大化。  

 3.按下检流计“电计”按钮,测量待测电阻 $R_x$,并测出该状态下电桥的灵敏度,用交换法进行系统误差分析,估算出测量误差 $\Delta R_x$,写出测量结果表达式。

\textbf{使用 QJ-23 型盒式惠斯登电桥测量未知电阻:}  

 1.打开盒式惠斯登电桥开关并调零。将 B 接上 4.5\,V 直流稳压电源,“G”和“外接”短接,然后将待测电阻接入 $R_x$ 接线端。  

 2.根据待测电阻盘上 8 个待测电阻 $R_{x1}, R_{x2}, \dots, R_{x8}$ 的数值,选取适当的倍率臂,确保测量结果有四位有效数字。  

 3.先按 B 键,后按 G 键以接通电路,调节 $R_s$ 的 4 个旋钮使电桥达到平衡,此时 $R_s$ 的 4 个旋钮所示数值乘以比率盘读数即为待测电阻阻值。  

 4.测量 8 个待测电阻,写出结果表达式,并确定这批电阻的离散程度。

\subsection{实验重点(3分)}
(简述本实验的学习重点,不超过100字。)

1.掌握惠斯登电桥的工作原理、平衡条件及灵敏度。

2.学会自组电桥并运用交换法测量未知电阻,并对结果进行误差分析。

3.掌握QJ-23型盒式惠斯登电桥的使用方法,并评估测量结果的离散程度。

\subsection{实验难点(2分)}
(简述本实验的实现难点,不超过100字。)

本实验的难点在于精确判断电桥的平衡点,尤其是在自组电桥中,需细致观察检流计指针的微小偏转。同时,准确选择比率臂以确保测量精度和有效数字,以及对系统误差和随机误差进行合理的分析与处理,也是实验成功的关键挑战。

\vspace{3\baselineskip}

\input{注意事项}
\end{document}
