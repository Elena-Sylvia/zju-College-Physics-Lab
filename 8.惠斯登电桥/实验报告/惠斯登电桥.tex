\documentclass{Phyport}
\usepackage{longtable}
\usepackage{multirow}      % 用于跨行
\usepackage{float} % 在导言区添加
\usepackage{amsmath}  % 必须加这一行!
\usepackage{ctex}     % 如果要显示中文,加上这一行

\usepackage{hyperref}
\hypersetup{
    colorlinks=true,
}

\exname{惠斯登电桥} %实验名称
\extable{11} %实验桌号
\instructor{郑昕颖老师} %指导教师
\class{-} %班级
\name{-} %姓名
\stuid{-} %学号

\nyear{2025} %年
\nmonth{11} %月
\nday{17} %日
\nweekday{一} %星期几

\redate{} %如有实验补做,补做日期
\resitu{} %情况说明:

\begin{document}
\setcounter{page}{0}
\makecover

\section{预习报告(10分)}
(注:将已经写好的“物理实验预习报告”内容拷贝过来)

\subsection{实验综述(5分)}
(自述实验现象、实验原理和实验方法,包括必要的光路图、电路图、公式等。不超过500字。)

\textbf{惠斯登电桥的实验}旨在通过自组惠斯登电桥和使用盒式惠斯登电桥,测量未知电阻并进行误差分析,从而掌握惠斯登电桥的工作原理和特点。惠斯登电桥是一种经典的直流平衡单臂电桥,广泛应用于中等数值电阻($10^1-10^6\Omega$)的精确测量。它通过比较法在平衡条件下确定待测电阻的阻值,是众多电学实验的基础。

\textbf{实验现象:}

在自组电桥的实验中,当电桥接近平衡时,检流计的指针偏转会逐渐减小,直至指示为零。通过调节比较臂电阻$R_s$和比率臂$R_1/R_2$,可以观察到电桥的灵敏度对平衡判断的影响。在交换法测量中,互换待测电阻和比较臂电阻的位置后,需要重新调节以达到平衡,这消除了R1、R2自身的误差对测量结果的影响。使用盒式惠斯登电桥时,通过旋钮的调节,可以快速实现电桥平衡,并且可以观察到在不同的倍率设置下,测量的范围和精度会有所不同。

\vspace{1\baselineskip}

\textbf{实验原理:}

\textbf{1. 惠斯登电桥测量电阻的原理}

下图是惠斯登电桥的原理图。

\begin{figure}[H]
    \centering
    \includegraphics[width=0.6\textwidth]{../images/实验原理图.png}
    \caption{惠斯登电桥原理图}
    \label{fig:惠斯登电桥原理图}
\end{figure}

当通过检流计 $G$ 的电流 $I_g$ 等于零时,B、D 两点电位相同,电桥达到平衡。此时,流过电阻 $R_1$ 和 $R_x$ 的电流同为 $I_1$,流过电阻 $R_2$ 和 $R_s$ 的电流同为 $I_2$,满足:

$$
U_{AB} = U_{AD} \quad \text{即} \quad I_1 R_1 = I_2 R_2
$$
$$
U_{BC} = U_{DC} \quad \text{即} \quad I_1 R_x = I_2 R_s
$$

两式相除,得到电桥的平衡条件:
$$
\frac{R_1}{R_2} = \frac{R_x}{R_s} \quad \text{即} \quad R_x = \frac{R_1}{R_2} \cdot R_s
$$

式中 $\frac{R_1}{R_2}$ 称为电桥的比率臂,$R_s$ 称为电桥的比较臂。通过调节 $R_s$ 使检流计 $G$ 无电流通过,即可求得 $R_x$ 值。

\textbf{2.交换法减小自组电桥系统误差}

为了尽量减小自组电桥的系统误差,可采用交换法。在电桥平衡后,将 $R_x$ 和 $R_s$ 位置互换,重新调节 $R_s$ 达到平衡,得到 $R_s'$。此时有:
$$
R_s' = \frac{R_1}{R_2} \cdot R_x
$$

因此我们可得:
$$
R_x = \sqrt{R_s R_s'}
$$

这样可以消除比率臂 $R_1, R_2$ 自身误差对测量结果的影响。

\textbf{3.电桥灵敏度与不确定度}

电桥灵敏度 $S$ 定义为检流计偏转格数 $\Delta d$ 与比较臂电阻相对改变量 $\Delta R_s / R_s$ 之比:
$$
S = \frac{\Delta d}{\Delta R_s / R_s}
$$

灵敏度越高,电桥对电阻变化的响应越明显,测量结果越准确。待测电阻 $R_x$ 的相对不确定度 $E$ 可表示为:
$$
E = \frac{\Delta R_x}{R_x} = \sqrt{ \left( \frac{\Delta R_s}{R_s} \right)^2 + \left( \frac{\Delta S}{S} \right)^2 }
$$

其中 $\displaystyle \frac{\Delta R_s}{R_s} = \pm \left( a + b \frac{m}{R_s} \right)\%$ 是电阻箱的不确定度,而电桥灵敏度引入的不确定度
$$
\frac{\Delta S}{S} = \frac{0.2\, R_s}{S}
$$

最终测量结果表示为
$$
R_x = \bar{R}_x \pm \Delta R_x
$$

其中 $\Delta R_x = E \cdot \bar{R}_x$。

\vspace{1\baselineskip}

\textbf{实验装置与方法:}  

实验器材包括检流计、电阻箱(四旋钮和六旋钮)、待测电阻、直流稳压电源和盒式惠斯登电桥(QJ-23 型)。

\textbf{自组电桥测量未知电阻:}  

 1.利用检流计、电阻箱、待测电阻及电源等组装电桥,其中 $R_1, R_2$ 选用四旋钮电阻箱,$R_s$ 选用六旋钮电阻箱。

 2.选取适当的比率臂,使测量结果的有效数字最大化。  

 3.按下检流计“电计”按钮,测量待测电阻 $R_x$,并测出该状态下电桥的灵敏度,用交换法进行系统误差分析,估算出测量误差 $\Delta R_x$,写出测量结果表达式。

\textbf{使用 QJ-23 型盒式惠斯登电桥测量未知电阻:}  

 1.打开盒式惠斯登电桥开关并调零。将 B 接上 4.5\,V 直流稳压电源,“G”和“外接”短接,然后将待测电阻接入 $R_x$ 接线端。  

 2.根据待测电阻盘上 8 个待测电阻 $R_{x1}, R_{x2}, \dots, R_{x8}$ 的数值,选取适当的倍率臂,确保测量结果有四位有效数字。  

 3.先按 B 键,后按 G 键以接通电路,调节 $R_s$ 的 4 个旋钮使电桥达到平衡,此时 $R_s$ 的 4 个旋钮所示数值乘以比率盘读数即为待测电阻阻值。  

 4.测量 8 个待测电阻,写出结果表达式,并确定这批电阻的离散程度。

\subsection{实验重点(3分)}
(简述本实验的学习重点,不超过100字。)

1.掌握惠斯登电桥的工作原理、平衡条件及灵敏度。

2.学会自组电桥并运用交换法测量未知电阻,并对结果进行误差分析。

3.掌握QJ-23型盒式惠斯登电桥的使用方法,并评估测量结果的离散程度。

\subsection{实验难点(2分)}
(简述本实验的实现难点,不超过100字。)

本实验的难点在于精确判断电桥的平衡点,尤其是在自组电桥中,需细致观察检流计指针的微小偏转。同时,准确选择比率臂以确保测量精度和有效数字,以及对系统误差和随机误差进行合理的分析与处理,也是实验成功的关键挑战。

\section{原始数据(20分)}
(将有老师签名的“自备数据记录草稿纸”的扫描或手机拍摄图粘贴在下方,完整保留姓名,学号,教师签字和日期。)

\begin{figure}[H]
    \centering
    \includegraphics[width=0.75\textwidth]{../images/original_data.jpg}
    \caption{original data}
    \label{fig:original_data}
\end{figure}

\section{结果与分析(60分)}
\subsection{数据处理与结果(30分)}
(列出数据表格、选择适合的数据处理方法、写出测量或计算结果。)

\textbf{实验一:自组电桥测未知电阻}

\begin{longtable}{|c|c|c|c|c|}
  \caption{自组电桥测未知电阻} \label{tab:自组电桥测未知电阻} \\
  \hline
     & \textbf{$R_1$} & \textbf{$R_2$} & \textbf{$R_s$} & \textbf{$R_s'$} \\
  \hline
  \textbf{交换前} & 200 & 200 & 219.8 & - \\
  \hline
  \textbf{交换后} & 200 & 200 & - & 220.0  \\
  \hline
\end{longtable}

利用交换法消除桥臂误差,待测电阻 $R_x$ 为:
\begin{equation}
    R_x = \sqrt{R_s \cdot R_s'} = \sqrt{219.8 \times 220.0} \approx 219.9 \, \Omega
\end{equation}

当将$R_s$调节$\Delta R_s = 0.1 \Omega$,此时检流计指针偏转$\Delta d=6.4$格,因此电桥灵敏度 $S$ 为:
\begin{equation}
    S = \frac{\Delta d}{\Delta R_s / R_s} = \frac{6.4}{0.1 / 219.9} = 14073.6 \approx 1.41 \times 10^4 \, \text{格}
\end{equation}

$R_x$的相对不确定度 $E$ 由仪器误差项和灵敏度误差项组成(取 $m=6$ 为电阻箱转盘数):

\begin{equation}
    E = \sqrt{\left( 0.001 + \frac{0.002m}{R_s} \right)^2 + \left( \frac{0.2}{S} \right)^2} = \sqrt{\left( 0.001 + \frac{0.002 \times 6}{219.9} \right)^2 + \left( \frac{0.2}{14074} \right)^2} \approx 0.11\%
\end{equation}

待测电阻的绝对不确定度 $\Delta R_x$ 为:
\begin{equation}
    \Delta R_x = R_x \times E = 219.9 \times 0.001055 \approx 0.23 \, \Omega
\end{equation}

因此修正后的测量结果为:
\begin{equation}
    R_x = (219.9 \pm 0.2) \, \Omega
\end{equation}

\textbf{实验二:用QJ-23型盒式惠斯登电桥测电阻离散度}

\begin{longtable}{|c|c|c|c|c|c|c|c|c|}
  \caption{盒式惠斯登电桥测量结果} \label{tab:盒式惠斯登电桥测量结果} \\
  \hline
  \textbf{待测电阻$R_{ni}$} & $R_{n1}$ & $R_{n2}$ & $R_{n3}$ & $R_{n4}$ & $R_{n5}$ & $R_{n6}$ & $R_{n7}$ & $R_{n8}$ \\
  \hline
  \textbf{测得阻值$R_{ni}/\Omega$} & 686.6 & 688.2 & 680.1 & 676.6 & 681.1 & 682.5 & 676.6 & 680.2 \\
  \hline
\end{longtable}

\begin{equation}
    \text{测得电阻的平均值}\overline{R_n} = \frac{1}{8} \sum_{i=1}^{8} R_{ni} = \frac{5451.9}{8} \approx 681.5 \, \Omega
\end{equation}

标准偏差 $S$ 为:
\begin{equation}
    S = \sqrt{\frac{\sum_{i=1}^{n} (R_{ni} - \overline{R_n})^2}{n-1}} = \sqrt{\frac{123.71}{7}} = \approx 4.2 \, \Omega
\end{equation}

因此该批电阻的离散度 $D$ 为:
\begin{equation}
    D = \frac{S}{\overline{R_n}} \times 100\% = \frac{4.2}{681.5} \times 100\% \approx 0.62\%
\end{equation}

\subsection{误差分析(20分)}
(运用测量误差、相对误差或不确定度等分析实验结果,写出完整的结果表达式,并分析误差原因。)

\textbf{自组惠斯登电桥测电阻($R_x$)的误差分析:
}

1.\textbf{仪器误差} :本次实验中,电阻箱$R_s$的精度等级直接决定了测量的不确定度。电阻箱内部电阻丝的老化、接触点的氧化都可能导致示数与实际阻值存在一定的偏差,同时电阻箱的精度不够也导致了无法完全找到指针不偏转的点,只能根据偏转幅度进行一定的估测,这也会对结果产生一定的影响。

2.\textbf{电桥灵敏度引入的误差}:当电桥趋于平衡时,微小的电流不足以克服检流计线圈的摩擦力矩,导致指针虽然指零,但实际上电桥并未绝对平衡。

3.\textbf{接触电阻}:导线接头、电阻箱旋钮接触不良会引入附加电阻,影响测量结果的准确性。

4.检流计较难调零,且有时按下“电计”按钮再松开后后指针会有轻微的偏转,从而给测量结果带来一定的误差。

\textbf{QJ-23 型盒式电桥测电阻离散度}

1.\textbf{仪器本身的精度等级}:QJ-23 是便携式电桥,其内部电阻(比率臂和比较臂)的精度等级通常低于实验室用的高精度电阻箱,这会给每一个电阻盘的测量值带来系统误差。

2.\textbf{接线电阻的影响}:盒式电桥采用两端接线法,待测电阻接入“$R_x$”端钮时的接触电阻以及连接导线的电阻都会直接计入测量结果中,从而影响测量的准确性。

3.\textbf{读数与估读误差}: 在判断检流计指针是否回零时,存在视差或人为判断的偏差。

\subsection{实验探讨(10分)}
(对实验内容、现象和过程的小结,不超过100字。)

本次实验分别采用自组惠斯登电桥和 QJ-23 型盒式电桥完成了电阻测量任务。在自组电桥实验中,通过引入“交换法”,有效消除了比率臂不对称引入的系统误差,测得待测电阻 $R_x \approx 219.9\,\Omega$。实验结果表明电桥灵敏度较高,验证了该方法的合理性。在随后进行的电阻离散度测量中,我利用盒式电桥测得该批电阻的离散度约为 $0.62\%$。

通过本次实验,我直观地观察了检流计偏转与电阻微小变化之间的关系,熟练掌握了电桥平衡的调节技巧,深刻理解了“零示法”测量的基本原理,达到了预期的实验目的。

\section{思考题(10分)}
(解答教材或讲义或老师布置的思考题,请先写题干,再作答。)

\textbf{思考题一:为什么用惠斯登电桥测电阻比伏安法测量的准确度高?用电桥法测电阻产生误差的主要因素是什么?}

伏安法测量时,电压表的分流或电流表的分压会引入不可避免的系统误差。而惠斯登电桥采用的是“零示法”,当电桥平衡时,检流计中无电流通过,因此不存在电表内阻带来的接入误差,检流计的测量准确度主要取决于标准电阻的精度和检流计的灵敏度($R_1,R_2$的误差可以通过交换法消除)。

\textbf{主要误差因素:}

1.测量$R_s$的电阻箱引入的误差,同时电阻箱精度不够导致无法完全平衡;

2.检流计灵敏度不够导致的“死区”误差,即无法准确判断绝对平衡点;

3.电路中的接触电阻和导线电阻;

4.环境因素如温度变化对电阻值的影响。

\textbf{思考题二:为了提高电桥测量灵敏度,应采取哪些措施?为什么?}

为了提高电桥测量灵敏度,可以采取以下措施:

1.选择高灵敏度的检流计:灵敏度越高,能够检测到的电流变化越小,从而更容易判断电桥的平衡状态。

2.适当提高电源电压E(在电阻允许功率范围内): 增加电源电压可以增大电桥两臂的电压差,使得微小的电阻变化能引起更明显的检流计偏转。

3.尽可能选用低内阻的电源和检流计,并调整比率臂至合适的值:这样可以减少电路中的能量损耗,提高测量的灵敏度。

\textbf{思考题三:用电桥测电阻时,若线路接通后检流计指针总是往一个方向偏转或总不偏转,试分析是什么原因?}

总是往一个方向偏转,说明电桥无法达到平衡。可能的原因有:
  
  1.比率臂(倍率)选择不当,导致待测电阻超出$R_s$的调节范围;

  2.桥臂电路存在断路(如待测电阻没接好)或短路;

  3.待测电阻$R_x$的阻值过大或过小,超出了电桥测量范围。

总不偏转:说明检流计回路不通或无电源。可能的原因有:

  1.检流计或其连接导线内部断路;

  2.电源未接通或电压过低;

  3.桥臂电路断路,导致无电流通过检流计。

\textbf{思考题四:惠斯登电桥比率臂选取的原则是什么?为什么要这样选取?}

\textbf{选取原则:} 应选择合适的比率 $K=R_1/R_2$,使得比较臂电阻箱 $R_s$的读数尽可能大,一般要求 $R_S$的四个刻度盘(千、百、十、个位)都有示数(即保持 4 位有效数字)。

\textbf{原因}:为了保证测量结果的有效数字位数。如果$R_s$读数过小(例如只有几十欧姆),电阻箱自身的相对误差会变大,且读数精度降低,从而严重影响$R_x$测量的准确度。

\textbf{思考题五: 如何使用自组电桥测量电表内阻(注意电表所能允许通过的最大电流)?根据电桥平衡的特点,可否将桥路中的检流计去掉,换成待测电表判别电桥的平衡?}

\textbf{测量方法:}

1.将待测电表(如微安表)接入电桥的$R_x$臂位置

2.注意为防止烧坏电表,应先根据欧姆定律估算电路电流,必要时串联保护电阻或降低电源电压。

可以去掉检流计。将待测电表接入 $R_x$臂后,原检流计位置(B、D 两点间)直接用导线串联一个开关S连接。调节 $R_s$,观察开关S闭合前后,待测电表的示数是否发生变化。若闭合S后待测电表示数不变,说明 B、D 两点电位相同(无电流流过支路),此时电桥达到平衡,即可计算内阻。

\vspace{3\baselineskip}

\input{注意事项}
\end{document}
