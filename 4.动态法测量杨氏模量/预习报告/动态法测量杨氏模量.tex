\documentclass{Preport}

\usepackage{hyperref}
\hypersetup{
    colorlinks=true,
}

\exname{动态法测量杨氏模量} %实验名称
\instructor{张贯乔老师} %指导教师
\class{-} %班级
\name{-} %姓名
\stuid{-} %学号

\nyear{2025} %年
\nmonth{10} %月
\nday{20} %日
\nweekday{一} %星期几

\begin{document}
\setcounter{page}{0}
\makecover

\section{预习报告(10分)}
\subsection{实验综述(5分)}
(自述实验现象、实验原理和实验方法,包括必要的光路图、电路图、公式等。不超过500字。)

\textbf{动态法测量杨氏模量实验}旨在通过测定金属棒的弯曲共振频率来计算材料的弹性模量,从而理解固体在动态激振下的振动规律。该实验利用共振原理,通过测得金属棒的基频并结合其几何与物理参数,间接求得杨氏模量。相比静态拉伸法,动态法具有\textbf{非接触测量、精度高、实验周期短和抗干扰能力强}等优点。

\textbf{实验现象:}当信号发生器输出的驱动频率接近金属棒的固有频率时,棒体振动幅度明显增大,示波器上显示的电信号峰峰值达到最大,出现明显的共振峰。随着频率偏离共振点,振幅迅速减小。通过改变悬挂点或扫描不同频率,我们可以观察到共振峰的变化过程。

\textbf{实验原理:}

信号源输出等幅正弦波信号经激振器转为机械振动加在被测器件上,使其受迫做横振动,拾振器将其振动转为电信号,经放大后通过示波器显示出来。

金属棒的横向自由振动满足\textbf{欧拉–伯努利梁方程:}
$$
E I \frac{d^4 y}{d x^4} + \rho A \frac{d^2 y}{d t^2} = 0,
$$

其中,$E$ 为杨氏模量,$I=\frac{\pi d^4}{64}$ 为截面惯性矩,$\rho$ 为材料密度,$A=\frac{\pi d^2}{4}$ 为横截面积。

对两端自由边界条件求解可得基频 $f_1$与杨氏模量的关系式:

$$
E = 1.6067 \times 10^{12} \frac{L^3 f_1^2m}{d^4},
$$
其中 $L$ 为棒长,$d$ 为直径,$f_1$ 为基频共振基频,m为被测件质量。

\textbf{实验装置:}实验系统由信号发生器、功率放大器、激振换能器、拾振换能器、示波器以及金属棒悬挂装置组成。铜棒悬挂在 0.224L 和 0.776L 处的两波节点间,通过激振与拾振换能器形成闭环,利用示波器观测信号强度随频率的变化。

\textbf{实验步骤:}

\begin{enumerate}[leftmargin=4em]
    \item 测量金属棒的长度、直径与质量;
    \item 调节悬挂点,使棒对称悬挂并连接换能器;
    \item 扫描信号发生器频率,记录共振峰对应的频率;
    \item 采用多项式拟合确定基频 $f_1$,计算杨氏模量并分析不确定度。
\end{enumerate}

\subsection{实验重点(3分)}
(简述本实验的学习重点,不超过100字。)

理解杨氏模量的物理意义及其与材料弹性性质的关系;掌握利用动态法(弯曲共振法)测定杨氏模量的基本原理与实验方法;熟悉共振现象的识别以及基频的测定过程;学会通过多项式拟合确定共振基频,并利用测得的频率、长度和直径计算杨氏模量;能够依据误差传递公式分析实验结果的不确定度,提高自己的实验数据处理与结果分析能力。

\subsection{实验难点(2分)}
(简述本实验的实现难点,不超过100字。)

如何准确获取共振基频是本实验的主要难点。由于共振峰较窄且信号易受噪声干扰,需精确调节激振频率并判断峰值位置;同时,悬挂点位置及换能器接触方式对振动模式影响显著,若不对称或接触不良,会导致共振峰偏移或测量误差增大。



\input{注意事项}
\end{document}
