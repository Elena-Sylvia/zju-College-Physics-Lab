\documentclass{Phyport}
\usepackage{longtable}
\usepackage{multirow}      % 用于跨行
\usepackage{float} % 在导言区添加
\usepackage{amsmath}  % 必须加这一行!
\usepackage{ctex}     % 如果要显示中文,加上这一行

\usepackage{hyperref}
\hypersetup{
    colorlinks=true,
}

\exname{动态法测量杨氏模量} %实验名称
\extable{9} %实验桌号
\instructor{张贯乔老师} %指导教师
\class{-} %班级
\name{-} %姓名
\stuid{-} %学号

\nyear{2025} %年
\nmonth{10} %月
\nday{20} %日
\nweekday{一} %星期几

\redate{} %如有实验补做,补做日期
\resitu{} %情况说明:

\begin{document}
\setcounter{page}{0}
\makecover

\section{预习报告(10分)}
(注:将已经写好的“物理实验预习报告”内容拷贝过来)

\subsection{实验综述(5分)}
(自述实验现象、实验原理和实验方法,包括必要的光路图、电路图、公式等。不超过500字。)

\textbf{动态法测量杨氏模量实验}旨在通过测定金属棒的弯曲共振频率来计算材料的弹性模量,从而理解固体在动态激振下的振动规律。该实验利用共振原理,通过测得金属棒的基频并结合其几何与物理参数,间接求得杨氏模量。相比静态拉伸法,动态法具有\textbf{非接触测量、精度高、实验周期短和抗干扰能力强}等优点。

\textbf{实验现象:}当信号发生器输出的驱动频率接近金属棒的固有频率时,棒体振动幅度明显增大,示波器上显示的电信号峰峰值达到最大,出现明显的共振峰。随着频率偏离共振点,振幅迅速减小。通过改变悬挂点或扫描不同频率,我们可以观察到共振峰的变化过程。

\textbf{实验原理:}

信号源输出等幅正弦波信号经激振器转为机械振动加在被测器件上,使其受迫做横振动,拾振器将其振动转为电信号,经放大后通过示波器显示出来。

金属棒的横向自由振动满足\textbf{欧拉–伯努利梁方程:}
$$
E I \frac{d^4 y}{d x^4} + \rho A \frac{d^2 y}{d t^2} = 0,
$$

其中,$E$ 为杨氏模量,$I=\frac{\pi d^4}{64}$ 为截面惯性矩,$\rho$ 为材料密度,$A=\frac{\pi d^2}{4}$ 为横截面积。

对两端自由边界条件求解可得基频 $f_1$与杨氏模量的关系式:

$$
E = 1.6067  \frac{L^3 f_1^2m}{d^4},
$$
其中 $L$ 为棒长,$d$ 为直径,$f_1$ 为基频共振基频,m为被测件质量。

\textbf{实验装置:}实验系统由信号发生器、功率放大器、激振换能器、拾振换能器、示波器以及金属棒悬挂装置组成。铜棒悬挂在 0.224L 和 0.776L 处的两波节点间,通过激振与拾振换能器形成闭环,利用示波器观测信号强度随频率的变化。

\textbf{实验步骤:}

\begin{enumerate}[leftmargin=4em]
    \item 测量金属棒的长度、直径与质量;
    \item 调节悬挂点,使棒对称悬挂并连接换能器;
    \item 扫描信号发生器频率,记录共振峰对应的频率;
    \item 采用多项式拟合确定基频 $f_1$,计算杨氏模量并分析不确定度。
\end{enumerate}

\subsection{实验重点(3分)}
(简述本实验的学习重点,不超过100字。)

理解杨氏模量的物理意义及其与材料弹性性质的关系;掌握利用动态法(弯曲共振法)测定杨氏模量的基本原理与实验方法;熟悉共振现象的识别以及基频的测定过程;学会通过多项式拟合确定共振基频,并利用测得的频率、长度和直径计算杨氏模量;能够依据误差传递公式分析实验结果的不确定度,提高自己的实验数据处理与结果分析能力。

\subsection{实验难点(2分)}
(简述本实验的实现难点,不超过100字。)

如何准确获取共振基频是本实验的主要难点。由于共振峰较窄且信号易受噪声干扰,需精确调节激振频率并判断峰值位置;同时,悬挂点位置及换能器接触方式对振动模式影响显著,若不对称或接触不良,会导致共振峰偏移或测量误差增大。

\section{原始数据(20分)}
(将有老师签名的“自备数据记录草稿纸”的扫描或手机拍摄图粘贴在下方,完整保留姓名,学号,教师签字和日期。)

\begin{figure}[H]
    \centering
    \includegraphics[width=0.85\textwidth]{../images/original_data.jpg}
    \caption{original data}
    \label{fig:original_data}
\end{figure}

\section{结果与分析(60分)}
\subsection{数据处理与结果(30分)}
(列出数据表格、选择适合的数据处理方法、写出测量或计算结果。)

\textbf{步骤一:金属棒参数测量}

1.使用刻度尺测量金属棒长度L

2.使用螺旋测微计测量金属棒直径

3.使用分析天平测量金属棒质量m

\textbf{其中,测量金属棒直径d时,使用的螺旋测微计存在-0.028mm的误差,因此下表中记录的金属棒直径为螺旋测微计的读数加上原始误差}

\textbf{金属棒参数测量记录表:}

\begin{longtable}{|c|c|c|c|c|c|c|c|}
  \caption{金属棒参数测量数据} \label{tab:金属棒参数测量数据} \\
  \hline
    \textbf{实验次数} & \textbf{1} & \textbf{2} & \textbf{3} & \textbf{4} & \textbf{5} & \textbf{6} & \textbf{平均值} \\
  \hline
  \textbf{金属棒长度L(mm)} & 159.9 & 159.8 & 160.0 & 159.8 & 159.9 & 159.9 & 159.88 \\
  \hline
  \textbf{金属棒直径d(mm)} & 5.951 & 5.949 & 5.945 & 5.952 & 5.951 & 5.950 & 5.950 \\
  \hline
  \textbf{金属棒质量m(g)} & \multicolumn{7}{|c|}{37.652} \\
  \hline
\end{longtable}

又测量质量的分析天平$\Delta_\text{仪}m=\pm 1 mg$,故不确定度$U_m=U_B=\frac{\Delta_\text{仪}m}{\sqrt{3}}=0,577mg$,$m=37.652\pm 0.577mg$。

而对于金属棒的长度:$\Delta_\text{仪}L=\pm 0.2mm$
$$
L = \frac{\sum_{i=1}^6 L_i}{6} = 159.88mm
$$
$$
u_A = \sqrt{\frac{\sum_{i=1}^6 (L_i - L)^2}{6 \times 5}} = 0.04mm
$$
$$
u_B = \frac{\Delta_L}{\sqrt{3}} = \frac{0.20}{\sqrt{3}} = 0.12mm  
$$
$$
u_L = \sqrt{u_A^2 + u_B^2} = 0.13mm 
$$

因此$L=159.88 \pm 0.13mm$

对于金属棒的直径:$\Delta_\text{仪}d=\pm 0.004mm$
$$
d = \frac{\sum_{i=1}^6 d_i}{6} = 5.950mm
$$
$$
u_A = \sqrt{\frac{\sum_{i=1}^6 (d_i - d)^2}{6 \times 5}} = 0.0010mm 
$$
$$
u_B = \frac{\Delta_d}{\sqrt{3}} = \frac{0.004}{\sqrt{3}} = 0.0023mm
$$
$$
u_d = \sqrt{u_A^2 + u_B^2} = 0.0025mm
$$

因此$d=5.950 \pm 0.0025mm$

\textbf{步骤二:共振频率测量}

\textbf{因为悬挂点距端点距离为35mm时,波形图无法看出峰值在什么地方,因此直接略过这个数据。}

\begin{longtable}{|c|c|c|c|c|c|c|c|c|c|c|}
  \caption{共振频率测量表} \label{tab:共振频率测量} \\
  \hline
    \textbf{悬挂点距离端点距离x(mm)} & 5 & 10 & 15 & 20 & 25 & 30 & 35 & 40 & 45 & 50 \\
  \hline
  \textbf{共振频率f/Hz} & 751.2 & 749.6 & 747.9 & 746.6 & 745.4 & 743.6 & - & 743.5 & 745.4 & 745.9 \\
  \hline
\end{longtable}

且对共振频率的测量$\Delta_\text{仪}f=\pm 0.1 Hz$,故$U_f=U_B = \frac{\Delta_\text{仪}f}{\sqrt{3}}=0.06HZ$

使用matplotlib对实验数据进行二次函数拟合,并获得最低点对应的频率值,获得如下结果:

\begin{figure}[H]
    \centering
    \includegraphics[width=0.85\textwidth]{../images/output.png}
    \caption{二次拟合结果图}
    \label{fig:二次拟合}
\end{figure}

由拟合曲线求得极小值点$x_{min}=35.674mm$,对应的共振基频为$f_{min}=744.131Hz$。

根据残差计算得频率标准差
$$
\sigma_f = \sqrt{\frac{1}{n-1}\sum_{i=1}^n [f_i - f(x_i)]^2} = 0.539Hz
$$

以三倍标准差估计本实验的频率测量不确定度$U_f = 3\sigma_f = 1.618Hz$

\textbf{杨氏模量的计算:}
$$
E = 1.6067\frac{m L^3 f_1^2}{d^4}
$$
$$
=1.6067\cdot\frac{(3.7652\times10^{-2})\cdot(0.15988)^3\cdot(744.131)^2}{(5.95\times10^{-3})^4}
$$
$$
\approx 1.09\times10^{11}\ \mathrm{Pa}
$$

不确定度传导式:
\[
\frac{\Delta E}{E} =
\sqrt{\left(\frac{3\Delta L}{L}\right)^2
+\left(\frac{4\Delta d}{d}\right)^2
+\left(\frac{\Delta m}{m}\right)^2
+\left(\frac{2\Delta f_1}{f_1}\right)^2 }=5.26\times 10^{-3}.
\]

因此$\Delta E=5.8\times 10^{8}Pa$,$E=(1.09\pm 0.06)\times 10^{11}Pa$

\subsection{误差分析(20分)}
(运用测量误差、相对误差或不确定度等分析实验结果,写出完整的结果表达式,并分析误差原因。)

不确定度的计算在实验报告前一部分已经给出,本部分我将直接分析误差的原因。

1.质量测量。金属棒质量的测量使用的是分析天平,并且只进行了一次的测量,因此如果分析天平的精度存在一定问题,就会对质量的测量结果产生一定的误差。

2.直径的测量。我们所使用的螺旋测微计在零点处存在一定的偏差,而我调零时旋转的程度不同会对零点初始修正值产生一定的影响,从而影响最终的测量结果。

3.共振频率的测量。

(1)由于两根悬线的长度存在一定的差异,铜棒很难真正调节至水平,这可能给实验结果引入一定的误差。

(2)在测量某一x对应的f时,发现不同的靠近方法下,最大峰值对应的f值会产生一定的波动,因此我所读取到的f可能并不是十分准确。

(3)悬挂铜棒时,无法特别精准地将悬线对准金属棒上的不同刻痕,这也可能会对实验结果产生一定的影响。

(4)如果示波器的分辨率有限,比如在本次实验中,当x=35时,峰值的变化太小,我们使用的示波器无法分辨,导致该距离下的数据缺失。

(5)实验环境存在较大噪声,测量f的时候峰值的变化较为剧烈,可能会对我们的读数与判断产生一定的影响。

\subsection{实验探讨(10分)}
(对实验内容、现象和过程的小结,不超过100字。)

本实验通过使用动态法中的弯曲共振法来测量金属棒的杨氏模量,利用激振装置使金属棒产生弯曲振动,并通过扫频记录不同悬挂点下的共振频率。随后根据二次拟合结果求得共振基频,结合金属棒的长度、直径和质量,计算出金属棒的杨氏模量约为$1.09\times 10^{11}$,与铜的理论值较为接近。实验过程中我们可以直观地观察到共振峰在特定频率处振幅显著增大。通过不确定度分析,我发现频率测量为为误差的主要来源。本次实验加深了我对杨氏模量物理意义及动态测量方法的理解。

\section{思考题(10分)}
(解答教材或讲义或老师布置的思考题,请先写题干,再作答。)

\textbf{思考题一:如何区分假的共振峰?}

课本[实验原理]一节中给出了真假共振峰的判别方法:

(1)共振频率预估法:做实验前先用理论公式估算出共振频率的大致范围,然后进行细致的测量。

(2)峰宽判别法:真正的共振峰峰宽十分尖锐,特别是在室温时,只要改变激振信号频率约为0.1Hz,即可判断出试样是否处于最佳共振状态;而虚假共振峰的峰宽较宽。

\textbf{思考题二:测量的实验条件是否满足杆振动模型的要求?}

本次实验中所使用的金属棒在尺寸与边界条件上基本满足杆振动模型的要求。

理论上,杆振动模型假设材料为均匀、各向同性的细长圆柱体,且振动为小幅弯曲振动,边界条件为自由–自由梁。

实验中金属棒长度L=159.88mm,直径d=5.95mm,其长径比约为27,远大于10,满足“细长梁”条件;同时棒的材质均匀、弹性良好,激振振幅较小,形变处于弹性范围内,因此线性振动模型可用。

但是由于悬挂点并非理想的自由支撑,换能器与夹具的质量、接触刚度以及空气阻尼等因素会使实际边界条件与理论模型略有偏差,这可能导致共振频率略高或略低于理想值,从而带来系统误差。

\textbf{思考题三:假如不满足,为什么能采用这种方法测出横杆基频?}

基频主要由系统的整体刚度与有效质量决定,这些非理想的因素只会使频率产生较小的系统性偏移,而不会破坏共振现象本身。实验中的杆仍然可以形成稳定的驻波,当激振频率接近其自然频率时,振幅会明显增强,从而可以准确识别出共振峰并进行二次拟合求得基频。

因此,只要偏差较小、系统响应清晰,动态法仍能通过拟合获得可靠的基频,从而间接计算出杨氏模量。

\textbf{思考题四:我们能否观察n = 2,3等情况的本征振动来测量杨氏模量?有什么困难?}

理论上,金属杆的高阶本征振动(n=2,3……)同样满足
$$
f_n = \frac{\beta_n^2}{2\pi L^2} \sqrt{\frac{E I}{\rho A}}
$$
因此若能准确测得$f_n$,也可以用相同公式反算出杨氏模量E。

但在实际实验中,直接测量高阶模态存在明显的困难:

1.高阶模态频率较高,我们使用的激振装置输出能量有限,难以有效激发;

2.高阶峰更尖且更易受阻尼影响,共振幅度减小,频率分辨率的要求更高;

3.模式间耦合与噪声叠加,使高阶共振峰易与其他共振或寄生振动混淆

\vspace{3\baselineskip}
\input{注意事项}
\end{document}
