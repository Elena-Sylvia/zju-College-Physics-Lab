\documentclass{Phyport}
\usepackage{longtable}
\usepackage{multirow}      % 用于跨行
\usepackage{array}
\usepackage{float} % 在导言区添加
\usepackage{amsmath}  % 必须加这一行!
\usepackage{ctex}     % 如果要显示中文,加上这一行

\usepackage{hyperref}
\hypersetup{
    colorlinks=true,
}

\exname{分光计的调整与使用} %实验名称
\extable{9} %实验桌号
\instructor{王宙洋老师} %指导教师
\class{-} %班级
\name{-} %姓名
\stuid{-} %学号

\nyear{2025} %年
\nmonth{12} %月
\nday{22} %日
\nweekday{一} %星期几

\redate{} %如有实验补做,补做日期
\resitu{} %情况说明:

\begin{document}
\setcounter{page}{0}
\makecover

\section{预习测试(10分)}
上课前到学在浙大上完成,注意测试仅1次机会。期末时测试分数会与报告其他部分的分数进行加和处理。

\section{原始数据(20分)}
(将有老师签名的“自备数据记录草稿纸”的扫描或手机拍摄图粘贴在下方,完整保留姓名,学号,教师签字和日期。)

\begin{figure}[H]
    \centering
    \includegraphics[width=0.70\textwidth]{../images/originaldata.jpg}
    \caption{original data}
    \label{fig:originaldata}
\end{figure}

\section{结果与分析(60分)}
\subsection{数据处理与结果(30分)}
(列出数据表格、选择适合的数据处理方法、写出测量或计算结果。)

\begin{longtable}{|c|c|c|c|c|c|c|c|}
\caption{实验数据记录表} \label{tab:experiment} \\
\hline
\multirow{2}{*}{\textbf{实验次数}} 
& \multicolumn{2}{c|}{\textbf{方位左}} 
& \multicolumn{2}{c|}{\textbf{方位右}} 
& \multirow{2}{*}{\textbf{$|\theta_{\text{左I}} - \theta_{\text{右I}}|$}} 
& \multirow{2}{*}{\textbf{$|\theta_{\text{左II}} - \theta_{\text{右II}}|$}} 
& \multirow{2}{*}{\textbf{$\Delta A$}} \\
\cline{2-5}
& 游标 I 窗 $\theta_{\text{左I}}$ 
& 游标 II 窗 $\theta_{\text{左II}}$ 
& 游标 I 窗 $\theta_{\text{右I}}$ 
& 游标 II 窗 $\theta_{\text{右II}}$ 
& \multicolumn{1}{c|}{} 
& \multicolumn{1}{c|}{} 
& \\ % 这一行用于补全竖线
\hline
\textbf{1} & $108 ^\circ 53' $ & $288 ^\circ 54' $ & $348 ^\circ 59' $ & $168 ^\circ 55' $ & $119 ^\circ 54' $ & $119 ^\circ 59' $ & $59 ^\circ 58' $ \\
\hline
\textbf{2} & $112 ^\circ 13' $ & $292 ^\circ 13' $ & $352 ^\circ 14' $ & $172 ^\circ 11' $ & $119 ^\circ 59' $ & $120 ^\circ 02' $ & $60^\circ 00' $ \\
\hline
\textbf{3} & $111 ^\circ 57' $ & $291 ^\circ 57' $ & $351 ^\circ 59' $ & $171 ^\circ 54' $ & $119 ^\circ 58' $ & $120 ^\circ 03' $ & $60 ^\circ 00' $ \\
\hline
\textbf{4} & $109 ^\circ 59' $ & $290 ^\circ 00' $ & $350 ^\circ 03' $ & $170 ^\circ 00' $ & $119 ^\circ 56' $ & $120 ^\circ 00' $ & $59 ^\circ 59' $ \\
\hline
\textbf{5} & $108 ^\circ 16' $ & $288 ^\circ 17' $ & $348 ^\circ 21' $ & $168 ^\circ 18' $ & $119 ^\circ 55' $ & $119 ^\circ 59' $ & $59 ^\circ 59' $ \\
\hline
\textbf{6} & $105 ^\circ 04' $ & $285 ^\circ 05' $ & $345 ^\circ 08' $ & $165 ^\circ 05' $ & $119 ^\circ 56' $ & $120 ^\circ 00' $ & $59 ^\circ 59' $ \\
\hline
\end{longtable}

由于反射光线偏转角约为$120^\circ $,游标I窗读数跨越$0^\circ $刻度线,故计算差值采用补角公式:
$$
|\theta_{\text{左I}} - \theta_{\text{右I}}| = (360^\circ -  \theta_{\text{右I}}) + \theta_{\text{左I}} 
$$

游标II窗直接作差:
$$
|\theta_{\text{左II}} - \theta_{\text{右II}}| = \theta_{\text{左II}} - \theta_{\text{右II}}
$$

三棱镜顶角计算公式为:
$$
\angle A_i= \frac{|\theta_{\text{左I}} - \theta_{\text{右I}}| + |\theta_{\text{左II}} - \theta_{\text{右II}}|}{4}
$$

因此,六次测量后获得的三棱镜顶角的平均值为:
$$
\overline{\angle A }=\frac{\sum_{i=1}^{6}\angle A_i}{6} = 59^\circ 59.17' \approx 59 ^\circ 59'
$$

\subsection{误差分析(20分)}
(运用测量误差、相对误差或不确定度等分析实验结果,写出完整的结果表达式,并分析误差原因。)

\textbf{A 类不确定度:}

根据贝塞尔公式计算标准偏差(n=6):
$$
u_A = \sqrt{\frac{\sum_{i=1}^{6}(\angle A_i-\overline{\angle A})^2}{n(n-1)}} \approx 0.32'
$$

\textbf{B 类不确定度:}

分光计游标盘的仪器允差为$\Delta_{ins}=1'$,取均匀分布($k=\sqrt{3}$):
$$
u_B = \frac{\Delta_{ins}}{\sqrt{3}} \approx 0.58'
$$

\textbf{合成不确定度:}
$$
u_c=\sqrt{(u_A)^2 + (u_B)^2} \approx 0.66'
$$

\textbf{最终结果表达式:}

结合不确定度的修约原则,最终结果表达式为:$\angle A=\overline{\angle A}\pm u_c = 59^\circ 59' \pm 1'$

\vspace{1\baselineskip}

\textbf{误差原因分析:}

1.仪器系统误差与偏心差(已修正)

分光计的刻度盘中心与仪器的主轴旋转中心往往不完全重合,这会产生偏心差。在本次实验中,我们采用了双游标读数法,并取两者差值的平均值参与计算。这种数据处理方法有效地消除了偏心差对测量结果的影响

2.仪器读数精度限制

分光计游标盘的最小分度值为$1'$,这是仪器的极限分辨率,因此无论我们的操作多么精确,我们在读数过程中都会不可避免的引入一定的读数误差,这也会最终的测量结果产生一定的影响。

3.视差与调节的误差

在观测过程中我发现,无论如何消除视察,我们在读数过程中,眼睛位置的微小移动都会导致叉丝与成像的相对位置发生一定的偏移,从而引入一定的实验误差。此外,我们很难真正调节到光轴完全垂直于主轴,如果三棱镜的反射面没有严格平行于仪器主轴,反射光线会产生倾斜,导致测量角度出现微小偏差。


\subsection{实验探讨(10分)}
(对实验内容、现象和过程的小结,不超过100字。)

本此实验中,我们学习了利用自准直法对分光计进行调整;通过“减半逼近法”反复调节,确保了望远镜光轴及载物台平面垂直于仪器主轴,从而能观察到清晰且无视差的反射像;实验测量中我们采用了双游标读数法,有效消除了偏心差的影响,最终测得三棱镜顶角为$59^\circ 59' \pm 1'$,极接近理论值。在实验过程中,精细的“三垂直”调节和准确的瞄准读数是获得高精度测量结果的关键。

\section{思考题(10分)}
(解答教材或讲义或老师布置的思考题,请先写题干,再作答。)

\textbf{1.为什么在调节分光计时要采用“减半逼近法”来调整望远镜光轴与仪器转轴垂直?}

因为在本次实验中,我们判断望远镜的光轴是否与仪器转轴垂直,是采用将载物台转动180度前后,观察亮十字是否都在上叉丝处。如果转动前后,亮十字都处于上叉丝的同一侧(同上或同下),说明望远镜光轴与转轴不垂直;如果转动前后,亮十字分别处于上叉丝的上下两侧,说明望远镜光轴与转轴垂直,误差是由于载物台平面与仪器主轴不完全垂直。由于人的视觉判断存在一定的主观误差,同时我们会在转动180度前后分别进行调节,因此我们采用“减半逼近法”来逐步调整望远镜光轴与转轴的垂直性,即转动前后分别调节螺钉,消除亮十字一半的偏移量。每次调整后,我们都将载物台转动180度进行观察,根据亮十字的位置来判断调整的方向和幅度。通过不断地减小调整的幅度,我们可以更精确地使望远镜光轴与仪器转轴垂直,从而提高实验的准确性。

\textbf{2.在用自准直法测量三棱镜顶角的时候,为什么要测量两个游标读数并求平均值?}

因为分光计的游标盘可能存在偏心差,即游标盘的旋转中心与仪器的主轴旋转中心不完全重合。这种偏心差会导致单一游标读数产生系统误差,影响测量结果的准确性。通过测量两个相对位置相差$180^\circ$的游标读数并求平均值,我们可以有效地抵消这种偏心差的影响。具体来说,左侧游标读数与右侧游标读数时,偏心差的影响方向正好相反。通过取二者的平均值,我们可以从数学上将这些相反的误差部分相互抵消,从而获得更接近真实值的测量结果。

\textbf{3.三棱镜应摆放在载物台的什么位置?简述其理由。}

三棱镜的顶角应接近载物台的中心位置。理由如下:

1.如果三棱镜的顶角太靠两侧,则平行光只会在一个面上反射,另一侧观测不到反射光;

2.如果三棱镜的顶角太靠前或者靠后,光线经过棱镜反射后,很可能会偏离主轴,导致反射光线无法进入望远镜的物镜孔径内,从而无法观测到反射像。

因此,三棱镜的顶角应该接近载物台平台中心偏上一点的位置,保证我们能从望远镜中观测到正确的结果。

\vspace{3\baselineskip}

\input{注意事项}
\end{document}
