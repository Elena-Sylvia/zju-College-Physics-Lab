\documentclass{Preport}
\usepackage{longtable}
\usepackage{multirow}      % 用于跨行
\usepackage{float} % 在导言区添加
\usepackage{amsmath}  % 必须加这一行!
\usepackage{ctex}     % 如果要显示中文,加上这一行

\usepackage{hyperref}
\hypersetup{
    colorlinks=true,
}

\exname{用双臂电桥测低电阻} %实验名称
\instructor{郑昕颖老师} %指导教师
\class{-} %班级
\name{-} %姓名
\stuid{-} %学号

\nyear{2025} %年
\nmonth{12} %月
\nday{1} %日
\nweekday{一} %星期几

\begin{document}
\setcounter{page}{0}
\makecover

\section{预习报告(10分)}
\subsection{实验综述(5分)}
(自述实验现象、实验原理和实验方法,包括必要的光路图、电路图、公式等。不超过500字。)

\textbf{直流双臂电桥实验}(又称开尔文电桥)旨在解决惠斯登单臂电桥无法精确测量低值电阻($10^{-5} \sim 1 \Omega$)的问题。在测量低电阻时,导线电阻和接触电阻(约 $10^{-4} \sim 10^{-2} \Omega$)与被测电阻处于同一数量级,直接串联会带来巨大的误差。本实验通过“四端接入法”和特殊的双桥臂结构,消除或减小了这些附加电阻对测量结果的影响。

\vspace{1\baselineskip}

\textbf{实验原理:}

\textbf{1. 四端接入法与双臂电桥结构}

为了消除接线电阻,待测电阻 $R_x$ 采用四端接法(如下左图所示):包含两个电流接头($C_1', C_2'$)和两个电位接头($P_1', P_2'$)。电流流经 $C'$ 端,电压由 $P'$ 端引出,使得接触电阻被“转移”到高阻抗回路或不影响平衡的回路中。

双臂电桥原理如下右图所示,电路中包含待测电阻 $R_x$、标准电阻 $R_s$、以及两对联动比率臂电阻 $R_1, R_2$ 和 $R_3, R_4$。$r$ 为连接 $R_x$ 和 $R_s$ 的粗导线电阻。

\begin{figure}[htbp]
    \centering
    \begin{minipage}[t]{0.45\linewidth}
        \centering
        \includegraphics[width=\linewidth]{../images/四端输入法.png} 
        \caption{四端输入法}
        \label{fig:四端输入法}
    \end{minipage}
    \hfill
    \begin{minipage}[t]{0.45\linewidth}
        \centering
        \includegraphics[width=\linewidth]{../images/直流双臂电桥.png} 
        \caption{直流双臂电桥原理图}
        \label{fig:直流双臂电桥原理图}
    \end{minipage}
    \label{fig:combined_2}
\end{figure}

\textbf{2. 平衡条件与公式}

当检流计 $G$ 中无电流通过($I_G = 0$)时,电桥达到平衡。根据基尔霍夫定律,可推导出被测电阻 $R_x$ 的表达式:
\begin{equation}
    R_x = \frac{R_1}{R_2} R_s + \frac{R_4 r}{R_3 + R_4 + r} \left( \frac{R_1}{R_2} - \frac{R_3}{R_4} \right)
    \label{eq:FullEquation}
\end{equation}

公式中第二项为附加电阻 $r$ 带来的误差项。为了消除此项,实验仪器(如 QJ-44)在设计上通过机械联动装置,确保调节过程中始终满足:
$$ \frac{R_1}{R_2} = \frac{R_3}{R_4} $$
此时,公式简化为单臂电桥的形式,从而消除了接触电阻 $r$ 的影响:
\begin{equation}
    R_x = \frac{R_1}{R_2} R_s
\end{equation}

\vspace{1\baselineskip}

\textbf{实验方法:}

\textbf{1. 测量金属导体的电阻率}

利用 QJ-44 双臂电桥测出金属棒的阻值 $R$,用游标卡尺测量其直径 $d$ 和电位接头间的长度 $L$,计算电阻率:
$$ \rho = R \cdot \frac{S}{L} = R \cdot \frac{\pi d^2}{4L} $$

\textbf{2. 测量金属导体的电阻温度系数}

将金属导体浸没在变温装置(如机油浴)中。根据金属电阻随温度线性变化的规律(温度不太高时) $R_t = R_0 (1 + \alpha t)$,通过升温或降温法,每隔 $5^\circ\text{C}$ 记录一组电阻 $R_x$ 和温度 $t$。
利用以下公式避免在0摄氏度下测量 $R_0$ ,计算温度系数 $\alpha$:
\begin{equation}
    \alpha = \frac{R_{x2} - R_{x1}}{R_{x1}t_2 - R_{x2}t_1}
\end{equation}
或通过绘制 $R_t - t$ 特性曲线,由斜率求解 $\alpha$。

\subsection{实验重点(3分)}
(简述本实验的学习重点,不超过100字。)

1.\textbf{原理掌握:} 理解“四端接入法”消除接触电阻的物理机制,以及双臂电桥中两个比率臂($R_1/R_2$ 与 $R_3/R_4$)必须同步调节的必要性。

2.\textbf{仪器操作:} 掌握 QJ-44 型直流双臂电桥的使用方法,包括正确接线(C端在外,P端在内)、倍率选择以及“先 B 后 G”的操作规范。

3.\textbf{结果测量:} 学会测量低值电阻,并能结合温控装置测量并计算金属的电阻温度系数 $\alpha$。

\subsection{实验难点(2分)}
(简述本实验的实现难点,不超过100字。)

1.\textbf{接线要求:} 实验对连线要求较高,待测电阻必须严格按照四端接法接入(电流端与电位端不能接反或短接),且连接点必须紧固,否则极小的接触电阻波动都会引起检流计大幅偏转。

2.\textbf{热平衡控制:} 在测量温度系数时,加热器、介质(油)与金属电阻之间存在热惯性。若升温过快,温度计读数与电阻实际温度不一致,会导致 $R-t$ 关系偏离线性,产生系统误差。

\vspace{3\baselineskip}

\input{注意事项}
\end{document}
