\documentclass{Preport}
\usepackage{longtable}
\usepackage{multirow}      % 用于跨行
\usepackage{float} % 在导言区添加
\usepackage{amsmath}  % 必须加这一行!
\usepackage{ctex}     % 如果要显示中文,加上这一行

\usepackage{hyperref}
\hypersetup{
    colorlinks=true,
}

\exname{非平衡电桥} %实验名称
\instructor{谭艾林老师} %指导教师
\class{-} %班级
\name{-} %姓名
\stuid{-} %学号

\nyear{2025} %年
\nmonth{11} %月
\nday{24} %日
\nweekday{一} %星期几

\begin{document}
\setcounter{page}{0}
\makecover

\section{预习报告(10分)}
\subsection{实验综述(5分)}
(自述实验现象、实验原理和实验方法,包括必要的光路图、电路图、公式等。不超过500字。)

\textbf{非平衡直流电桥实验}旨在帮助我们掌握非平衡电桥的工作原理,并利用它来测量金属(铜)的电阻温度系数。与平衡电桥不同的是,非平衡电桥不仅能测量相对稳定的物理量,更能将传感器(如热敏电阻)因环境变化(温度、压力等)引起的电阻微小变化转化为电压信号输出,是自动控制和非电量电测技术的基础。

\vspace{1\baselineskip}

\textbf{实验现象:}

在实验中,将铜电阻(Cu50)置于加热装置中。随着温度 t 的升高,铜电阻阻值$R_t$增大。

1.在\textbf{非平衡电压法}中,电桥初始处于平衡状态(U=0),随温度升高,电桥失去平衡,数字电压表显示的非平衡电压 U逐渐增大。

2.在\textbf{平衡电桥法}中,随温度升高,检流计偏转,需调节比较臂电阻使电桥重新达到平衡,此时测量出的电阻值$R_t$随温度线性增加。

\vspace{1\baselineskip}

\textbf{实验原理:}

\textbf{1. 非平衡电桥工作原理}

非平衡电桥原理如下图所示,由$R_1,R_2,R_3,R_x$组成桥臂,电源电压为 E:

\begin{figure}[H]
    \centering
    \includegraphics[width=0.6\textwidth]{../images/非平衡电桥原理图.png}
    \caption{非平衡电桥原理图}
    \label{fig:非平衡电桥原理图}
\end{figure}

当负载电阻 $R_g \to \infty $(输出端开路)时,输出电压 
U为 B、D 两点电位差,又根据分压原理,B 点和 D 点对负极(C点)的电压分别为:
\[
U_{BC} = \frac{R_x}{R_1 + R_x} E, \quad U_{DC} = \frac{R_3}{R_2 + R_3} E
\]

则输出电压 $U$ 为:
\begin{equation}
    U = U_{BC} - U_{DC} = \left( \frac{R_x}{R_1 + R_x} - \frac{R_3}{R_2 + R_3} \right) E = \frac{R_2 R_x - R_1 R_3}{(R_1 + R_x)(R_2 + R_3)} E
    \label{eq:general_U}
\end{equation}

当满足 $R_1 R_3 = R_2 R_x$ 时,$U=0$,电桥处于平衡状态。固定$R_1,R_2,R_3$,调节电阻$R_x$为$R_x+ \Delta R_x$,电桥失去平衡,输出的非平衡电压 $U$ 为:
$$
U = \frac{R_2 R_x + R_2 \Delta R_x - R_1 R_3}{(R_1 + R_x + \Delta R_x)(R_2 + R_3)} \cdot E
$$

根据U的变化,我们就能得知电桥中电阻的变化情况。

\textbf{2. 金属电阻温度系数 $\alpha$ 的测量}

变温金属电阻阻值随温度线性变化,即 $R_t = R_0 (1 + \alpha t)$。其中$R_0$为变温电阻0℃时的阻值,$\alpha$为电阻温度系数。

实验中,当B、D处于开路状态时,设定桥臂电阻 $R_1 = R_2 = R_3 = R_0$,此时待测电阻 $R_x = R_t = R_0 + \Delta R_x$,其中 $\Delta R_x = R_0 \alpha t$。

将上述条件代入公式 (1) 得输出电压与温度的关系:
\begin{equation}
    U = \frac{\alpha t}{4 + 2\alpha t} E
\end{equation}

由此可反解出电阻温度系数 $\alpha$ 的计算公式:

\begin{equation}
    \alpha = \frac{4U}{t(E - 2U)}
    \label{eq:alpha_calc}
\end{equation}

本次实验中 $E \approx 1.3\,\text{V}$。因此通过测量不同温度 $t$ 下的非平衡电压 $U$,我们即可求得 $\alpha$。


\vspace{1\baselineskip}

\textbf{实验方法:}

\vspace{1\baselineskip}

\textbf{平衡电桥法(绘 $R_t - t$ 曲线):}

1.将开关置于“平衡 5V”档,此时电路转换为惠斯登电桥

2.设定 $R_1 = R_2$,调节 $R_3$(即测量臂 $R_c$)使检流计示数为 0。

3.此时 $R_x = R_3$。逐点测量不同温度下的阻值,绘制 $R_t - t$ 曲线,验证线性关系并由斜率求 $\alpha$。

\vspace{1\baselineskip}

\textbf{非平衡电压法(测 $\alpha$):}

1.将 FQJ 电桥功能开关置于“非平衡电压”档。

2.按照 $R_1 = R_2 = R_3 = 50\,\Omega$ 设置桥臂,接入 $0^\circ\text{C}$ 时阻值为 $50\,\Omega$ 的 Cu50 铜电阻。

3.\textbf{预调平衡}:若当前非 $0^\circ\text{C}$,需通过微调 $R_3$ 或利用电桥调零功能,模拟 $t = 0$ 时 $U = 0$ 的初始状态(或在数据处理时扣除初始电压)。

4.开启加热器,每隔 $5^\circ\text{C}$ 记录温度 $t$ 和电压 $U$,计算 $\alpha$ 并取平均值。

\subsection{实验重点(3分)}
(简述本实验的学习重点,不超过100字。)

1.\textbf{原理掌握:} 深刻理解非平衡电桥如何将电阻的微小增量 $\Delta R$ 转化为电压信号 $U$,以及公式推导过程。

2.\textbf{参数测量:} 掌握利用公式 $ \alpha = \frac{4U}{t(E - 2U)}$ 计算温度系数的方法。

3.\textbf{仪器操作:} 熟练使用 FQJ 教学电桥的“平衡”与“非平衡”模式切换,以及 PID 温控仪的设定与读数。

\subsection{实验难点(2分)}
(简述本实验的实现难点,不超过100字。)

1.\textbf{热平衡的控制:} 加热过程中铜电阻内部与温控传感器之间存在温度梯度的滞后(热惯性),需控制升温速率,待温度示数稳定后再读取电压或调节电阻,否则会导致U与t不对应,产生较大系统误差。

2.\textbf{接触电阻影响:} 实验连线(特别是$R_x$端)必须紧固,连接导线的不稳定或接触不良会引起电压读数跳动或引入接触电阻,严重影响非平衡电桥微弱电压信号的测量精度。

\vspace{3\baselineskip}

\input{注意事项}
\end{document}
