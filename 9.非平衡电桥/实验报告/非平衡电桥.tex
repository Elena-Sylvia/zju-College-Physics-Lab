\documentclass{Phyport}
\usepackage{longtable}
\usepackage{multirow}      % 用于跨行
\usepackage{float} % 在导言区添加
\usepackage{amsmath}  % 必须加这一行!
\usepackage{ctex}     % 如果要显示中文,加上这一行

\usepackage{hyperref}
\hypersetup{
    colorlinks=true,
}

\exname{非平衡电桥} %实验名称
\extable{4} %实验桌号
\instructor{谭艾林老师} %指导教师
\class{-} %班级
\name{-} %姓名
\stuid{-} %学号

\nyear{2025} %年
\nmonth{11} %月
\nday{24} %日
\nweekday{一} %星期几

\redate{} %如有实验补做,补做日期
\resitu{} %情况说明:

\begin{document}
\setcounter{page}{0}
\makecover

\section{预习报告(10分)}
(注:将已经写好的“物理实验预习报告”内容拷贝过来)

\subsection{实验综述(5分)}
(自述实验现象、实验原理和实验方法,包括必要的光路图、电路图、公式等。不超过500字。)

\textbf{非平衡直流电桥实验}旨在帮助我们掌握非平衡电桥的工作原理,并利用它来测量金属(铜)的电阻温度系数。与平衡电桥不同的是,非平衡电桥不仅能测量相对稳定的物理量,更能将传感器(如热敏电阻)因环境变化(温度、压力等)引起的电阻微小变化转化为电压信号输出,是自动控制和非电量电测技术的基础。

\vspace{1\baselineskip}

\textbf{实验现象:}

在实验中,将铜电阻(Cu50)置于加热装置中。随着温度 t 的升高,铜电阻阻值$R_t$增大。

1.在\textbf{非平衡电压法}中,电桥初始处于平衡状态(U=0),随温度升高,电桥失去平衡,数字电压表显示的非平衡电压 U逐渐增大。

2.在\textbf{平衡电桥法}中,随温度升高,检流计偏转,需调节比较臂电阻使电桥重新达到平衡,此时测量出的电阻值$R_t$随温度线性增加。

\vspace{1\baselineskip}

\textbf{实验原理:}

\textbf{1. 非平衡电桥工作原理}

非平衡电桥原理如下图所示,由$R_1,R_2,R_3,R_x$组成桥臂,电源电压为 E:

\begin{figure}[H]
    \centering
    \includegraphics[width=0.6\textwidth]{../images/非平衡电桥原理图.png}
    \caption{非平衡电桥原理图}
    \label{fig:非平衡电桥原理图}
\end{figure}

当负载电阻 $R_g \to \infty $(输出端开路)时,输出电压 
U为 B、D 两点电位差,又根据分压原理,B 点和 D 点对负极(C点)的电压分别为:
\[
U_{BC} = \frac{R_x}{R_1 + R_x} E, \quad U_{DC} = \frac{R_3}{R_2 + R_3} E
\]

则输出电压 $U$ 为:
\begin{equation}
    U = U_{BC} - U_{DC} = \left( \frac{R_x}{R_1 + R_x} - \frac{R_3}{R_2 + R_3} \right) E = \frac{R_2 R_x - R_1 R_3}{(R_1 + R_x)(R_2 + R_3)} E
    \label{eq:general_U}
\end{equation}

当满足 $R_1 R_3 = R_2 R_x$ 时,$U=0$,电桥处于平衡状态。固定$R_1,R_2,R_3$,调节电阻$R_x$为$R_x+ \Delta R_x$,电桥失去平衡,输出的非平衡电压 $U$ 为:
$$
U = \frac{R_2 R_x + R_2 \Delta R_x - R_1 R_3}{(R_1 + R_x + \Delta R_x)(R_2 + R_3)} \cdot E
$$

根据U的变化,我们就能得知电桥中电阻的变化情况。

\textbf{2. 金属电阻温度系数 $\alpha$ 的测量}

变温金属电阻阻值随温度线性变化,即 $R_t = R_0 (1 + \alpha t)$。其中$R_0$为变温电阻0℃时的阻值,$\alpha$为电阻温度系数。

实验中,当B、D处于开路状态时,设定桥臂电阻 $R_1 = R_2 = R_3 = R_0$,此时待测电阻 $R_x = R_t = R_0 + \Delta R_x$,其中 $\Delta R_x = R_0 \alpha t$。

将上述条件代入公式 (1) 得输出电压与温度的关系:
\begin{equation}
    U = \frac{\alpha t}{4 + 2\alpha t} E
\end{equation}

由此可反解出电阻温度系数 $\alpha$ 的计算公式:

\begin{equation}
    \alpha = \frac{4U}{t(E - 2U)}
    \label{eq:alpha_calc}
\end{equation}

本次实验中 $E \approx 1.3\,\text{V}$。因此通过测量不同温度 $t$ 下的非平衡电压 $U$,我们即可求得 $\alpha$。


\vspace{1\baselineskip}

\textbf{实验方法:}

\vspace{1\baselineskip}

\textbf{平衡电桥法(绘 $R_t - t$ 曲线):}

1.将开关置于“平衡 5V”档,此时电路转换为惠斯登电桥

2.设定 $R_1 = R_2$,调节 $R_3$(即测量臂 $R_c$)使检流计示数为 0。

3.此时 $R_x = R_3$。逐点测量不同温度下的阻值,绘制 $R_t - t$ 曲线,验证线性关系并由斜率求 $\alpha$。

\vspace{1\baselineskip}

\textbf{非平衡电压法(测 $\alpha$):}

1.将 FQJ 电桥功能开关置于“非平衡电压”档。

2.按照 $R_1 = R_2 = R_3 = 50\,\Omega$ 设置桥臂,接入 $0^\circ\text{C}$ 时阻值为 $50\,\Omega$ 的 Cu50 铜电阻。

3.\textbf{预调平衡}:若当前非 $0^\circ\text{C}$,需通过微调 $R_3$ 或利用电桥调零功能,模拟 $t = 0$ 时 $U = 0$ 的初始状态(或在数据处理时扣除初始电压)。

4.开启加热器,每隔 $5^\circ\text{C}$ 记录温度 $t$ 和电压 $U$,计算 $\alpha$ 并取平均值。

\subsection{实验重点(3分)}
(简述本实验的学习重点,不超过100字。)

1.\textbf{原理掌握:} 深刻理解非平衡电桥如何将电阻的微小增量 $\Delta R$ 转化为电压信号 $U$,以及公式推导过程。

2.\textbf{参数测量:} 掌握利用公式 $ \alpha = \frac{4U}{t(E - 2U)}$ 计算温度系数的方法。

3.\textbf{仪器操作:} 熟练使用 FQJ 教学电桥的“平衡”与“非平衡”模式切换,以及 PID 温控仪的设定与读数。

\subsection{实验难点(2分)}
(简述本实验的实现难点,不超过100字。)

1.\textbf{热平衡的控制:} 加热过程中铜电阻内部与温控传感器之间存在温度梯度的滞后(热惯性),需控制升温速率,待温度示数稳定后再读取电压或调节电阻,否则会导致U与t不对应,产生较大系统误差。

2.\textbf{接触电阻影响:} 实验连线(特别是$R_x$端)必须紧固,连接导线的不稳定或接触不良会引起电压读数跳动或引入接触电阻,严重影响非平衡电桥微弱电压信号的测量精度。

\section{原始数据(20分)}
(将有老师签名的“自备数据记录草稿纸”的扫描或手机拍摄图粘贴在下方,完整保留姓名,学号,教师签字和日期。)

\begin{figure}[H]
    \centering
    \includegraphics[width=0.75\textwidth]{../images/originaldata.jpg}
    \caption{original data}
    \label{fig:original_data}
\end{figure}

\section{结果与分析(60分)}
\subsection{数据处理与结果(30分)}
(列出数据表格、选择适合的数据处理方法、写出测量或计算结果。)

\textbf{实验一:测量铜电阻Cu50温度系数}

\begin{longtable}{|c|c|c|c|c|c|c|c|c|}
  \caption{测量铜电阻Cu50温度系数原始数据} \label{tab:测量铜电阻Cu50温度系数原始数据} \\
  \hline
     \textbf{次数} & \textbf{1} & \textbf{2} & \textbf{3} & \textbf{4} & \textbf{5} & \textbf{6} & \textbf{7} & \textbf{8} \\
  \hline
  \textbf{温度$t/^\circ\text{C} $} & 30.0 & 35.0 & 40.0 & 45.0 & 50.0 & 55.0 & 60.0 & 65.0 \\
  \hline
  \textbf{U/mV} & 40.0 & 46.2 & 52.1 & 58.4 & 64.0 & 69.7 & 75.1 & 80.8  \\
  \hline
  \textbf{$\alpha/ ^\circ \text{c}^{-1}$} & 0.004372 & 0.004372 & 0.004357 & 0.004387 & 0.004369 & 0.004368 & 0.004354 & 0.004368  \\
  \hline
\end{longtable}

其中,每个$\alpha$均由公式$\alpha = \frac{4U}{t(E - 2U)}$计算得到,本实验中E的值为1.3V。

因此,我们所测得的$\alpha$的平均值为:
$$
\overline{\alpha}=\frac{\sum \alpha_i}{8} =0.004368\,^\circ \text{C}^{-1}
$$

查找资料我们可以得到,Cu50的温度系数的标准值为$0.004280^\circ \text{C}^{-1}$

故相对误差$E_r$为:
$$E_r=\frac{|\bar{\alpha}-\alpha_{theory}|}{\alpha_{theory}} \times 100\% = \frac{|0.004368-0.004280|}{0.004280} \times 100\% \approx 2.06\%$$

\textbf{实验二:用平衡电桥描绘铜电阻(Cu50)温度特性
曲线}

\begin{longtable}{|c|c|c|c|c|c|c|c|c|}
  \caption{平衡电桥原始数据} \label{tab:平衡电桥原始数据} \\
  \hline
     \textbf{次数} & \textbf{1} & \textbf{2} & \textbf{3} & \textbf{4} & \textbf{5} & \textbf{6} & \textbf{7} & \textbf{8} \\
  \hline
  \textbf{温度$t/^\circ\text{C} $} & 30.0 & 35.0 & 40.0 & 45.0 & 50.0 & 55.0 & 60.0 & 65.0 \\
  \hline
  \textbf{$R_t/\Omega$} & 56.99 & 58.06 & 59.15 & 60.23 & 61.31 & 62.37 & 63.46 & 64.56  \\
  \hline
\end{longtable}

使用matplotlib对数据进行线性拟合,得到拟合直线为:

\begin{figure}[H]
    \centering
    \includegraphics[width=0.75\textwidth]{../images/output.png}
    \caption{output}
    \label{fig:output}
\end{figure}

线性拟合结果:

- 拟合方程: $R_t = 0.2160 * t + 50.5051$

- 截距$R_0 = 50.5051 \Omega $

- 斜率k = $0.2160 \,\Omega/^\circ \text{C}$

因此我们测得的温度系数为:$\alpha = \frac{k}{R_0} = 0.004277 ^\circ \text{C} $ ,相对误差$E_r$为:
$$
E_r=\frac{|\bar{\alpha}-\alpha_{theory}|}{\alpha_{theory}} \times 100\% = \frac{|0.004277-0.004280|}{0.004280} \times 100\% \approx 0.07\%
$$

\subsection{误差分析(20分)}
(运用测量误差、相对误差或不确定度等分析实验结果,写出完整的结果表达式,并分析误差原因。)

\textbf{相对误差在实验数据处理部分已经计算完毕,本部分直接分析实验误差来源。}

\vspace{1\baselineskip}

在本实验中,我们可以发现,非平衡电桥法的相对误差显著大于平衡电桥法,这主要源于以下几个方面的误差:

\textbf{系统误差(主要影响非平衡电桥法)}

1.\textbf{电源电压E的准确性}:非平衡电桥法中,$\alpha$的计算高度依赖于电源电压E的准确值。而实际测量过程中,电源电压的大小会随负载或时间产生微小漂移,这会直接对$\alpha$的计算结果产生较大影响。

2.\textbf{预调平衡的偏差:}非平衡法要求在 $t=0^\circ \text{C}$时电桥平衡(U=0)。而在实际操作中,我们是在室温下通过调零旋钮来抵消初始电压。这种“电气调零”与真实的“$0^\circ \text{C}$电阻平衡”存在差异,若调零不准,会给后续所有电压读数引入一个固定的系统偏差。

\textbf{环境与操作误差(同时影响两种方法)}

1.\textbf{温度测量误差:}实验中温度传感器与铜电阻之间存在热接触不良或热滞后现象,导致实际铜电阻温度与读数存在偏差,进而影响$\alpha$的计算。同时,实验中我们可以观察到加热的速度较快,我是采用距离设定温度还有一定差距时即停止加热,使用余热使铜电阻温度达到设定值。但是铜电阻的温度仍然存在波动,且加热装置内部温度可能不均,也会对测量结果产生影响。

2.\textbf{接触电阻与连接线不稳定:}导线与FDJ型非平衡直流电桥连接端口的接触电阻会引入额外的电压下降,,这种接触不良会导致读数波动,增加实验的随机误差。

3.\textbf{电压表读数误差}数字电压表的读数存在一定的分辨率限制,存在$\pm0.1mV$的跳动,在测量微小电压时,读数的不确定性会对结果产生较大影响。

\subsection{实验探讨(10分)}
(对实验内容、现象和过程的小结,不超过100字。)

本实验利用 FQJ 型直流电桥,分别采用非平衡电压法和平衡电桥法测量了铜电阻(Cu50)的温度系数。实验发现,非平衡法中,输出电压U随温度升高显著增大,计算得$\alpha \approx 0.004368\,^\circ \text{C}^{-1}$,误差约2.06\%;平衡法通过绘制$R_t - t$曲线,得$\alpha \approx 0.004277\,^\circ \text{C}^{-1}$,误差仅0.07\%。两种方法均验证了铜电阻阻值随温度线性增加的特性。

\section{思考题(10分)}
(解答教材或讲义或老师布置的思考题,请先写题干,再作答。)

\textbf{思考题一:简述非平衡电桥与平衡电桥之间的区别。}

非平衡电桥与平衡电桥虽然电路结构相似(均为惠斯登电桥结构),但在工作原理、测量方法和应用场景上有显著区别:

\begin{enumerate}
    \item \textbf{测量原理不同(零示法 vs 偏转法):}
    \begin{itemize}
        \item \textbf{平衡电桥}采用“零示法”。测量时必须调节比较臂电阻($R_3$),使检流计示数为零($I_g=0$),此时 B、D 两点电势相等。计算公式为 $R_x = \frac{R_1}{R_2} R_3$。
        \item \textbf{非平衡电桥}采用“偏转法”。测量时桥臂电阻($R_1, R_2, R_3$)固定不变,电桥不处于平衡状态。它通过测量 B、D 间的输出电压 $U$ 来反推待测电阻 $R_x$ 的变化。
    \end{itemize}

    \item \textbf{操作方式与测量速度不同:}
    \begin{itemize}
        \item 平衡电桥在每次测量时都需要手动或自动调节电阻使电桥平衡,操作较繁琐,且无法实时跟随快速变化的信号。
        \item 非平衡电桥一旦预调平衡后,无需再调节电阻,可直接读取输出电压。其响应速度快,能够连续测量和记录。
    \end{itemize}

    \item \textbf{应用场景不同:}
    \begin{itemize}
        \item 平衡电桥适用于测量静态、固定且对精度要求较高的电阻值。
        \item 非平衡电桥适用于测量连续变化的物理量(如随时间变化的温度、压力等),常用于自动控制和非电量电测系统中。
    \end{itemize}
\end{enumerate}

\textbf{思考题二:非平衡电桥在工程中有些哪些应用?举例说明。}

非平衡电桥在工程中主要作为传感器信号的转换电路,将非电物理量引起的电阻的变化转化为电压或电流信号,从而实现对非电量的测量和控制。具体的应用包括:

\begin{enumerate}
    \item \textbf{温度测量与控制(热敏电阻/热电阻):}
    \begin{itemize}
        \item \textbf{原理:} 利用金属热电阻(如本实验中的 Cu50)或半导体热敏电阻的阻值随温度变化的特性。
        \item \textbf{举例:} 在工业生产中监测电机内部线圈的温度;或如家用空调、热水器的温度控制器,通过非平衡电桥读取热敏电阻的变化来控制加热或制冷。
    \end{itemize}

    \item \textbf{应力与称重测量(电阻应变片):}
    \begin{itemize}
        \item \textbf{原理:} 某些材料(如康铜丝)在受力发生形变时,其电阻率和几何尺寸会发生变化,导致电阻改变(应变效应)。
        \item \textbf{举例:} 电子秤(称重传感器)、桥梁建筑的结构健康监测、飞机机翼的受力分析。将应变片接入非平衡电桥,将其微小的形变转化为电压信号输出。
    \end{itemize}

    \item \textbf{压力测量:}
    \begin{itemize}
        \item \textbf{原理:} 压力传感器利用压阻效应,当气体或液体压力作用于传感器膜片时,引起电阻变化,通过非平衡电桥测量电压的变化情况,从而得知压力大小。
        \item \textbf{举例:} 广泛用于气压计、油压监测等。
    \end{itemize}
\end{enumerate}

\vspace{2\baselineskip}

\input{注意事项}
\end{document}
