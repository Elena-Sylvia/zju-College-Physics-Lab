\documentclass{Preport}

\usepackage{hyperref}
\hypersetup{
    colorlinks=true,
}

\exname{金属材料杨氏模量的测定} %实验名称
\instructor{张贯乔老师} %指导教师
\class{-} %班级
\name{-} %姓名
\stuid{-} %学号

\nyear{2025} %年
\nmonth{10} %月
\nday{13} %日
\nweekday{一} %星期几

\begin{document}
\setcounter{page}{0}
\makecover

\section{预习报告(10分)}
\subsection{实验综述(5分)}
(自述实验现象、实验原理和实验方法,包括必要的光路图、电路图、公式等。不超过500字。)

金属材料测定杨氏模量实验,旨在通过使用拉伸法来研究材料的弹性性质,从而理解应力与应变之间的线性关系,并利用光杠杆原理放大微小的形变,实现高精度测量,从而求得金属丝的杨氏模量。该实验不仅验证了胡克定律在弹性形变范围内的适用性,还帮助我们熟悉了光学放大测量的方法以及逐差法等误差分析方法。

\textbf{实验现象:}当逐步增大悬挂在金属丝下端的砝码个数时,金属丝随拉力增加而逐渐伸长,当移去砝码后长度恢复。通过望远镜观察光杠杆反射点的微小位移,我们可以直观地看到随载荷变化的微小形变,且在弹性限度之内形变与所加拉力基本成正比。

\textbf{实验原理:}在弹性形变范围内,金属丝的应力与应变满足胡克定律:
$$
E = \frac{\sigma}{\varepsilon} = \frac{F L}{S \Delta L}
$$

其中,E为杨氏模量,F为拉力,L为金属丝原长,$𝑆=\frac{\pi d^2}{4}$为横截面积,$\Delta L$为伸长量。

在实验中通过光杠杆原理放大微小的$\Delta L$:
$$
\Delta L = \frac{b}{2D} \Delta s
$$
其中b为光杠杆短臂,D为望远镜与镜面间距离,$\Delta s$为从标尺上读出的位移。

综合以上两式我们可以得出:
$$
E = \frac{8 m g L D}{\pi d^2 b \Delta s}
$$
通过使用逐差法测定不同载荷下的位移差并进行线性拟合,即可求得杨氏模量。

\textbf{实验装置:}本次实验实验仪由双柱支架、金属丝、上下夹具、光杠杆镜、读数望远镜、螺旋测微计、游标卡尺及米尺等组成。

光路结构为:光源→光杠杆镜→望远镜→刻度尺。

\textbf{实验步骤:}

\begin{enumerate}[leftmargin=4em]
    \item 调平支架并校准光杠杆系统
    \item 测量金属丝长度、直径、光杠杆臂长及镜距
    \item 逐级增加砝码,记录望远镜中观测得到的标尺位移
    \item 采用逐差法求平均伸长量,计算杨氏模量并分析本次实验的不确定度
    
\end{enumerate}

\subsection{实验重点(3分)}
(简述本实验的学习重点,不超过100字。)

理解杨氏模量的物理意义以及胡克定律在弹性限度内的应用;掌握利用光杠杆来放大微小形变的测量方法;能使用逐差法与线性拟合的方法来处理实验数据,计算金属丝的杨氏模量,并计算出本次实验中的不确定度。

\subsection{实验难点(2分)}
(简述本实验的实现难点,不超过100字。)

在本次实验中,准确测量微小的伸长量以及金属丝的直径是实验的难点,需要精确调节光杠杆系统、消除视差与振动的影响;此外,光路调整、仪器读数误差及金属丝非线性形变都会影响杨氏模量的计算精度。

\input{注意事项}
\end{document}
