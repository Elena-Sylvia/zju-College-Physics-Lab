\documentclass{Phyport}
\usepackage{longtable}
\usepackage{multirow}      % 用于跨行
\usepackage{float} % 在导言区添加

\usepackage{hyperref}
\hypersetup{
    colorlinks=true,
}

\exname{金属材料杨氏模量的测定} %实验名称
\extable{2} %实验桌号
\instructor{张贯乔老师} %指导教师
\class{-} %班级
\name{-} %姓名
\stuid{-} %学号

\nyear{2025} %年
\nmonth{10} %月
\nday{13} %日
\nweekday{一} %星期几

\redate{} %如有实验补做,补做日期
\resitu{} %情况说明:

\begin{document}
\setcounter{page}{0}
\makecover

\section{预习报告(10分)}
(注:将已经写好的“物理实验预习报告”内容拷贝过来)

\subsection{实验综述(5分)}
(自述实验现象、实验原理和实验方法,包括必要的光路图、电路图、公式等。不超过500字。)

金属材料测定杨氏模量实验,旨在通过使用拉伸法来研究材料的弹性性质,从而理解应力与应变之间的线性关系,并利用光杠杆原理放大微小的形变,实现高精度测量,从而求得金属丝的杨氏模量。该实验不仅验证了胡克定律在弹性形变范围内的适用性,还帮助我们熟悉了光学放大测量的方法以及逐差法等误差分析方法。

\textbf{实验现象:}当逐步增大悬挂在金属丝下端的砝码个数时,金属丝随拉力增加而逐渐伸长,当移去砝码后长度恢复。通过望远镜观察光杠杆反射点的微小位移,我们可以直观地看到随载荷变化的微小形变,且在弹性限度之内形变与所加拉力基本成正比。

\textbf{实验原理:}在弹性形变范围内,金属丝的应力与应变满足胡克定律:
$$
E = \frac{\sigma}{\varepsilon} = \frac{F L}{S \Delta L}
$$

其中,E为杨氏模量,F为拉力,L为金属丝原长,$𝑆=\frac{\pi d^2}{4}$为横截面积,$\Delta L$为伸长量。

在实验中通过光杠杆原理放大微小的$\Delta L$:
$$
\Delta L = \frac{b}{2D} \Delta s
$$
其中b为光杠杆短臂,D为望远镜与镜面间距离,$\Delta s$为从标尺上读出的位移。

综合以上两式我们可以得出:
$$
E = \frac{8 m g L D}{\pi d^2 b \Delta s}
$$
通过使用逐差法测定不同载荷下的位移差并进行线性拟合,即可求得杨氏模量。

\textbf{实验装置:}本次实验实验仪由双柱支架、金属丝、上下夹具、光杠杆镜、读数望远镜、螺旋测微计、游标卡尺及米尺等组成。

光路结构为:光源→光杠杆镜→望远镜→刻度尺。

\textbf{实验步骤:}

\begin{enumerate}[leftmargin=4em]
    \item 调平支架并校准光杠杆系统
    \item 测量金属丝长度、直径、光杠杆臂长及镜距
    \item 逐级增加砝码,记录望远镜中观测得到的标尺位移
    \item 采用逐差法求平均伸长量,计算杨氏模量并分析本次实验的不确定度
    
\end{enumerate}

\subsection{实验重点(3分)}
(简述本实验的学习重点,不超过100字。)

理解杨氏模量的物理意义以及胡克定律在弹性限度内的应用;掌握利用光杠杆来放大微小形变的测量方法;能使用逐差法与线性拟合的方法来处理实验数据,计算金属丝的杨氏模量,并计算出本次实验中的不确定度。

\subsection{实验难点(2分)}
(简述本实验的实现难点,不超过100字。)

在本次实验中,准确测量微小的伸长量以及金属丝的直径是实验的难点,需要精确调节光杠杆系统、消除视差与振动的影响;此外,光路调整、仪器读数误差及金属丝非线性形变都会影响杨氏模量的计算精度。

\section{原始数据(20分)}
(将有老师签名的“自备数据记录草稿纸”的扫描或手机拍摄图粘贴在下方,完整保留姓名,学号,教师签字和日期。)

\begin{figure}[H]
    \centering
    \includegraphics[width=0.85\textwidth]{../images/originaldata.jpg}
    \caption{originaldata}
    \label{fig:originaldata}
\end{figure}

\section{结果与分析(60分)}
\subsection{数据处理与结果(30分)}
(列出数据表格、选择适合的数据处理方法、写出测量或计算结果。)

\textbf{步骤一:基础物理量测量}

1.使用螺旋测微计测量金属丝直径d

2.使用钢直尺测量金属丝原长L

3.使用卷尺测量标尺到光杠杆镜面距离D

4.使用游标卡尺测量光杠杆短臂长b

\textbf{其中,测量金属丝直径d时,使用的螺旋测微计存在-0.061的误差,因此下表中记录的金属丝直径为螺旋测微计的读数加上原始误差}

\textbf{原始数据记录一:}

\begin{longtable}{|c|c|c|c|c|}
  \caption{基础物理量测量数据} \label{tab:基础物理量测量} \\
  \hline
    & \textbf{标尺到光杠杆镜面距离D(cm)} & \textbf{短臂长b(cm)} & \textbf{金属丝原长L(cm)} & \textbf{金属丝直径d(mm)}  \\
  \hline
  \textbf{1} & 134.78 & 7.666 & 108.85 & 0.651 \\
  \hline
  \textbf{2} & 134.51 & 7.668 & 108.76 & 0.646 \\
  \hline
  \textbf{3} & 134.57 & 7.666 & 108.81 & 0.648 \\
  \hline
  \textbf{4} & 134.70 & 7.666 & 108.73 & 0.651 \\
  \hline
  \textbf{5} & 134.52 & 7.668 & 108.79 & 0.648 \\
  \hline
  \textbf{6} & 134.61 & 7.668 & 108.68 & 0.650 \\
  \hline
  \textbf{平均值}  & 134.615 & 7.667 & 108.77 & 0.649 \\
  \hline
  \textbf{$\Delta$仪}  & $\pm 0.05$ & $\pm 0.002$& $\pm 0.05$ & $\pm 0.004$ \\
  \hline
  \textbf{$\Delta_A$} & 0.0434 & 0.000447 & 0.0246 & 0.000816 \\
  \hline
  \textbf{$\Delta_B$} & 0.0289 & 0.00116 & 0.0289 & 0.00231 \\
  \hline
  \textbf{$\Delta$}=$\sqrt{ \Delta_A^2 + \Delta_B^2}$  & 0.0521 & 0.00124 & 0.0379 & 0.00245 \\
  \hline
  \textbf{修正值}  & $134.615 \pm 0.0521$ & $7.667 \pm 0.00124$ & $108.770 \pm 0.0379$ & $0.649 \pm 0.00245$ \\
  \hline
\end{longtable}

\textbf{原始数据记录2:}

\textbf{不挂砝码时的初始读数:4.34,全部砝码取下后的末读数:4.01}

\begin{longtable}{|c|c|c|c|c|}
  \caption{实验原始数据} \label{tab:实验原始数据} \\
  \hline
    \textbf{实验次数} & \textbf{砝码(kg)} & \textbf{标尺读数s(增)} & \textbf{标尺读数s(减)} & \textbf{平均值}  \\
  \hline
  \textbf{1} & 1.00 & 4.92 & 4.70 & 4.81\\
  \hline
  \textbf{2} & 2.00 & 5.50 & 5.32 & 5.41\\
  \hline
  \textbf{3} & 3.00 & 6.11 & 5.99 & 6.05 \\
  \hline
  \textbf{4} & 4.00 & 6.66 & 6.59 & 6.63 \\
  \hline
  \textbf{5} & 5.00 & 7.29 & 7.22 & 7.26 \\
  \hline
  \textbf{6} & 6.00 & 7.87 & 7.93 & 7.90 \\
  \hline
  \textbf{7} & 7.00 & 8.40 & 8.53 & 8.47 \\
  \hline
  \textbf{8} & 8.00 & 9.18 & 9.17 & 9.18 \\
  \hline
  \textbf{9} & 9.00 & 9.82 & 9.82 & 9.82 \\
  \hline
\end{longtable}

根据实验讲义上给出的公式:
\begin{equation}
    \Delta s_i = \frac{s_{i+4} - s_i}{4}, \qquad (i = 1,2,3,4)
\end{equation}
\begin{equation}
    \overline{\Delta s} = \frac{1}{4} \sum_{i=1}^{4} \Delta s_i
\end{equation}

代入我在实验中测得的数据:
$$
\Delta s_1 = \frac{7.26 - 4.81}{4} = 0.6125 cm 
$$
$$
\Delta s_2 = \frac{7.90 - 5.41}{4} = 0.6225 cm 
$$
$$
\Delta s_3 = \frac{8.47 - 6.05}{4} = 0.6050 cm 
$$
$$
\Delta s_4 = \frac{9.18 - 6.63}{4} = 0.6375 cm 
$$

因此,逐差法得到的平均值为:
\begin{equation}
    \Delta s = \frac{0.6125 + 0.6225 + 0.6050 + 0.6375}{4} = 0.6194cm
\end{equation}

又样本不确定度:
\begin{equation}
    \Delta_A(\Delta s) = \sqrt{\frac{\sum_{i=1}^{4}(\Delta s_i - \overline{\Delta s})^2}{4\times 3}} = 0.0070cm
\end{equation}

测量$\Delta s$的仪器的允差$\Delta s=\pm 0.2mm$ ,
故$\Delta_B(\Delta s)=\frac{\Delta_{仪} s}{\sqrt{3}}=0.0115cm$

因此,
$$
\Delta(\Delta s) = \sqrt{ \Delta_A(\Delta s)^2 + \Delta_B(\Delta s)^2 }=0.0135cm
$$

$$
\Delta s=0.6194 \pm 0.0135cm
$$

\textbf{杨氏模量的计算:}

1.直接计算:

我们取g=9.8$m/s^2$,则
$$
E = \frac{8 m g L D}{\pi d^2 b \Delta s}=1.827\times10^{11}(Pa)
$$

又测量质量的仪器的允差为$\pm 500mg$,因此:

\begin{equation}
\left(\frac{\Delta E}{E}\right)^2
= \left(\frac{\Delta m}{m}\right)^2
+ \left(\frac{\Delta L}{L}\right)^2
+ \left(\frac{\Delta D}{D}\right)^2
+ \left(2\frac{\Delta d}{d}\right)^2
+ \left(\frac{\Delta b}{b}\right)^2
+ \left(\frac{\Delta (\Delta s)}{\Delta s}\right)^2 .
\label{eq:rel}
\end{equation}

$$
\frac{\Delta E}{E} = 1.361 \times 10^{-2}
$$
$$
\Delta E = 1.361 \times 10^{-2} \times E = 2.485 \times 10^9 Pa
$$

测得的金属材料杨氏模量的修正值为$E=(1.827\pm 0.02485)\times10^{11} Pa$

2.使用matplotlib进行线性拟合

\begin{figure}[H]
    \centering
    \includegraphics[width=0.85\textwidth]{../images/线性拟合.png}
    \caption{线性拟合结果}
    \label{fig:线性拟合}
\end{figure}

从线性拟合结果我们可以看出,
$$
E=\frac{8LD}{\pi d^2b}\times \frac{1}{k}=1.812\times 10^{11} Pa
$$

\subsection{误差分析(20分)}
(运用测量误差、相对误差或不确定度等分析实验结果,写出完整的结果表达式,并分析误差原因。)

不确定度的计算在\textbf{数据处理与结果}部分已经给出。

\textbf{误差原因分析:}

1.实验过程中将每个砝码按照1kg计算,但实际由于磨损等问题,每个砝码的质量并非为精确的1kg,这会对计算的结果带来一定的误差。

2.在读取伸长量时,即使我已经很努力在减小上下以及左右震动,但是在观察刻度时,我仍能观察到整个金属丝在震动,这也会对结果产生一定的影响。

3.从我的数据中可以发现,在一次递增砝码重量与依次递减砝码重量时,统一重量下伸长量相差还是较大的,可能材料发生的不完全是弹性形变,这也会对实验的结果产生一定的影响。

4.光杠杆读数容易受视差、机械振动、光源不稳定和人为读数主观差异影响。虽然使用逐差法可以减小系统误差,但随机波动仍然明显。

5.若金属丝未完全拉直、夹具不垂直或安装松动,以及金属丝直径不均匀,也会引入系统偏差。

\subsection{实验探讨(10分)}
(对实验内容、现象和过程的小结,不超过100字。)

本实验通过测定金属丝在受拉时的微小伸长量,验证了胡克定律。通过这个实验,我掌握了光杠杆放大法和逐差法的使用;同时在这个实验中,我们使用了游标卡尺,螺旋测微器,米尺等测量工具,对于数据的估读以及仪器使用认识更加深刻。最后通过误差分析,分析了影响本实验测量精度的关键因素,帮助我更好的在后续的实验中提高自己的实验准确度与精度。

\section{思考题(10分)}
(解答教材或讲义或老师布置的思考题,请先写题干,再作答。)

\textbf{1.本实验中测量了哪些物理量?分别用什么量具进行测量?有效数字分别是几位?}

本实验共测量了以下物理量:

\begin{longtable}{|c|c|c|c|}
  \caption{实验所测物理量} \label{tab:实验所测物理量} \\
  \hline
    \textbf{物理量} & \textbf{测量仪器} & \textbf{最小估读刻度} & \textbf{有效数字} \\
  \hline
  \textbf{金属丝原长 L} & 钢直尺 & 0.1mm & 5位 \\
  \hline
  \textbf{光杠杆镜面到标尺的距离 D} & 卷尺 & 0.1mm & 5位 \\
  \hline
  \textbf{光杠杆短臂长度 b} & 游标卡尺 & 不估读 & 4位 \\
  \hline
  \textbf{金属丝直径 d} & 螺旋测微计 & 0.001mm & 3位 \\
  \hline
  \textbf{标尺读数 s} & 望远镜中的标尺 & 0.1mm & 3位 \\
  \hline
\end{longtable}

\textbf{2.光杠杆中,增大D和减小b都可以增加放大倍数,那么两者有何不同?是否可以无限放大光杠杆的倍数?}

光杠杆的放大倍数约为 M = 2D/b

因此,增大D(标尺到镜面的距离)或减小b(光杠杆短臂长度)都能使放大倍数增大。

然而,这两种方式的效果和限制并不相同:

1)增大D时,入射光与反射光路径变长,读数灵敏度提高,但若D过大,光斑容易移出标尺范围,且光路受外界振动的影响更明显,整个系统的稳定性下降。

2)减小b时,放大倍数同样提高,但b过小会导致光杠杆支点机械结构不稳定,镜面旋转角度微小变化时难以保持线性关系。

\textbf{因此,光杠杆放大倍数不能无限增大。过大的放大倍数会导致系统灵敏度虽高,但是稳定性变差,会产生较大的系统误差。}

\textbf{3.逐差法、作图法、最小二乘法在处理数据时有何不同?}

\textbf{逐差法:}通过取间隔相同的数据点做差,利用平均差值减少系统误差对结果的影响,其计算简便,适用于线性关系明显且数据点较少的情形。

\textbf{作图法:}将实验数据作图,通过绘制坐标直线求斜率得到物理量,能直观反映数据的线性关系和实验趋势,但读图和取线会引入较大的人为误差,精度较低。

\textbf{最小二乘法:}利用全部数据点求出最佳的拟合直线,使得误差平方和最小。最小二乘法能有效减小随机误差的影响,获得的结果最为客观、精确。

因此,逐差法最为简便,作图法直观但精度一般,而最小二乘法计算复杂但结果最可靠。

\vspace{3\baselineskip}
\input{注意事项}
\end{document}
