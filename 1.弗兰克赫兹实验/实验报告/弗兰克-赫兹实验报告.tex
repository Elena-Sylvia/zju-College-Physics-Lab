\documentclass{Phyport}

\usepackage{hyperref}
\usepackage{longtable}
\usepackage{multirow}      % 用于跨行
\usepackage{float} % 在导言区添加
\hypersetup{
    colorlinks=true,
}

\exname{弗兰克-赫兹实验} %实验名称
\extable{-} %实验桌号
\instructor{张妍老师} %指导教师
\class{-} %班级
\name{-} %姓名
\stuid{-} %学号

\nyear{2025} %年
\nmonth{9} %月
\nday{22} %日
\nweekday{一} %星期几

\redate{} %如有实验补做,补做日期
\resitu{} %情况说明:

\begin{document}
\setcounter{page}{0}
\makecover

\section{预习报告(10分)}
(注:将已经写好的“物理实验预习报告”内容拷贝过来)

\subsection{实验综述(5分)}
(自述实验现象、实验原理和实验方法,包括必要的光路图、电路图、公式等。不超过500字。)

弗兰克-赫兹实验,旨在通过测量氩原子的第一激发电势,验证原子能级的存在,从而帮助我们学习关于原子碰撞激发和测量的方法,并加深对“量子化”概念的认识。

\textbf{实验现象:}当加速电压$U_{G2K}$逐渐增大时,阳极电流$I_A$并不是单调上升,而是出现一系列周期性的峰谷(如图1所示)。每当电子能量达到氩原子的第一激发电势时,二者发生非弹性碰撞,电子的能量骤降,无法克服拒斥电压$U_{G2A}$到达阳极,电流减小;随后随着电压继续升高,电流又重新增大。\textbf{相邻谷值间的电压差即为氩原子的第一激发电势。}

\begin{figure}[ht]
    \centering
    \includegraphics[width=0.5\textwidth]{../images/image1.png}
    \caption{$I_A$随$U_{G2K}$变化曲线}
    \label{fig:image_1_setup}
\end{figure}

\textbf{实验原理:}根据波尔原子理论,原子的能级是分立的。电子从阴极K出发,经加速电压$U_{G2K}$加速,与氩原子发生碰撞。当电子能量小于氩原子激发电势时,只发生弹性碰撞;当电子能量达到氩原子激发电势$U_0$时,发生非弹性碰撞,氩原子由基态跃迁至第一激发态,电子能量损失:
$$
eU_0 = E_2 - E_1
$$
因此,测定相邻极小值的电压差即可得到氩原子第一激发电势,从而验证能级的量子化。

\textbf{实验方法:}实验采用充氩的四极弗兰克-赫兹管,内部包括灯丝F1F2、阴极K、栅极G1、G2和阳极A。

\begin{itemize}[leftmargin=4em]
    \item 灯丝电压 $U_{F1F2}$:加热灯丝,使灯丝发出热电子。
    \item 第一栅极电压 $U_{G1K}$:消除空间电荷对阴极电子发
射的影响,提高发射效率。
    \item 加速电压 $U_{G2K}$:使从阴极K发出
的电子被加速,进入管内与氩原子碰撞。
    \item 拒斥电压 $U_{G2A}$:筛选能量大于$eU_{G2A}$的电子到达
阳极A。
\end{itemize}

通过自动或手动方式调节 $U_{G2K}$的大小,测量并绘制 $I_A - U_{G2K}$ 曲线,并使用逐差法计算氩原子的第一激发电势。

\subsection{实验重点(3分)}
(简述本实验的学习重点,不超过100字。)

掌握原子能级量子化的概念,理解原子能级跃迁与非弹性碰撞的关系,学会通过测量与分析 $I_A - U_{G2K}$ 曲线,使用逐差法求取氩原子的第一激发电势,从而验证分立能级的存在。

\subsection{实验难点(2分)}
(简述本实验的实现难点,不超过100字。)

需要精确控制灯丝、栅极与拒斥电压的合适参数,避免电子管过热或被击穿;同时需要准确测量峰谷电流差,尽可能减小接触电势差带来的误差。

\section{原始数据(20分)}
(将有老师签名的“自备数据记录草稿纸”的扫描或手机拍摄图粘贴在下方,完整保留姓名,学号,教师签字和日期。)
(原始数据扫描版放在本报告的最后两页)

\begin{figure}[ht]
    \centering
    \includegraphics[width=0.8\textwidth]{../images/raw_data1.jpg}
    \caption{实验原始数据1}
    \label{fig:raw_data_1_setup}
\end{figure}
\begin{figure}[ht]
    \centering
    \includegraphics[width=0.8\textwidth]{../images/raw_data2.jpg}
    \caption{实验原始数据2}
    \label{fig:raw_data_2_setup}
\end{figure}

\section{结果与分析(60分)}
\subsection{数据处理与结果(30分)}
(列出数据表格、选择适合的数据处理方法、写出测量或计算结果。)

\textbf{1.手动测量实验结果}

\textbf{实验中的仪器参数:} $U_{F1F2}$=3.10V , $U_{G1K}$=1.25V , $U_{G2A}$=3.5V

\textbf{微电流测量量程:} $× 10^{-9}A$

\textbf{手动测量数据如下:}

\begin{longtable}{|c|c|c|c|c|c|c|c|c|c|c|}
  \caption{手动测量实验数据记录表} \label{tab:manual-data} \\
  \hline
  \textbf{$U_{G2K}(V)$} & 1.0 & 2.0 & 3.0 & 4.0 & 5.0 & 6.0 & 7.0 & 8.0 & 9.0 & 10.0 \\
  \textbf{$I_A(nA)$} & 0 & 4 & 11 & 21 & 32 & 47 & 62 & 76 & 90 & 101 \\
  \hline
  \textbf{$U_{G2K}(V)$} & 11.0 & 12.0 & 13.0 & 14.0 & 15.0 & 16.0 & 17.0 & 18.0 & 19.0 & 20.0 \\
  \textbf{$I_A(nA)$} & 112 & 122 & 131 & 139 & 146 & 152 & 158 & 160 & 158 & 155 \\
  \hline
  \textbf{$U_{G2K}(V)$} & 21.0 & 22.0 & 23.0 & 24.0 & 25.0 & 26.0 & 27.0 & 28.0 & 29.0 & 30.0 \\
  \textbf{$I_A(nA)$} & 157 & 166 & 176 & 185 & 193 & 200 & 205 & 209 & 206 & 197 \\
  \hline
  \textbf{$U_{G2K}(V)$} & 31.0 & 32.0 & 33.0 & 34.0 & 35.0 & 36.0 & 37.0 & 38.0 & 39.0 & 40.0 \\
  \textbf{$I_A(nA)$} & 183 & 179 & 189 & 204 & 219 & 229 & 237 & 244 & 246 & 244 \\
  \hline
  \textbf{$U_{G2K}(V)$} & 41.0 & 42.0 & 43.0 & 44.0 & 45.0 & 46.0 & 47.0 & 48.0 & 49.0 & 50.0 \\
  \textbf{$I_A(nA)$} & 233 & 217 & 204 & 211 & 227 & 246 & 262 & 273 & 280 & 283 \\
  \hline
  \textbf{$U_{G2K}(V)$} & 51.0 & 52.0 & 53.0 & 54.0 & 55.0 & 56.0 & 57.0 & 58.0 & 59.0 & 60.0 \\
  \textbf{$I_A(nA)$} & 283 & 278 & 264 & 244 & 238 & 248 & 268 & 287 & 303 & 313 \\
  \hline
  \textbf{$U_{G2K}(V)$} & 61.0 & 62.0 & 63.0 & 64.0 & 65.0 & 66.0 & 67.0 & 68.0 & 69.0 & 70.0 \\
  \textbf{$I_A(nA)$} & 320 & 323 & 322 & 316 & 300 & 283 & 278 & 289 & 307 & 325 \\
  \hline
  \textbf{$U_{G2K}(V)$} & 71.0 & 72.0 & 73.0 & 74.0 & 75.0 & 76.0 & 77.0 & 78.0 & 79.0 & 80.0 \\
  \textbf{$I_A(nA)$} & 341 & 353 & 362 & 366 & 367 & 361 & 348 & 334 & 329 & 335 \\
  \hline
  \textbf{$U_{G2K}(V)$} & 81.0 & 82.0 & 83.0 & 84.0 & 85.0 & 86.0 & 87.0 & 88.0 & 89.0 & 90.0 \\
  \textbf{$I_A(nA)$} & 349 & 366 & 382 & 396 & 408 & 416 & 420 & 418 & 411 & 400 \\
  \hline
  \textbf{$U_{G2K}(V)$} & 91.0 & 92.0 & 93.0 & 94.0 & 95.0 & 96.0 & 97.0 & 98.0 & 99.0 & 100.0 \\
  \textbf{$I_A(nA)$} & 394 & 396 & 405 & 417 & 433 & 449 & 463 & 474 & 485 & 488 \\
  \hline
\end{longtable}

用python进行数据拟合,得到的曲线如下:
\begin{figure}[H]
    \centering
    \includegraphics[width=0.7\textwidth]{../images/manual_data_plot.png}
    \caption{拟合出的$I_A - U_{G2K}$变化曲线}
    \label{fig:manual_testing_result}
\end{figure}

获得的峰值结果如下:

\begin{longtable}{|c|c|c|c|c|c|c|c|}
  \caption{手动测量拟合峰值点} \label{tab:manual-data-result} \\
  \hline
  \textbf{峰值序号n} & 1 & 2 & 3 & 4 & 5 & 6 & 7\\
  \hline
  \textbf{$U_{G2K}/V$} & 18.2 & 27.9 & 38.9 & 50.3 & 62.2 & 74.4 & 87.2 \\
  \hline
\end{longtable}

使用逐差法计算测定的第一激发电势:
$$
U = \frac{(U_5 + U_6 + U_7)-(U_2 + U_3 + U_4)}{3^2} = 11.86 V
$$

通过线性拟合方法,计算得到接触电势差为:$U_c=5.110 V ± 0.932 V$
\vspace{1cm}

\textbf{2.自动测量实验结果}

\textbf{实验中的仪器参数:} $U_{F1F2}$=3.10V , $U_{G1K}$=1.25V , $U_{G2A}$=3.5V

\textbf{微电流测量量程:} $× 10^{-9}A$

\textbf{自动测量数据如下:}

\begin{longtable}{|c|c|c|c|c|c|c|}
  \caption{自动测量实验数据记录表} \label{tab:auto-data} \\
  \hline
  \multicolumn{7}{|c|}{\textbf{第一组实验数据}} \\ 
  \hline
  \textbf{峰值序号n} & 1 & 2 & 3 & 4 & 5 & 6 \\
  \textbf{$U_{G2K}(V)$} & 27.9 & 38.8 & 50.1 & 61.7 & 74.0 & 86.9 \\
  \hline
  \multicolumn{7}{|c|}{\textbf{第二组实验数据}} \\ 
  \hline
  \textbf{峰值序号n} & 1 & 2 & 3 & 4 & 5 & 6 \\
  \textbf{$U_{G2K}(V)$} & 27.7 & 39.3 & 50.2 & 61.8 & 74.4 & 87.1 \\
  \hline
  \multicolumn{7}{|c|}{\textbf{第三组实验数据}} \\ 
  \hline
  \textbf{峰值序号n} & 1 & 2 & 3 & 4 & 5 & 6 \\
  \textbf{$U_{G2K}(V)$} & 28.1 & 39.2 & 49.5 & 61.9 & 74.6 & 87.1 \\
  \hline
  \multicolumn{7}{|c|}{\textbf{第四组实验数据}} \\ 
  \hline
  \textbf{峰值序号n} & 1 & 2 & 3 & 4 & 5 & 6 \\
  \textbf{$U_{G2K}(V)$} & 27.6 & 38.9 & 50.1 & 62.0 & 73.9 & 87.3 \\
  \hline
  \multicolumn{7}{|c|}{\textbf{第五组实验数据}} \\ 
  \hline
  \textbf{峰值序号n} & 1 & 2 & 3 & 4 & 5 & 6 \\
  \textbf{$U_{G2K}(V)$} & 27.9 & 39.2 & 49.6 & 62.0 & 74.4 & 86.9 \\
  \hline
\end{longtable}

使用逐差法求出每组测得的第一激发电势:
$$
U_1 = 11.76V \quad U_2 = 11.79V \quad U_3 = 11.87V \quad U_4 = 11.89V \quad U_5 = 11.84V
$$

$$
U_{average} = \frac{U_1 + U_2 + U_3 + U_4 + U_5}{5} = 11.83V
$$

\subsection{误差分析(20分)}
(运用测量误差、相对误差或不确定度等分析实验结果,写出完整的结果表达式,并分析误差原因。)

\textbf{1.手动测量实验误差:}

相对误差:
$$
E_1 = \frac{11.86V - 11.61V}{11.61V} \times 100\% = 2.15\%
$$

\textbf{2.自动测量实验误差:}

相对误差:
$$
E_2 = \frac{11.83V - 11.61V}{11.61V} \times 100\% = 1.89\%
$$

不确定度:

取样本标准差(不偏估计):
\[
s=\sqrt{\frac{1}{N-1}\sum_{i=1}^N (U_i-\bar{U})^2}
\]

先计算偏差平方和:
\[
\sum (U_i-\bar{U})^2=(-0.070)^2+(-0.040)^2+(0.040)^2+(0.060)^2+(0.010)^2 =0.014800\ (V^2).
\]

于是
\[
s=\sqrt{\frac{0.014800}{5-1}}=\sqrt{0.003700}\approx 0.05431\ \mathrm{V}.
\]

平均值的a类不确定度为:
\[
u_{\bar{U}}=\frac{s}{\sqrt{N}}=\frac{0.05431}{\sqrt{5}}\approx 0.02429\ \mathrm{V}.
\]

\textbf{误差原因分析:}

1.氩原子的基态与第一激发态之间有两个亚稳态。氩原子的基态电势(电离电势)是 $U_0=15.76V$;两个亚稳态的电势分别是 $U_1'=4.21V$,$U_1'' =4.04V$;第一激发态的电势是 $U_1=2.68eV$;基态与两个亚稳
态的电势差分别是 $\delta U_1'=U_0 -U_1' =11.55V$,$\delta U_1''= U_0- U_2''=11.72V$;两个亚稳态与基态的能量
差分别是$\delta E_1' =11.55eV$,$\delta E_1'' =11.72eV$。

实际碰撞中,可能出现三种跃迁路径的混合。如果测得的“第一激发能”实际上是多个跃迁能量的加权平均,测得值会偏离单一理论值(11.61 eV)

2.电子发射极和加速极材料不同,存在接触电势差。如果修正不足,会导致第一激发电势系统性偏移。

3.测量过程中,$U_{G2K}$一直存在较大的波动,可能对测得的电流造成一定的影响。

4.电流变化较小时,信号可能淹没在噪声中,导致峰值点判读不够准确。

5.我在数据处理时使用的峰位判读/检测算法(平滑窗口、峰检测阈值)会引入一定的不确定度。

\subsection{实验探讨(10分)}

本实验利用氩气弗兰克-赫兹管,观察到输出电流随加速电压呈周期性波动,反映电子与氩原子的非弹性碰撞与氩原子的能级跃迁。通过测量相邻峰谷电压差,得到了第一激发电势并验证原子能级的离散性,同时认识了实验中接触电势差等因素对结果的影响。

\section{思考题(10分)}
(解答教材或讲义或老师布置的思考题,请先写题干,再作答。)

\textbf{思考题:探究拒斥电压$U_{G2A}$大小对测得第一激发电势的影响}

我更改了拒斥电压$U_{G2A}$的大小,分别检测了$U_{G2A} = 0.5V,\;2.5V,\;10V$、;时的峰值变化,结果如下:

\begin{longtable}{|c|c|c|c|c|c|c|c|}
  \caption{不同拒斥电压下的峰值变化} \label{tab:diff_auto-data} \\
  \hline
  \multicolumn{8}{|c|}{\textbf{$U_{G2A}$=0.5V}} \\ 
  \hline
  \textbf{峰值序号n} & 1 & 2 & 3 & 4 & 5 & 6 & 逐差法计算结果\\
  \textbf{$U_{G2K}(V)$} & 28.0 & 39.0 & 50.2 & 61.5 & 74.2 & 86.9 & 11.9\\
  \hline
  \multicolumn{8}{|c|}{\textbf{$U_{G2A}$=2.5V}} \\ 
  \hline
  \textbf{峰值序号n} & 1 & 2 & 3 & 4 & 5 & 6 & 逐差法计算结果\\
  \textbf{$U_{G2K}(V)$} & 28.4 & 38.7 & 50.1 & 61.5 & 74.2 & 87.9 & 11.82\\
  \hline
  \multicolumn{8}{|c|}{\textbf{$U_{G2A}$=10.0V}} \\ 
  \hline
  \textbf{峰值序号n} & 1 & 2 & 3 & 4 & 5 & 6 & 逐差法计算结果\\
  \textbf{$U_{G2K}(V)$} & 27.1 & 38.5 & 49.3 & 61.2 & 74.4 & 86.7 & 11.93\\
  \hline
\end{longtable}

\textbf{分析:}几种不同拒斥电压下得到的$U_0$非常接近,说明 拒斥电压$U_{G2A}$对峰间距(即第一激发电势)的影响很小,但峰的绝对位置和第一个峰对$U_{G2A}$更敏感(变化集中在第1、2、第3峰),且各组峰高和形状会随拒斥电压改变。

\textbf{可能原因:}拒斥电压作为能量筛选器,改变了能被阳极收集的电子能谱 —— 当$U_{G2A}$增大时,低能电子被滤除,导致某些因非弹性碰撞失能的电子不被计数,从而使峰位微移或峰幅减小;同时第一个峰还受接触电势与初始电子能分布影响更大。


\input{注意事项}
\end{document}
