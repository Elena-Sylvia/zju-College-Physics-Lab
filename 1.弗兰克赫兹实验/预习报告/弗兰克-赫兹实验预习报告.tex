%!TEX root = Preport底板.tex
%!TEX program = xelatex
\documentclass{Preport}

\usepackage{hyperref}
\hypersetup{
    colorlinks=true,
}

\exname{弗兰克-赫兹实验} %实验名称
\instructor{张妍老师} %指导教师
\class{-} %班级
\name{-} %姓名
\stuid{-} %学号

\nyear{2025} %年
\nmonth{9} %月
\nday{22} %日
\nweekday{一} %星期几

\begin{document}
\setcounter{page}{0}
\makecover

\section{预习报告(10分)}
\subsection{实验综述(5分)}
(自述实验现象、实验原理和实验方法,包括必要的光路图、电路图、公式等。不超过500字。)

弗兰克-赫兹实验,旨在通过测量氩原子的第一激发电势,验证原子能级的存在,从而帮助我们学习关于原子碰撞激发和测量的方法,并加深对“量子化”概念的认识。

\textbf{实验现象:}当加速电压$U_{G2K}$逐渐增大时,阳极电流$I_A$并不是单调上升,而是出现一系列周期性的峰谷(如图1所示)。每当电子能量达到氩原子的第一激发电势时,二者发生非弹性碰撞,电子的能量骤降,无法克服拒斥电压$U_{G2A}$到达阳极,电流减小;随后随着电压继续升高,电流又重新增大。\textbf{相邻谷值间的电压差即为氩原子的第一激发电势。}

\begin{figure}[ht]
    \centering
    \includegraphics[width=0.5\textwidth]{../images/image1.png}
    \caption{$I_A$随$U_{G2K}$变化曲线}
    \label{fig:setup}
\end{figure}

\textbf{实验原理:}根据波尔原子理论,原子的能级是分立的。电子从阴极K出发,经加速电压$U_{G2K}$加速,与氩原子发生碰撞。当电子能量小于氩原子激发电势时,只发生弹性碰撞;当电子能量达到氩原子激发电势$U_0$时,发生非弹性碰撞,氩原子由基态跃迁至第一激发态,电子能量损失:
$$
eU_0 = E_2 - E_1
$$
因此,测定相邻极小值的电压差即可得到氩原子第一激发电势,从而验证能级的量子化。

\textbf{实验方法:}实验采用充氩的四极弗兰克-赫兹管,内部包括灯丝F1F2、阴极K、栅极G1、G2和阳极A。

\begin{itemize}[leftmargin=4em]
    \item 灯丝电压 $U_{F1F2}$:加热灯丝,使灯丝发出热电子。
    \item 第一栅极电压 $U_{G1K}$:消除空间电荷对阴极电子发
射的影响,提高发射效率。
    \item 加速电压 $U_{G2K}$:使从阴极K发出
的电子被加速,进入管内与氩原子碰撞。
    \item 拒斥电压 $U_{G2A}$:筛选能量大于$eU_{G2A}$的电子到达
阳极A。
\end{itemize}

通过自动或手动方式调节 $U_{G2K}$的大小,测量并绘制 $I_A - U_{G2K}$ 曲线,并使用逐差法计算氩原子的第一激发电势。

\subsection{实验重点(3分)}
(简述本实验的学习重点,不超过100字。)

掌握原子能级量子化的概念,理解原子能级跃迁与非弹性碰撞的关系,学会通过测量与分析 $I_A - U_{G2K}$ 曲线,使用逐差法求取氩原子的第一激发电势,从而验证分立能级的存在。

\subsection{实验难点(2分)}
(简述本实验的实现难点,不超过100字。)

需要精确控制灯丝、栅极与拒斥电压的合适参数,避免电子管过热或被击穿;同时需要准确测量峰谷电流差,尽可能减小接触电势差带来的误差。

\vspace{3\baselineskip}
\input{注意事项}
\end{document}
